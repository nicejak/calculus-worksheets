\documentclass[11pt]{article}
\usepackage[left=1in, right=1in, top=0.75in, bottom=1in]{geometry}
\usepackage{mathexam}
\usepackage{amsmath}
\usepackage{graphicx}
\usepackage[export]{adjustbox} %positioning of images
\usepackage[dvipsnames]{xcolor}
\usepackage{enumitem}
\usepackage{setspace}
\usepackage{latexsym}
\usepackage{wasysym}
\usepackage{amssymb}
\usepackage{pgfplots}
\usepackage{comment}
\usepackage{exsheets}
\usepackage{pdfpages}
\usepackage{multicol}

\ExamClass{Math 242}
\ExamName{Classwork 03}
\ExamHead{Fall 2024}
\fancyfoot{}
\setlength{\headheight}{13.59999pt}

\let\ds\displaystyle
\newcommand{\ddx}{\frac{d}{dx}}
\newcommand{\red}{\textcolor{red}}
\newcommand{\blue}{\textcolor{blue}}
\newcommand{\pink}{\textcolor{CarnationPink}}
\newcommand{\orange}{\textcolor{orange}}
\newcommand{\purple}{\textcolor{purple}}
\newcommand{\violet}{\textcolor{violet}}
\newcommand{\cyan}{\textcolor{cyan}}
\newcommand{\grn}{\textcolor{green}}
\newcommand{\uh}{\textcolor{ForestGreen}}
\newcommand{\bas}[1]{\begin{align*}{#1}\end{align*}}

\pgfplotsset{compat=1.18}

\pgfplotsset%Default tikz axis style
{
    axis lines=center, 
    grid,
    grid style={very thin, densely dotted, black!50},
    xmin=-5,    xmax=5,         xtick distance=1,
    ymin=-5,    ymax=5,         ytick distance=1,
    restrict y to domain=-10:10, % <-------
    ticklabel style={font=\scriptsize, fill=white, inner sep=2pt},
    domain=-5:5, samples=100,
    no marks, 
    every axis plot post/.append style={ultra thick, semitransparent,},
}

\pgfkeys{/pgfplots/Axis Style/.style=
{
    grid style={thin, densely dotted, black!50},
    width=11.5cm, height=5cm,
    axis x line=center, 
    axis y line=middle, 
    samples=100,
    ymin=-1.1, ymax=1.1,
    xmin=-.1, xmax=6.4,
    domain=0:2*pi
}}

\def\myalign#1{%
  \def\trule{\noalign{\smallskip\hrule\medskip}}
  \def\nebc{\nearrow\bigcup}
  \def\sebc{\searrow\bigcup}
  \def\pminf{{}_{-\infty}|^{+\infty}}
  \let\Inf\infty
  \def\amp{&}% props to Bruno; I just love this trick
  \vbox{\mathsurround0pt\openup1\jot
    \halign{%
      &$\displaystyle##\hfil\tabskip0pt$&\amp##\tabskip1em\crcr
      \noalign{\hrule height1pt\smallskip}#1\noalign{\smallskip\hrule height1pt}\crcr}}}

\linespread{1.3}

\begin{document}

    \hrule
    \vspace{.5cm}
    \noindent\textbf{Name:} \underline{\qquad\qquad\qquad\qquad\qquad\qquad\qquad\qquad\qquad\qquad\qquad\qquad\qquad}

    \begin{enumerate}
        \item What is the range of $\displaystyle f(x) =  \arctan(e^x)$?\\[3em]
        
        % \red{If $f(x)=\arctan(e^{x})$, then $\tan(f(x))=e^{x}$.}
        \item Find the exact value of $\displaystyle \arcsin(\sin\frac{ 7\pi}{3} )$.\\[4em]
        \item  Evaluate the integral and simply the answer such that there is no trigonometric, or  inverse trigonometric function
        $$\int \frac{\sin (\arctan(x/2))}{4+x^2}\mathrm{d} x$$
        \newpage
        \item A dead body is found at 4pm. 
        First responders arrive 15 minutes later and they immediately take its temperature, noting that it is 90.5$^\circ$ F.
        When detectives arrive at 5:30pm, the corpse's temperature is 86.5$^\circ$ F. Assume the body was at 98.6$^\circ $F at the time of death.
        The temperature outside is constant 77.6$^\circ$ F. \\
        \begin{itemize}
            \item Newton's law of cooling states that $\frac{dT}{dt} = k(T-T_s)$. 
            For this problem, describe the quantities $T,T_{0}$, $T_{s}$, and $t$.\\[5em]
            %
            \item Use the information you have been given to determine $k$.\\[7em]
            \item What was the time of death? Don't try to calculate the horrible logarithmic expression you get, just describe it as, ``f(t) minutes before . . . ''\\[8em]
        \end{itemize}
        \newpage
        \item Find the limits\\
        \begin{enumerate}
            \item  $\displaystyle \lim_{x\to -\infty} x^2 e^x$\vfill
            \item  $\displaystyle \lim_{x\to 0} \frac{\cos x}{1-x}$\vfill
            % \item $\displaystyle \lim_{x\to 0^+} \ln(\sin x)$\vfill
            % \item $\displaystyle \lim_{x\to 0^} \frac{\sin x - x}{x^3}$\vfill
            \item $\displaystyle \lim_{x\to0^{+}}(1+x)^{\frac{1}{x}}$\vfill
        \end{enumerate}
    \end{enumerate}    

\end{document}