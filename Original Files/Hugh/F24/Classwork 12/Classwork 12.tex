\documentclass[11pt]{article}
\usepackage[left=1in, right=1in, top=0.75in, bottom=1in]{geometry}
\usepackage{mathexam}
\usepackage{amsmath}
\usepackage{graphicx}
\usepackage[export]{adjustbox} %positioning of images
\usepackage[dvipsnames]{xcolor}
\usepackage{enumitem}
\usepackage{setspace}
\usepackage{latexsym}
\usepackage{wasysym}
\usepackage{amssymb}
\usepackage{pgfplots}
\usepackage{comment}
\usepackage{exsheets}
\usepackage{pdfpages}
\usepackage{multicol}

\ExamClass{Math 242}
\ExamName{Classwork 12, \S 11.11}
\ExamHead{Fall 2024}
\fancyfoot{}
\setlength{\headheight}{13.59999pt}

\let\ds\displaystyle
\newcommand{\ddx}{\frac{d}{dx}}
\newcommand{\red}{\textcolor{red}}
\newcommand{\blue}{\textcolor{blue}}
\newcommand{\pink}{\textcolor{CarnationPink}}
\newcommand{\orange}{\textcolor{orange}}
\newcommand{\purple}{\textcolor{purple}}
\newcommand{\violet}{\textcolor{violet}}
\newcommand{\cyan}{\textcolor{cyan}}
\newcommand{\grn}{\textcolor{green}}
\newcommand{\uh}{\textcolor{ForestGreen}}
\newcommand{\bas}[1]{\begin{align*}{#1}\end{align*}}

\pgfplotsset{compat=1.18}

\pgfplotsset%Default tikz axis style
{
    axis lines=center, 
    grid,
    grid style={very thin, densely dotted, black!50},
    xmin=-1,    xmax=2,         xtick distance=.25,
    ymin=-1,    ymax=1,         ytick distance=.125,
    restrict y to domain=-10:10, % <-------
    ticklabel style={font=\scriptsize, fill=white, inner sep=2pt},
    domain=-5:5, samples=100,
    no marks, 
    every axis plot post/.append style={ultra thick, semitransparent,},
}

\pgfkeys{/pgfplots/Axis Style/.style=
{
    grid style={thin, densely dotted, black!50},
    width=11.5cm, height=5cm,
    axis x line=center, 
    axis y line=middle, 
    samples=100,
    ymin=-1.1, ymax=1.1,
    xmin=-.1, xmax=6.4,
    domain=0:2*pi
}}

\def\myalign#1{%
  \def\trule{\noalign{\smallskip\hrule\medskip}}
  \def\nebc{\nearrow\bigcup}
  \def\sebc{\searrow\bigcup}
  \def\pminf{{}_{-\infty}|^{+\infty}}
  \let\Inf\infty
  \def\amp{&}% props to Bruno; I just love this trick
  \vbox{\mathsurround0pt\openup1\jot
    \halign{%
      &$\displaystyle##\hfil\tabskip0pt$&\amp##\tabskip1em\crcr
      \noalign{\hrule height1pt\smallskip}#1\noalign{\smallskip\hrule height1pt}\crcr}}}

\linespread{1.3}

\begin{document}

    \hrule
    \vspace{.5cm}
    \noindent\textbf{Name:} \underline{\qquad\qquad\qquad\qquad\qquad\qquad\qquad\qquad\qquad\qquad\qquad\qquad\qquad}

    Consider the function $f(x)=\int_{0}^{x}te^{-t^{3}}dt$. This function can't be written any more simply than this; we don't have a an expression of an antiderivative of $te^{-t^{3}}$, however, we can still work with functions like this.
    \begin{enumerate}
        \item Show that near $x=0$, $f(x)\approx A(x)=\dfrac{x^{2}}{2}-\dfrac{x^{5}}{5}+\dfrac{x^{8}}{16}-\dfrac{x^{11}}{66}$. Note that we can express $e^{x}$ as $$e^{x}=\sum_{n=0}^{\infty}\frac{x^{n}}{n!}.$$
        \vfill
        \item Estimate the error in using this formula to calculate $f(0.3)$ and $f(0.5)$.
        \vfill
        \newpage
        \item Suppose we had a calculator accurate to within 8 digits that we could use to calculate $f(0.3)$. Would it be more accurate to compute this value with the calculator or the polynomial? Why?
        \vfill
        \item We can compare $f(x)$ and it's approximation $A(x)$ by graphing them both and comparing the graphs (since we don't have a symbolic antiderivative for the integrand $te^{-t^{3}}$, graphing functions like this is no trivial task, even for computers). Below is a graph of $f'(x)$. On the same axes, graph the function $A'(x)$ and conclude whether or not $A(x)$ is a good representation of $f(x)$ near 0.
        
        \begin{tikzpicture}
            \begin{axis}[scale=2.0, declare function = {f(\x)=\x*e^(-\x^3);},declare function = {g(\x)=\x-\x^4+.5*\x^7-(1/6)*\x^10;}] % or f(\x)=1/(\x*\x);
                \addplot    {f(x)};
                % \addplot    {g(x)};
            \end{axis}
        \end{tikzpicture}
        \vfill
    \end{enumerate}

\end{document}