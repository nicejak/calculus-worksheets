\documentclass[11pt]{article}
\usepackage[left=1in, right=1in, top=0.75in, bottom=1in]{geometry}
\usepackage{mathexam}
\usepackage{amsmath}
\usepackage{graphicx}
\usepackage[export]{adjustbox} %positioning of images
\usepackage[dvipsnames]{xcolor}
\usepackage{enumitem}
\usepackage{setspace}
\usepackage{latexsym}
\usepackage{wasysym}
\usepackage{amssymb}
\usepackage{pgfplots}
\usepackage{comment}
\usepackage{exsheets}
\usepackage{pdfpages}
\usepackage{multicol}

\ExamClass{Math 242}
\ExamName{Classwork 09: \S 11.3-11.5}
\ExamHead{Fall 2024}
\fancyfoot{}
\setlength{\headheight}{13.59999pt}

\let\ds\displaystyle
\newcommand{\ddx}{\frac{d}{dx}}
\newcommand{\red}{\textcolor{red}}
\newcommand{\blue}{\textcolor{blue}}
\newcommand{\pink}{\textcolor{CarnationPink}}
\newcommand{\orange}{\textcolor{orange}}
\newcommand{\purple}{\textcolor{purple}}
\newcommand{\violet}{\textcolor{violet}}
\newcommand{\cyan}{\textcolor{cyan}}
\newcommand{\grn}{\textcolor{green}}
\newcommand{\uh}{\textcolor{ForestGreen}}
\newcommand{\bas}[1]{\begin{align*}{#1}\end{align*}}
\newcommand{\dsum}[2]{\displaystyle\sum_{#1}^{#2}}
\newcommand{\dsumn}{\displaystyle\sum_{n=1}^{\infty}}

\pgfplotsset{compat=1.18}

\pgfplotsset%Default tikz axis style
{
    axis lines=center, 
    grid,
    grid style={very thin, densely dotted, black!50},
    xmin=-5,    xmax=5,         xtick distance=1,
    ymin=-5,    ymax=5,         ytick distance=1,
    restrict y to domain=-10:10, % <-------
    ticklabel style={font=\scriptsize, fill=white, inner sep=2pt},
    domain=-5:5, samples=100,
    no marks, 
    every axis plot post/.append style={ultra thick, semitransparent,},
}

\pgfkeys{/pgfplots/Axis Style/.style=
{
    grid style={thin, densely dotted, black!50},
    width=11.5cm, height=5cm,
    axis x line=center, 
    axis y line=middle, 
    samples=100,
    ymin=-1.1, ymax=1.1,
    xmin=-.1, xmax=6.4,
    domain=0:2*pi
}}

\def\myalign#1{%
  \def\trule{\noalign{\smallskip\hrule\medskip}}
  \def\nebc{\nearrow\bigcup}
  \def\sebc{\searrow\bigcup}
  \def\pminf{{}_{-\infty}|^{+\infty}}
  \let\Inf\infty
  \def\amp{&}% props to Bruno; I just love this trick
  \vbox{\mathsurround0pt\openup1\jot
    \halign{%
      &$\displaystyle##\hfil\tabskip0pt$&\amp##\tabskip1em\crcr
      \noalign{\hrule height1pt\smallskip}#1\noalign{\smallskip\hrule height1pt}\crcr}}}

\linespread{1.3}

\begin{document}

    \hrule
    \vspace{.5cm}
    \noindent\textbf{Name:} \underline{\qquad\qquad\qquad\qquad\qquad\qquad\qquad\qquad\qquad\qquad\qquad\qquad\qquad}

    \begin{enumerate}
        \item Use the integral test to show that $\displaystyle\sum_{n=1}^{\infty}\dfrac{n}{n^{2}+1}$ diverges.\vfill
        \item What does the integral test say about the series $\displaystyle\sum_{i=1}^{\infty}i\sin(\pi i)$?\vfill
        \newpage
        \item Write out the first five terms of $\displaystyle\sum_{i=1}^{\infty}i\sin(\pi i)$. Does the series converge or diverge?\vfill
        \item Do your answers to Problems 2 and 3 contradict each other? Explain.\vfill
        \newpage
        \item For each of the following, determine whether the series is convergent or divergent using the direct comparison or limit comparison test.
        \begin{enumerate}
            \item $\dsumn \dfrac{1}{n^{2}+7}$\vfill
            \item $\dsumn \dfrac{n}{n^{2}+7}$\vfill
            \item $\dsumn \dfrac{\ln(n)}{n^{2}+7}$\vfill
        \end{enumerate}
        \newpage
        \item Does the following series converge or diverge? If it converges, what does it converge to? $\dsumn \left(\displaystyle\int_{n}^{n+1}\dfrac{1}{x^{5/3}}dx\right)$ (hint: integrate and write out the first few terms)\vfill\vfill
    \end{enumerate}

\end{document}