\documentclass[11pt]{article}
\usepackage[left=1in, right=1in, top=0.75in, bottom=1in]{geometry}
\usepackage{mathexam}
\usepackage{amsmath}
\usepackage{graphicx}
\usepackage[export]{adjustbox} %positioning of images
\usepackage[dvipsnames]{xcolor}
\usepackage{enumitem}
\usepackage{setspace}
\usepackage{latexsym}
\usepackage{wasysym}
\usepackage{amssymb}
\usepackage{pgfplots}
\usepackage{comment}
\usepackage{exsheets}
\usepackage{pdfpages}
\usepackage{multicol}

\ExamClass{Math 242}
\ExamName{Notes \S11.7: Strategy for testing series}
\ExamHead{Fall 2024}
\fancyfoot{}
\setlength{\headheight}{13.59999pt}

\let\ds\displaystyle
\newcommand{\ddx}{\frac{d}{dx}}
\newcommand{\red}{\textcolor{red}}
\newcommand{\blue}{\textcolor{blue}}
\newcommand{\pink}{\textcolor{CarnationPink}}
\newcommand{\orange}{\textcolor{orange}}
\newcommand{\purple}{\textcolor{purple}}
\newcommand{\violet}{\textcolor{violet}}
\newcommand{\cyan}{\textcolor{cyan}}
\newcommand{\grn}{\textcolor{green}}
\newcommand{\uh}{\textcolor{ForestGreen}}
\newcommand{\bas}[1]{\begin{align*}{#1}\end{align*}}

\pgfplotsset{compat=1.18}

\pgfplotsset%Default tikz axis style
{
    axis lines=center, 
    grid,
    grid style={very thin, densely dotted, black!50},
    xmin=-5,    xmax=5,         xtick distance=1,
    ymin=-5,    ymax=5,         ytick distance=1,
    restrict y to domain=-10:10, % <-------
    ticklabel style={font=\scriptsize, fill=white, inner sep=2pt},
    domain=-5:5, samples=100,
    no marks, 
    every axis plot post/.append style={ultra thick, semitransparent,},
}

\pgfkeys{/pgfplots/Axis Style/.style=
{
    grid style={thin, densely dotted, black!50},
    width=11.5cm, height=5cm,
    axis x line=center, 
    axis y line=middle, 
    samples=100,
    ymin=-1.1, ymax=1.1,
    xmin=-.1, xmax=6.4,
    domain=0:2*pi
}}

\def\myalign#1{%
  \def\trule{\noalign{\smallskip\hrule\medskip}}
  \def\nebc{\nearrow\bigcup}
  \def\sebc{\searrow\bigcup}
  \def\pminf{{}_{-\infty}|^{+\infty}}
  \let\Inf\infty
  \def\amp{&}% props to Bruno; I just love this trick
  \vbox{\mathsurround0pt\openup1\jot
    \halign{%
      &$\displaystyle##\hfil\tabskip0pt$&\amp##\tabskip1em\crcr
      \noalign{\hrule height1pt\smallskip}#1\noalign{\smallskip\hrule height1pt}\crcr}}}

\linespread{1.3}

\begin{document}

    \hrule
    \vspace{.5cm}
    % \noindent\textbf{Name:} \underline{\qquad\qquad\qquad\qquad\qquad\qquad\qquad\qquad\qquad\qquad\qquad\qquad\qquad}
    Here is a brief outline of a strategy to test the series $\sum_{n=1}^{\infty}a_{n}$ for convergence divergence.
    \begin{enumerate}
        \item Does the series obviously diverge? If we suspect that the terms of a series do not trend to zero we can always try the test for divergence before doing any real work. Use immediately if you suspect $\lim\limits_{n\to\infty}a_{n}\neq0$.
        \item Identify one of the three ``types'' of series.
        \begin{enumerate}
            \item Non-negative terms; i.e. $a_{n}\geq0$
            \item Alternating series; i.e. $a_{n}=(-1)^{n}b_{n}$. In this series, every other term has a different sign (positive or negative).
            \item Some positive terms, some negative terms, but we may not be able to discern a pattern.
        \end{enumerate}
        \item For series of type (a), we may have series that are clearly $p$-series or geometric series, or look enough like one of these series that we can do a little bit of algebra and pulling constant multiples out of the series to get it to look exactly like one of these series.
        
        The integral test and our comparison tests will work for series of type (a), but not types (b) or (c).

        Ratio and roots tests may work as well, and we would want to use them if we have some type of factorial that depends on $n$ ($n!$, $(2n)!$, $(n+3)!$, etc.) or if we have $c^{f(n)}$ where $c$ is some constant and $f(n)$ is something like $n$, $2n+1$, etc. (ratio test), or if we have that $a_{n}=(b_{n})^{n}$ (root test).

        Remember that the ratio and roots test can be inconclusive!
        
        \item For series of type (b), our first instinct should be the alternating series test as it's in the name.
        
        \item For series of type (c), test for absolute convergence; i.e. if we have a series $\sum\limits_{n=1}^{\infty}a_{n}$, rewrite as $\sum\limits_{n=1}^{\infty}|a_{n}|$ and simplify (if possible) to get a series of type (a).

        If we can't simplify to a type (a) series, we can always try the root or ratio tests to test for absolute convergence.

        \item Remember that we have requirements for the integral test!

        \item Ratio test will generally be inconclusive for rational functions since $\frac{a_{n+1}}{a{n}}$ will be generally be a rational expression where numerator and denominator have the same degree and the same lead coefficient.
    \end{enumerate}

\end{document}