\documentclass[11pt]{article}
\usepackage[left=1in, right=1in, top=0.75in, bottom=1in]{geometry}
\usepackage{mathexam}
\usepackage{amsmath}
\usepackage{graphicx}
\usepackage[export]{adjustbox} %positioning of images
\usepackage[dvipsnames]{xcolor}
\usepackage{enumitem}
\usepackage{setspace}
\usepackage{latexsym}
\usepackage{wasysym}
\usepackage{amssymb}
\usepackage{pgfplots}
\usepackage{comment}
\usepackage{exsheets}
\usepackage{pdfpages}
\usepackage{multicol}

\ExamClass{Math 242}
\ExamName{Classwork 1: \S6.1, 6.2$^{*}$}
\ExamHead{Fall 2024}
\fancyfoot{}
\setlength{\headheight}{13.59999pt}

\let\ds\displaystyle
\newcommand{\ddx}{\frac{d}{dx}}
\newcommand{\red}{\textcolor{red}}
\newcommand{\blue}{\textcolor{blue}}
\newcommand{\pink}{\textcolor{CarnationPink}}
\newcommand{\orange}{\textcolor{orange}}
\newcommand{\purple}{\textcolor{purple}}
\newcommand{\violet}{\textcolor{violet}}
\newcommand{\cyan}{\textcolor{cyan}}
\newcommand{\grn}{\textcolor{green}}
\newcommand{\uh}{\textcolor{ForestGreen}}

\pgfplotsset{compat=1.18}

\pgfplotsset%Default tikz axis style
{
    axis lines=center, 
    grid,
    grid style={very thin, densely dotted, black!50},
    xmin=-5,    xmax=5,         xtick distance=1,
    ymin=-5,    ymax=5,         ytick distance=1,
    restrict y to domain=-10:10, % <-------
    ticklabel style={font=\scriptsize, fill=white, inner sep=2pt},
    domain=-5:5, samples=100,
    no marks, 
    every axis plot post/.append style={ultra thick, semitransparent,},
}

\pgfkeys{/pgfplots/Axis Style/.style=
{
    grid style={thin, densely dotted, black!50},
    width=11.5cm, height=5cm,
    axis x line=center, 
    axis y line=middle, 
    samples=100,
    ymin=-1.1, ymax=1.1,
    xmin=-.1, xmax=6.4,
    domain=0:2*pi
}}

\def\myalign#1{%
  \def\trule{\noalign{\smallskip\hrule\medskip}}
  \def\nebc{\nearrow\bigcup}
  \def\sebc{\searrow\bigcup}
  \def\pminf{{}_{-\infty}|^{+\infty}}
  \let\Inf\infty
  \def\amp{&}% props to Bruno; I just love this trick
  \vbox{\mathsurround0pt\openup1\jot
    \halign{%
      &$\displaystyle##\hfil\tabskip0pt$&\amp##\tabskip1em\crcr
      \noalign{\hrule height1pt\smallskip}#1\noalign{\smallskip\hrule height1pt}\crcr}}}

\linespread{1.3}

\begin{document}

    \hrule
    \vspace{.5cm}
    \noindent\textbf{Name:} \underline{\qquad\qquad\qquad\qquad\qquad\qquad\qquad\qquad\qquad\qquad\qquad\qquad\qquad}

    \begin{enumerate}
        \item What does it mean for a function to be the \textit{inverse} of another function? Give both an algebraic and a geometric explanation.
        
        \red{Algebraic: A function $f$ is an inverse of a function $g$ if $f(g(x))=x$ for any $x$ in the domain of $g$ and $g(f(x))=x$ for any $x$ in the domain of $f$.}
        

        \red{Geometric: A function $f$ is an inverse of a function $g$ if the graph of $f$ is a reflection across the line $y=x$ of the graph of $g$.}
        
        \item What attribute(s) must a function have to have an inverse? Note that the horizontal line test is not an attribute.
        

        \red{A function will have an inverse if it is injective (one-to-one).}
        
        \item Suppose the function $f$ has inverse $g$. How does $f'(a)$ relate to $g'(a)$ for some number $a$ in the domain of both functions? Write the formula and briefly explain what it means in a few sentences.
        
        \red{$$f'(a)=\frac{1}{g'(f(a))}$$}
        
        \item Algebraically demonstrate that the functions $f(x)=2x^{2}+3$ and $g(x)=\ds\sqrt{\frac{x-3}{2}}$ are inverses of each other. Hint: don't try to find the inverse of $f$ or $g$.
        
        \red{We need to show that $f(g(x))=x$ and that $g(f(x))=x$. First let's show that $f(g(x))=x$.}
        \textcolor{red}{
            \begin{align*}
                f(g(x)) &=  2\left[g(x)\right]^2+3\\
                        &=  2\left(\sqrt{\frac{x-3}{2}}\right)^{2}+3\\
                        &=  2\cdot\frac{x-3}{2}+3\\
                        &=  x-3+3\\
                        &=  x.
            \end{align*}
        }
        \red{Now let's show that $g(f(x))=x$.}
        \textcolor{red}{
            \begin{align*}
                g(f(x)) &=  \sqrt{\frac{f(x)-3}{2}}\\
                        &=  \sqrt{\frac{2x^{2}+3-3}{2}}\\
                        &=  \sqrt{\frac{2x^{2}}{2}}\\
                        &=  \sqrt{x^{2}}\\
                        &=  x.
            \end{align*}
        }

        \item The function $h:[0,\infty)\to\mathbb{R}$ defined by $h(x)=\dfrac{2x}{\sqrt[3]{3x^{3}+4}}$ has an inverse. Find the inverse.

                \red{Set $y=h(x)$, then solve for $x$ in terms of $y$.}

        \textcolor{red}{
            \begin{align*}
                y                                   &=  \frac{2x}{\sqrt[3]{3x^{3}+4}}\\
                y\sqrt[3]{3x^{3}+4}                 &=  2x\\
                y^{3}(3x^{3}+4)                     &=  8x^{3}\\
                y^{3}3x^{3}+4y^{3}                  &=  8x^{3}\\
                4y^{3}                              &=  8x^{3}-y^{3}3x^{3}\\
                4y^{3}                              &=  x^{3}(8-3y^{3})\\
                \frac{4y^{3}}{8-3y^{3}}             &=  x^{3}\\
                \sqrt[3]{\frac{4y^{3}}{8-3y^{3}}}   &=  x.
            \end{align*}
        }

        \red{Now rewrite each $x$ as $y$ and each $y$ as $x$ to get $$y=\sqrt[3]{\frac{4x^{3}}{8-3x^{3}}}.$$ Finally rename $y$ as $h^{-1}(x)$ to get our answer: $$h^{-1}(x)=\sqrt[3]{\frac{4x^{3}}{8-3x^{3}}}.$$}

        \red{One last thing: It's not really important for this problem, but pay attention to the domain here. $h$ is defined on $\mathbb{R}$ except for when the denominator is $0$, which would be when $x=-\sqrt[3]{\frac{4}{3}}$. By restricting our domain to non-negative real numbers we don't have to worry about asymptotes.}
        
        \item If $v(x)=\dfrac{8}{\sqrt{9x^{3}+7}}$, find $(v^{-1})'(2).$
        
        \red{There are a few ways to approach a problem like this. We could find the inverse function of $v(x)$ algebraically then differentiate it and then evaluate it at $x=2$ but that's so much work. What would be easier is to use the formula: $$(f^{-1})'(x)=\frac{1}{f'(f^{-1}(a))}.$$}

        \red{But! We shouldn't rush blindly into this equation all willy-nilly. If we're not careful we're still going to have to invert a function and find a derivative. Instead we should use what we know about functions. We need to find $v^{-1}(2)$, or phrased differently, the number we plug into $v$ to get 2. Since $v$ is a fraction with a numerator of 8, that means the denominator must be 4, so $\sqrt{9x^{3}+7}=4$. The square of 4 is 16, so the radicand is 16: $9x^{3}+7=16$. This tells us that $x=1$, so $v^{-1}(2)=1$.}

        \red{The function $v$ has derivative $v'(x)=-\cdot\dfrac{108x^2}{(\sqrt{9x^3+7})^{3}}$, so $v'(1)=-\dfrac{27}{16}$, giving us a solution of $-\dfrac{16}{27}$.}
        
        \item Differentiate the following expressions:
        \begin{enumerate}
            \item $\ln\left(\dfrac{2x}{1+x^{2}}\right)$

            \red{We have two methods we can use here. The chain rule and the $\ln$ differentiation rule gives us that $\frac{d}{dx}\ln(f(x))=\frac{f'(x)}{f(x)}$. For easier problems this method is fine. Easier is a tricky word though, so after we solve this directly we'll look at a different way.}

            \red{So to solve this directly, we just use what I outlined in the previous paragraph. Let $$f(x)=\dfrac{2x}{1+x^{2}}.$$ Then $$f'(x)=\frac{2-2x^{2}}{(1+x^{2})^{2}},$$ giving us that the derivative that we're looking for is $$\frac{\dfrac{2-2x^{2}}{(1+x^{2})^{2}}}{\dfrac{2x}{1+x^{2}}}=\frac{1-x^{2}}{x(1+x^{2})}.$$}

            \red{But I promised you another way to do this. We can use properties of logarithms to rewrite $\ln\left(\dfrac{2x}{1+x^{2}}\right)$ as $\ln(2x)-\ln(1+x^{2})$. Now we don't have to bother with the quotient rule or division of fractions. Taking this derivative we get $$\frac{d}{dx}\left[\ln(2x)-\ln(1+x^{2})\right]=\frac{2}{2x}-\frac{2x}{2+x^{2}}.$$ taking the difference of these fractions give us the same answer we got with the previous method.}
            
            \item $x^{3}\ln(x^{2})$

            \red{This is fairly straightforward. We use the product rule, power rule, $\ln$ rule, and chain rule to get $$\frac{d}{dx}\left[x^{3}\ln(x^{2})\right]=3x^{2}\ln(x^{2})+\frac{2x^4}{x^{2}}=3x^{2}\ln(x^{2})+2x^{2}.$$}
            
        \end{enumerate}
        \item Use logarithmic differentiation to find the derivative of $y=\dfrac{\sqrt{2x-3}\sin(2x)}{(3-4x)^{3}}$.
        
            \red{Since the natural log is a function, since $y=f(x)$, $\ln(y)=\ln(f(x));$ specifically, $$\ln(y)=\ln\left(\dfrac{\sqrt{2x-3}\sin(2x)}{(3-4x)^{3}}\right).$$}

            \red{Now we need to use properties of logarithms to rewrite the right-hand-side of the equation as the sums and/or differences of multiple natural logarithms. Since we have a fraction, the first thing I do is turn this into a difference of logarithms, making sure to parenthesize the logarithm coming from the denominator (this is done in case the denominator is the product of multiple terms; they will be added together but still be subtracted from the logarithm representing the numerator--it's just easier to get into the habit so you won't make mistakes later): $$\ln(y)=\ln\left(\sqrt{2x-3}\sin(2x)\right)-\left[\ln\left((3-4x)^{3}\right)\right].$$}

            \red{The logarithm which came from the numerator is the product of two terms so we'll have the sum of two logarithms there. The logarithm which came from the denominator is a single term to a power. Note that the power is \textit{inside} of the logarithm and we're not raising the logarithm itself to the third power, so the exponent becomes a coefficient. This is (one of) the use(es) of logarithms. We turn harder math into easier math. Applying these logarithm rules gives us: $$\ln(y)=\ln(\sqrt{2x-3})+\ln\left(\sin(2x)\right)-3\ln(3-4x).$$}

            \red{We can still simplify things a bit more before we start to take derivatives. Remember that the square root function is just us raising something to the $\frac{1}{2}$ power, so we actually get $$\ln(y)=\frac{1}{2}\ln(2x-3)+\ln\left(\sin(2x)\right)-3\ln(3-4x).$$}

            \red{Now we can take the derivatives without much trouble. We don't have to use quotient or product rules and the chain rule stuff we have to do only leaves us with coefficients. Now what we get is: $$\frac{y'}{y}=\frac{1}{2x-3}+\frac{2\cos(2x)}{\sin(2x)}+\frac{12}{3-4x}.$$}

            \red{We know that $y=\dfrac{\sqrt{2x-3}\sin(2x)}{(3-4x)^{3}}$, so we just multiply both sides of the equation by this to get $$y'=\dfrac{\sqrt{2x-3}\sin(2x)}{(3-4x)^{3}}\left(\frac{1}{2x-3}+\frac{2\cos(2x)}{\sin(2x)}+\frac{12}{3-4x}\right).$$}

        \item Evaluate the definite integral $\displaystyle\int_{0}^{1}\frac{7x^2+14}{x^{3}+6x+4}dx$.
        
        \red{We'll have to use $u$-substitution together with our new definition of the natural log function. Remember that $\ln(x)$ is defined as $\displaystyle\int_{1}^{x}\dfrac{1}{t}dt$ (for $x>1$), so the derivative of $\ln(x)$ is $\frac{1}{x}$. This means that $\ddx\ln\left(f(x)\right)=\dfrac{f'(x)}{f(x)}$. So we're looking for a fraction where the numerator is some multiple of the derivative of the denominator. Our denominator has derivative $3x^{2}+6=3(x^{2}+2)$. Our numerator is $7x^{2}+14=7(x^{2}+2)$, meaning our numerator is $\frac{7}{3}$ the derivative of the denominator. This gives us that $u=x^{3}+6x+4$ and $du=\frac{7}{3}dx$}

        \red{Now we need to figure out our new limits of integration. $u(0)=4$ and $u(1)=11$, so our new integral is $$\frac{7}{3}\int_{4}^{11}\frac{1}{u}du.$$}

        \red{This easily integrates to $\frac{7}{3}\left(\ln(11)-\ln(4)\right)$, and we can use our properties of logarithms to rewrite this as $$\frac{7}{3}\ln\left(\frac{11}{4}\right).$$}

        
        
        
    \end{enumerate}

\end{document}