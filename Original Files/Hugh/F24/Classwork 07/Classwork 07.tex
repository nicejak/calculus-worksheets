\documentclass[11pt]{article}
\usepackage[left=1in, right=1in, top=0.75in, bottom=1in]{geometry}
\usepackage{mathexam}
\usepackage{amsmath}
\usepackage{graphicx}
\usepackage[export]{adjustbox} %positioning of images
\usepackage[dvipsnames]{xcolor}
\usepackage{enumitem}
\usepackage{setspace}
\usepackage{latexsym}
\usepackage{wasysym}
\usepackage{amssymb}
\usepackage{pgfplots}
\usepackage{comment}
\usepackage{exsheets}
\usepackage{pdfpages}
\usepackage{multicol}

\ExamClass{Math 242}
\ExamName{Classwork 07: \S 7.8, 11.1}
\ExamHead{Fall 2024}
\fancyfoot{}
\setlength{\headheight}{13.59999pt}

\let\ds\displaystyle
\newcommand{\ddx}{\frac{d}{dx}}
\newcommand{\red}{\textcolor{red}}
\newcommand{\blue}{\textcolor{blue}}
\newcommand{\pink}{\textcolor{CarnationPink}}
\newcommand{\orange}{\textcolor{orange}}
\newcommand{\purple}{\textcolor{purple}}
\newcommand{\violet}{\textcolor{violet}}
\newcommand{\cyan}{\textcolor{cyan}}
\newcommand{\grn}{\textcolor{green}}
\newcommand{\uh}{\textcolor{ForestGreen}}
\newcommand{\bas}[1]{\begin{align*}{#1}\end{align*}}

\pgfplotsset{compat=1.18}

\pgfplotsset%Default tikz axis style
{
    axis lines=center, 
    grid,
    grid style={very thin, densely dotted, black!50},
    xmin=-5,    xmax=5,         xtick distance=1,
    ymin=-5,    ymax=5,         ytick distance=1,
    restrict y to domain=-10:10, % <-------
    ticklabel style={font=\scriptsize, fill=white, inner sep=2pt},
    domain=-5:5, samples=100,
    no marks, 
    every axis plot post/.append style={ultra thick, semitransparent,},
}

\pgfkeys{/pgfplots/Axis Style/.style=
{
    grid style={thin, densely dotted, black!50},
    width=11.5cm, height=5cm,
    axis x line=center, 
    axis y line=middle, 
    samples=100,
    ymin=-1.1, ymax=1.1,
    xmin=-.1, xmax=6.4,
    domain=0:2*pi
}}

\def\myalign#1{%
  \def\trule{\noalign{\smallskip\hrule\medskip}}
  \def\nebc{\nearrow\bigcup}
  \def\sebc{\searrow\bigcup}
  \def\pminf{{}_{-\infty}|^{+\infty}}
  \let\Inf\infty
  \def\amp{&}% props to Bruno; I just love this trick
  \vbox{\mathsurround0pt\openup1\jot
    \halign{%
      &$\displaystyle##\hfil\tabskip0pt$&\amp##\tabskip1em\crcr
      \noalign{\hrule height1pt\smallskip}#1\noalign{\smallskip\hrule height1pt}\crcr}}}

\linespread{1.3}

\begin{document}

    \hrule
    \vspace{.5cm}
    \noindent\textbf{Name:} \underline{\qquad\qquad\qquad\qquad\qquad\qquad\qquad\qquad\qquad\qquad\qquad\qquad\qquad}

    \begin{enumerate}
        \item Answer the following questions about improper integrals:
        \begin{enumerate}
            \item An example of an improper integral of type 1 is: (do not just copy an example from your class notes or textbook)\vfill
            \item An example of an improper integral of type 2 is: (do not just copy an example from your class notes or textbook)\vfill
            \item What does it mean for an improper integral to converge?\vfill
            \item Suppose we have two functions, $f$ and $g$ and for all real numbers $a$, $f(a)>g(a)$. If $\int_{1}^{\infty}g(x)dx$ diverges, what can we say about $\int_{1}^{\infty}f(x)dx$?\vfill
            \item Does $\int_{2}^{\infty}\frac{1}{x-1}dx$ converge or diverge? Why?\vfill
            \item Does $\int_{1}^{\infty}\frac{1}{(x+1)^{2}}dx$ converge or diverge? Why?\vfill
        \end{enumerate}
        \newpage
        \item Integrate the improper integral $$\int_{\frac{1}{2}}^{\infty}\frac{2}{x^{3}}dx.$$
        \vfill
        \item The integral $\displaystyle\int_{0}^{\sqrt{2}}\frac{1}{\sqrt{2-x^{2}}}dx$ is improper because the integrand is undefined at $x=\sqrt{2}$. Solve this integral.
        % \bas
        % {
        %     \int_{0}^{\sqrt{2}}\frac{1}{\sqrt{2-x^{2}}}dx   &=  \lim_{b\to\sqrt{2}}\int_{0}^{b}\frac{1}{\sqrt{2-x^{2}}}dx\\[1em]
        %                                                     &=\lim_{b\to\sqrt{2}}\int_{x=0}^{x=b}\frac{\sqrt{2}\cos(\theta)}{\sqrt{2-2\sin^{2}(\theta)}}d\theta\\[1em]
        %                                                     &=\lim_{b\to\sqrt{2}}\int_{x=0}^{x=b}\frac{\sqrt{2}\cos(\theta)}{\sqrt{2}\cos(\theta)}d\theta\\[1em]
        %                                                     &=\lim_{b\to\sqrt{2}}\int_{x=0}^{x=b}d\theta\\[1em]
        %                                                     &=\lim_{b\to\sqrt{2}}\theta\bigg\vert_{x=0}^{x=b}\\[1em]
        %                                                     &=\lim_{b\to\sqrt{2}}\arcsin\left(\frac{x}{\sqrt{2}}\right)\bigg\vert_{0}^{b}\\[1em]
        %                                                     &=\arcsin\left(\frac{\sqrt{2}}{\sqrt{2}}\right)-0\\[1em]
        %                                                     &=\frac{\pi}{2}
        % }
        \vfill
        \newpage
        \item Determine the convergence or divergence of the following sequences $\{a_{n}\}=f(n)$ by first looking at $f(x)$. Make sure to write out the first few terms of the sequences for each case, to emphasize their discrete nature.
        \begin{enumerate}
            \item $a_{n}=\dfrac{n}{1+n^{2}}$\vfill
            \item $b_{n}=\frac{(n+1)(-1)^{n}}{n}$\vfill
            \newpage
            \item $c_{n}=\sqrt{n+1}-\sqrt{n}$\vfill
            \item $d_{n}=\dfrac{1+n\cos(2\pi n)}{n}$\vfill
        \end{enumerate}
        
    \end{enumerate}

\end{document}