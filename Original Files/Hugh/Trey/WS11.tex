% DOCUMENT FORMATING
\documentclass[12pt]{article}
\usepackage[margin=1in]{geometry}

% PACKAGES
\usepackage{amsmath} % For extended formatting
\usepackage{amssymb} % For math symbols
\usepackage{amsthm} % For proof environment
\usepackage{array} % For tables
\usepackage{enumerate} % For lists
\usepackage{extramarks} % For headers and footers
\usepackage{fancyhdr} % For custom headers
\usepackage{graphicx} % For inserting images
\usepackage{multicol} % For multiple columns
\usepackage{verbatim} % For displaying code
\usepackage{tkz-euclide}
\usepackage{pgfplots}
\usepackage{gensymb}
\usepackage{mathtools}
\usepackage{graphicx}
% SET UP HEADER AND FOOTER


\title{MATH 242 - WS11}
\date{04/11/2024}


\begin{document}
\maketitle


\begin{enumerate}

\item Approximate the numerical value of $e^{\pi}$ using the $2$nd order Taylor polynomial $T_2(x)$ for the function $f(x)=e^x$ centered at $a=0$. How far off is the approximation? How far off does Taylor's Inequality allow (make sure you did better than this worst case guarantee)?
\newpage
\item Approximate the numerical value of $\arctan(1.5)$ using the $2$nd order Taylor polynomial $T_2(x)$ for the function $f(x)=\arctan(x)$ centered at $a=1$. How far off is the approximation? How far off does Taylor's Inequality allow (make sure you did better than this worst case guarantee)?


\end{enumerate}
\end{document}
