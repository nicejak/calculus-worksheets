% DOCUMENT FORMATING
\documentclass[12pt]{article}
\usepackage[margin=1in]{geometry}

% PACKAGES
\usepackage{amsmath} % For extended formatting
\usepackage{amssymb} % For math symbols
\usepackage{amsthm} % For proof environment
\usepackage{array} % For tables
\usepackage{enumerate} % For lists
\usepackage{extramarks} % For headers and footers
\usepackage{fancyhdr} % For custom headers
\usepackage{graphicx} % For inserting images
\usepackage{multicol} % For multiple columns
\usepackage{verbatim} % For displaying code
\usepackage{tkz-euclide}
\usepackage{pgfplots}
\usepackage{gensymb}
\usepackage{mathtools}
\usepackage{graphicx}
% SET UP HEADER AND FOOTER


\title{MATH 242 - Quiz 1}
\date{01/18/2024}


\begin{document}
\maketitle


\begin{enumerate}

\item \textbf{[3 pts]} Consider $g:\{-2,-1,0,1,2\}\to\{1,2,5,10\}$ given by $x\mapsto g(x)=x^2+1$. 

\begin{enumerate}
    \item Notice the function $g(x)$ is NOT onto as it is written. Explain why by providing a new, different co-domain other than $\{1,2,5,10\}$ that would make $g(x)$ an onto function.
    \vspace{1.25cm}
    \item Notice the function $g(x)$ is NOT one-to-one. Explain why by providing a new, different domain other than $\{-2,-1,0,1,2\}$ that would make $g(x)$ a one-to-one function.
    \vspace{1.25cm}
    \item By the previous two questions you have made $g$ invertible (as it is now onto and one-to-one). What is the rule for its inverse function $g^{-1}(x)$?
     \vspace{1.25cm}
\end{enumerate}


\item \textbf{[3 pts]} Let $f(x)=3x^3+x+2$. Without computing the inverse function directly, compute the derivative of the inverse function $(f^{-1})'(-2)$.

\vfill

\pagebreak

\item \textbf{[4 pts]} Use ``Logarithmic Differentiation" to find the derivative $h'(x)$ given that
$$h(x)=\frac{(3x-5)^4\sin(2x)}{(2x^5+7)^3}$$
\end{enumerate}






\end{document}