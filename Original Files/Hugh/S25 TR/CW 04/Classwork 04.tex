\documentclass[11pt]{article}
\usepackage[left=1in, right=1in, top=0.75in, bottom=1in]{geometry}
\usepackage{mathexam}
\usepackage{amsmath}
\usepackage{graphicx}
\usepackage[export]{adjustbox} %positioning of images
\usepackage[dvipsnames]{xcolor}
\usepackage{enumitem}
\usepackage{setspace}
\usepackage{latexsym}
\usepackage{wasysym}
\usepackage{amssymb}
\usepackage{pgfplots}
\usepackage{comment}
\usepackage{exsheets}
\usepackage{pdfpages}
\usepackage{multicol}

\ExamClass{Math 242}
\ExamName{Classwork 04}
\ExamHead{Spring 2025}
\fancyfoot{}
\setlength{\headheight}{13.59999pt}

\let\ds\displaystyle
\newcommand{\ddx}{\frac{d}{dx}}
\newcommand{\red}{\textcolor{red}}
\newcommand{\blue}{\textcolor{blue}}
\newcommand{\pink}{\textcolor{CarnationPink}}
\newcommand{\orange}{\textcolor{orange}}
\newcommand{\purple}{\textcolor{purple}}
\newcommand{\violet}{\textcolor{violet}}
\newcommand{\cyan}{\textcolor{cyan}}
\newcommand{\grn}{\textcolor{green}}
\newcommand{\uh}{\textcolor{ForestGreen}}
\newcommand{\bas}[1]{\begin{align*}{#1}\end{align*}}

\pgfplotsset{compat=1.18}

\pgfplotsset%Default tikz axis style
{
    axis lines=center, 
    grid,
    grid style={very thin, densely dotted, black!50},
    xmin=-5,    xmax=5,         xtick distance=1,
    ymin=-5,    ymax=5,         ytick distance=1,
    restrict y to domain=-10:10, % <-------
    ticklabel style={font=\scriptsize, fill=white, inner sep=2pt},
    domain=-5:5, samples=100,
    no marks, 
    every axis plot post/.append style={ultra thick, semitransparent,},
}

\pgfkeys{/pgfplots/Axis Style/.style=
{
    grid style={thin, densely dotted, black!50},
    width=11.5cm, height=5cm,
    axis x line=center, 
    axis y line=middle, 
    samples=100,
    ymin=-1.1, ymax=1.1,
    xmin=-.1, xmax=6.4,
    domain=0:2*pi
}}

\def\myalign#1{%
  \def\trule{\noalign{\smallskip\hrule\medskip}}
  \def\nebc{\nearrow\bigcup}
  \def\sebc{\searrow\bigcup}
  \def\pminf{{}_{-\infty}|^{+\infty}}
  \let\Inf\infty
  \def\amp{&}% props to Bruno; I just love this trick
  \vbox{\mathsurround0pt\openup1\jot
    \halign{%
      &$\displaystyle##\hfil\tabskip0pt$&\amp##\tabskip1em\crcr
      \noalign{\hrule height1pt\smallskip}#1\noalign{\smallskip\hrule height1pt}\crcr}}}

\linespread{1.3}

\begin{document}

    \hrule
    \vspace{.5cm}
    \noindent\textbf{Name:} \underline{\qquad\qquad\qquad\qquad\qquad\qquad\qquad\qquad\qquad\qquad\qquad\qquad\qquad}

    \begin{enumerate}
        \item Find the limits\\
        \begin{enumerate}
            \item  $\displaystyle \lim_{x\to -\infty} x^2 e^x$\vfill
            \item  $\displaystyle \lim_{x\to 0} \frac{\cos(x)}{1-x}$\vfill
            % \item $\displaystyle \lim_{x\to 0^+} \ln(\sin(x))$\vfill
            \item $\displaystyle \lim_{x\to 0^+} \frac{\sin(x)-x}{x^3}$\vfill
            \item $\displaystyle \lim_{x\to0^{+}}(1+x)^{\frac{1}{x}}$\vfill
        \end{enumerate}
        \newpage
        \item Integrate the following functions:
        \begin{enumerate}
            \item $\displaystyle \int_0^{\pi/2} x^2 \sin x \mathrm{d} x$\vfill
            % {\em Solution:} Use integration by parts, setting $u= x^2$ and $\mathrm{d}v= \sin x \mathrm{d}x$. Then
            % $\mathrm{d}u = 2x\mathrm{d}x$ and $v= -\cos x$. This leaves
            % $$ \int_0^{\pi/2} x^2 \sin x \mathrm{d} x=  \left. -x^2 \cos x\right|_0^{\pi/2} +  \int_0^{\pi/2} 2x \cos x \mathrm{d} x = 0+ 2  \int_0^{\pi/2} x \cos x \mathrm{d} x $$
            % since both endpoints vanish: $\cos(\pi/2) = 0$ and $0^2=0$. Now we apply integration by parts again, setting
            % $u = x$ and $ \mathrm{d} v= \cos x \mathrm{d} x $. Then $\mathrm{d} u = \mathrm{d} x $ and $v= \sin x$. This leaves 
            % $$
            % 2  \int_0^{\pi/2} x \cos x \mathrm{d} x 
            % =
            % 2\left[ \left. x\sin x\right|_0^{\pi/2} - \int_0^{\pi/2} \sin x  \mathrm{d} x \right]
            % =
            % 2\left[ \frac{\pi}{2} +\left. \cos x\right|_0^{\pi/2}\right]
            % = \pi -2.
            % $$
            % \\[1em]
            \item $\displaystyle \int \sin^{-1}(3x) \mathrm{d} x$\vfill
            % {\em Solution:} Start by making a very simple $u$-substitution: $w= 3x$ with $ \mathrm{d}w = 3 \mathrm{d} x $. Then 
            % $$ \int \sin^{-1}(3x) \mathrm{d} x =\frac{1}{3}  \int \sin^{-1}(w) \mathrm{d} w.$$
            % At first glance, this  doesn't seem like an integration by parts problem. But we know a formula for the derivative of arcsine:
            % $$\bigl(\sin^{-1}(w)\bigr)'  = \frac{1}{\sqrt{1-w^2}}.$$
            % So let $u = \sin^{-1}(w)$ and $ \mathrm{d} v =  \mathrm{d} w$. Then $ \mathrm{d} u =  \frac{1}{\sqrt{1-w^2}} \mathrm{d} w$ and $v= w$. Thus,
            % $$ \int \sin^{-1}(3x) \mathrm{d} x =\frac13\left[w \sin^{-1}(w) -  \int\frac{w}{\sqrt{1-w^2}}\mathrm{d} w\right]
            % =\frac13 w \sin ^{-1}(w) +\frac13 \sqrt{1-w^2}.$$
            % Subbing back for $x$ gives
            % $$ \int \sin^{-1}(3x) \mathrm{d} x=x \sin ^{-1}(3x) +\frac13 \sqrt{1-9x^2}$$
            \item $\displaystyle\int\sec^{6}(x)dx$\vfill
            \newpage
            \item $\displaystyle\int_{1}^{2}\dfrac{1}{x^{2}\sqrt{x^{2}+4}}dx$\vfill
            % \item $\displaystyle\int\dfrac{1}{x\sqrt{5-x^{2}}}dx$\vfill
            \item $\displaystyle\int\dfrac{x+1}{\sqrt{(x^{2}+1)^{3}}}dx$\vfill
        \end{enumerate}
    \end{enumerate}

\end{document}