\documentclass[11pt]{article}
\usepackage[left=1in, right=1in, top=0.75in, bottom=1in]{geometry}
\usepackage{mathexam}
\usepackage{amsmath}
\usepackage{graphicx}
\usepackage[export]{adjustbox} %positioning of images
\usepackage[dvipsnames]{xcolor}
\usepackage{enumitem}
\usepackage{setspace}
\usepackage{latexsym}
\usepackage{wasysym}
\usepackage{amssymb}
\usepackage{pgfplots}
\usepackage{comment}
\usepackage{exsheets}
\usepackage{pdfpages}
\usepackage{multicol}

\ExamClass{Math 242}
\ExamName{Classwork 1: \S6.1, 6.2$^{*}$}
\ExamHead{Spring 2025}
\fancyfoot{}
\setlength{\headheight}{13.59999pt}

\let\ds\displaystyle
\newcommand{\red}{\textcolor{red}}
\newcommand{\blue}{\textcolor{blue}}
\newcommand{\pink}{\textcolor{CarnationPink}}
\newcommand{\orange}{\textcolor{orange}}
\newcommand{\purple}{\textcolor{purple}}
\newcommand{\violet}{\textcolor{violet}}
\newcommand{\cyan}{\textcolor{cyan}}
\newcommand{\grn}{\textcolor{green}}
\newcommand{\uh}{\textcolor{ForestGreen}}

\pgfplotsset{compat=1.18}

\pgfplotsset%Default tikz axis style
{
    axis lines=center, 
    grid,
    grid style={very thin, densely dotted, black!50},
    xmin=-5,    xmax=5,         xtick distance=1,
    ymin=-5,    ymax=5,         ytick distance=1,
    restrict y to domain=-10:10, % <-------
    ticklabel style={font=\scriptsize, fill=white, inner sep=2pt},
    domain=-5:5, samples=100,
    no marks, 
    every axis plot post/.append style={ultra thick, semitransparent,},
}

\pgfkeys{/pgfplots/Axis Style/.style=
{
    grid style={thin, densely dotted, black!50},
    width=11.5cm, height=5cm,
    axis x line=center, 
    axis y line=middle, 
    samples=100,
    ymin=-1.1, ymax=1.1,
    xmin=-.1, xmax=6.4,
    domain=0:2*pi
}}

\def\myalign#1{%
  \def\trule{\noalign{\smallskip\hrule\medskip}}
  \def\nebc{\nearrow\bigcup}
  \def\sebc{\searrow\bigcup}
  \def\pminf{{}_{-\infty}|^{+\infty}}
  \let\Inf\infty
  \def\amp{&}% props to Bruno; I just love this trick
  \vbox{\mathsurround0pt\openup1\jot
    \halign{%
      &$\displaystyle##\hfil\tabskip0pt$&\amp##\tabskip1em\crcr
      \noalign{\hrule height1pt\smallskip}#1\noalign{\smallskip\hrule height1pt}\crcr}}}

\linespread{1.3}

\begin{document}

    \hrule
    \vspace{.5cm}
    \noindent\textbf{Name:} \underline{\qquad\qquad\qquad\qquad\qquad\qquad\qquad\qquad\qquad\qquad\qquad\qquad\qquad}

    \begin{enumerate}
        \item What does it mean for a function to be the \textit{inverse} of another function? Give both an algebraic and a geometric explanation.
        \vfill
        \item What attribute(s) must a function have to have an inverse? Note that the horizontal line test is not an attribute, it allows us to test for this attribute.
        \vfill
        \item Suppose the function $f$ has inverse $g$ and $f(a)=b$. How does $f'(a)$ relate to $g'(b)$ for some number $a$ in the domain of $f$? Write the formula and briefly explain what it means in a few sentences.
        \vfill
        \newpage
        \item Algebraically demonstrate that the functions $f(x)=2x^{2}+3$ and $g(x)=\ds\sqrt{\frac{x-3}{2}}$ are inverses of each other. Hint: don't try to find the inverse of $f$ or $g$.
        \vfill
        \item The function $h:[0,\infty)\to\mathbb{R}$ defined by $h(x)=\dfrac{2x}{\sqrt[3]{3x^{3}+4}}$ has an inverse. Find the inverse.
        \vfill
        \item If $v(x)=\dfrac{8}{\sqrt{9x^{3}+7}}$, find $(v^{-1})'(2).$
        \vfill
        \newpage
        \item Differentiate the following expressions:
        \begin{enumerate}
            \item $\ln\left(\dfrac{2x}{1+x^{2}}\right)$
            \vfill
            \item $x^{3}\ln(x^{2})$
            \vfill
        \end{enumerate}
        \item Use logarithmic differentiation to find the derivative of $y=\dfrac{\sqrt{2x-3}\sin(2x)}{(3-4x)^{3}}$.
        \vfill
        \newpage
        \item Evaluate the definite integral $\displaystyle\int_{0}^{1}\frac{7x^2+14}{x^{3}+6x+4}dx$.
    \end{enumerate}

\end{document}