\documentclass[letterpaper,11pt]{article}
\usepackage{amsmath}
\usepackage[letterpaper,margin=1in,includehead=true]{geometry}
\usepackage{comment}
\usepackage{graphicx}
\usepackage{fancyhdr}
\pagestyle{fancy}
\usepackage{color}
\usepackage{setspace}
\usepackage{tikz,tikz-3dplot,pgfplots}
\pgfplotsset{compat=1.10}
\usepgfplotslibrary{fillbetween}
\usepackage{comment}
\usepackage{multicol}
\usepackage{mdframed}
\usepackage{enumitem}

\newmdtheoremenv{definition}{Definition}
\newmdtheoremenv{theorem}{Theorem}
\setlength{\headheight}{15pt}

\newcounter{mycounter}  
\newenvironment{noindlist}
 {\begin{list}{\arabic{mycounter}.~~}{\usecounter{mycounter} \labelsep=0em \labelwidth=0em \leftmargin=0em \itemindent=0em}}
 {\end{list}}

%To print solutions, use \solutionstrue
%To repress solutions, use \solutionsfalse
%\sol takes two arguments. #1 is the vertical length. #2 is the text.

\newif\ifsolutions
\solutionstrue
\ifsolutions
    \newcommand{\sol}[2]{\begin{minipage}[c][#1]{\linewidth}{\textcolor{blue}{\textbf{Solution:}}\quad \textcolor{blue}{#2}}\end{minipage}}
    \newcommand{\opsol}[1]{#1}
    \newcommand{\tblsol}[1]{\textcolor{blue}{#1}}
\else
    \newcommand{\sol}[2]{\begin{minipage}[c][#1]{\linewidth}{\vfill}\end{minipage}}
    \newcommand{\opsol}[1]{0}
    \newcommand{\tblsol}[1]{\textcolor{white}{#1}}
\fi

\newcommand{\unenumerate}[1]{\setcounter{saveenum}{\value{enumi}}\end{enumerate}
	\noindent #1 
	\begin{enumerate} \setcounter{enumi}{\value{saveenum}}}

\newcounter{saveenum}

\def\ds{\displaystyle}

\begin{document}
\lhead{\bf Math 241: Calculus I}
\rhead{\bf Worksheet 24: Volumes} 

\begin{enumerate}
    \item Find the volume of the solid obtained by rotating the region bounded by the given curves about the specified line. Sketch the region, the solid, and a typical disk or washer.
    \begin{align*}
        & y = x^3 \\
        & y = x \\
        & x \geq 0 \\
        & \text{About the $x$-axis.}
    \end{align*}
    \begin{center}
    \begin{tikzpicture}
    \begin{axis}[thick,smooth,no markers,
            xmin=-0.3, xmax=1.3,
            ymin=-1.3, ymax=1.3,
            xtick={-1,-0.5,0,0.5,1},  
            % xticklabels= {,,},
            ytick={-1,-0.5,0,0.5,1},
            % yticklabels= {,,},
            major tick length={0},
            line width=1pt,
            axis lines=center, height=4.875in, width = 3in, grid=major]
            % \addplot [domain=0.01:4.5, samples=100, name path=f, thick, color=red!50]
            % {sqrt(x)} node[above,pos=0.5] {$\sqrt{x}$} ;
            % \addplot [domain=0.4:4.5, samples=100, name path=g, thick, color=blue!50]
            % {x^(-2)} node[right,pos=0.5] {$\frac{1}{x^2}$} ;
            % \addplot[blue!50, opacity=0.3] fill between[of= f and g, soft clip={domain=1:4}];
    \end{axis}
    \end{tikzpicture}
    \end{center}
    \newpage
    $\ $
    \newpage
    \item Find the volume of the solid obtained by rotating the region bounded by the given curves about the specified line. Sketch the region, the solid, and a typical disk or washer.
    \begin{align*}
        & y = x \\
        & y = 0 \\
        & x = 2 \\
        & x = 4 \\
        & \text{About $x=1$.}
    \end{align*}
    \begin{center}
    \begin{tikzpicture}
    \begin{axis}[thick,smooth,no markers,
            xmin=-2.3, xmax=4.3,
            ymin=-0.3, ymax=4.3,
            xtick={-2,...,4},  
            % xticklabels= {,,},
            ytick={0,...,4},
            % yticklabels= {,,},
            major tick length={0},
            line width=1pt,
            axis lines=center, height=2.79in, width = 4in, grid=major]
            % \addplot [domain=0.01:4.5, samples=100, name path=f, thick, color=red!50]
            % {sqrt(x)} node[above,pos=0.5] {$\sqrt{x}$} ;
            % \addplot [domain=0.4:4.5, samples=100, name path=g, thick, color=blue!50]
            % {x^(-2)} node[right,pos=0.5] {$\frac{1}{x^2}$} ;
            % \addplot[blue!50, opacity=0.3] fill between[of= f and g, soft clip={domain=1:4}];
    \end{axis}
    \end{tikzpicture}
    \end{center}
    \newpage
    $\ $
    \newpage
    \item Derive the formula for the volume of a cylinder with base radius $R$ and height $H$.
\end{enumerate}



\end{document}
