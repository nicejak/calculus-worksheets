\documentclass[letterpaper,11pt]{article}
\usepackage{amsmath}
\usepackage{mathtools}
\usepackage[letterpaper,margin=1in,includehead=true]{geometry}
\usepackage{comment}
\usepackage{graphicx}
\usepackage{fancyhdr}
\pagestyle{fancy}
\usepackage{color}
\usepackage{setspace}
\usepackage{tikz,tikz-3dplot,pgfplots}
\usepackage{comment}
\usepackage{multicol}
\usepackage{tabularx}
\usetikzlibrary{patterns}

\DeclarePairedDelimiter\abs{\lvert}{\rvert}

\setlength{\headheight}{15pt}%needed to remove fancyhdr error--not crucial%



\newcounter{mycounter}  
\newenvironment{noindlist}
 {\begin{list}{\arabic{mycounter}.~~}{\usecounter{mycounter} \labelsep=0em \labelwidth=0em \leftmargin=0em \itemindent=0em}}
 {\end{list}}

\newcommand{\unenumerate}[1]{\setcounter{saveenum}{\value{enumi}}\end{enumerate}
	\noindent #1 
	\begin{enumerate} \setcounter{enumi}{\value{saveenum}}}

\newcounter{saveenum}

\def\ds{\displaystyle}

%To print solutions, use \solutionstrue; To hide solutions, use \solutionsfalse.
%\sol takes two arguments. #1 is the vertical length. #2 is the text.

\newif\ifsolutions
\solutionstrue
\ifsolutions
    \newcommand{\sol}[2]{\begin{minipage}[c][#1]{\linewidth}{\textcolor{blue}{\textbf{Solution:}}\quad \textcolor{blue}{#2}}\end{minipage}}
    \newcommand{\opsol}[1]{#1}
    \newcommand{\tblsol}[1]{\textcolor{blue}{#1}}
\else
    \newcommand{\sol}[2]{\begin{minipage}[c][#1]{\linewidth}{\vfill}\end{minipage}}
    \newcommand{\opsol}[1]{0}
    \newcommand{\tblsol}[1]{\textcolor{white}{#1}}
\fi

\newcommand{\ww}{WeBWorK }
\newcommand{\ddx}{\frac{d}{dx}}
\newcommand{\dydx}{\frac{dy}{dx}}
\newcommand{\be}{\begin{enumerate}}
\newcommand{\ee}{\end{enumerate}}
\newcommand{\vs}[1]{\vspace{#1 pt}}

\begin{document}
\lhead{Math 241: Calculus I}
\rhead{\bf Midterm}
\begin{center}
Math 241 Midterm\\
Name: \\
\vspace{1in}
    \renewcommand{\arraystretch}{2}
    Grade Rubric \\ \ \\
    \begin{tabularx}{0.8\textwidth}{|X|X|X|}
        \hline
        1 & 15 & \\
        \hline
        2 & 15 & \\
        \hline
        3 & 15 & \\
        \hline
        4 & 20 & \\
        \hline
        5 & 20 & \\
        \hline
        6 & 15 & \\
        \hline
    \end{tabularx}
    
    \vspace{1in}
    Read each question carefully and answer appropriately. Do \textbf{not} use any digital device during the exam.
\end{center}
\newpage
\begin{enumerate}
    \item Compute the following limit
    \[\lim_{\theta \to \infty} \sin\left(\frac{1}{2\theta}\right)\theta\]
    \newpage
    \item Suppose you are given the curve 
    \[xy = \frac{4x^2}{\sqrt{\sin(\pi x)}}\]
    as well as the points $(1,1)$, $(\frac{1}{2},2)$. What are the tangent lines at these two points?
    \newpage
    \item Recall that a function $f$ is differentiable at a point if $f'(a)$ exists, and is not differentiable otherwise. 
    
    Using the definition of differentiable, prove that $f(x) = |x-6|$ is not differentiable at 6. Furthermore, find a piece-wise formula for $f'$.
    \newpage
    \item
    \begin{enumerate}
        \item The \textit{van der Waals} equation for a fixed $n$ moles of a gas is 
        \[\left(P + \frac{n^2 a}{V^2}\right) (V-nb) = nRT\]
        where $P$ is the pressure, $V$ is the volume, and $T$ is the temperature of the gas. The constant $R$ is the universal gas constant and $a,b$ are positive constants that characterize the particular gas. If $T$ remains constant, find $\frac{dV}{dP}$.
        \vfill
        \item Find the rate of change of volume with respect to pressure of $1$ mole of carbon dioxide at a volume of $V = 10 \text{ L}$ and a pressure of $P = 2.5 \text{ atm}$. Use $a = 4 \text{ L}^2\text{-atm}/\text{mole}^2$ and $b = 0.05 \text{ L/mole}$.
        \vfill
    \end{enumerate}
    \newpage
    \item Water is leaking out of an inverted conical tank at a rate of 10,000 $\text{cm}^3/\text{min}$ at the same time that water is being pumped into the tank at a constant rate. The tank has height 6 m and the diameter at the top is 4 m. If the water level is rising at a rate of 20 cm/min when the height of the water is 2 m, find the rate at which water is being pumped into the tank.
    \newpage
    \item Suppose that we don't have a formula for $g(x)$ but we know that $g(2)=-4$ and $g'(x) = \sqrt{x^2 + 5}$ for all $x$. Use linear approximation to estimate $g(1.95)$.
\end{enumerate}
\end{document}
