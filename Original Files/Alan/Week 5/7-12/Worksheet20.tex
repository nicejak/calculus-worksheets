\documentclass[letterpaper,11pt]{article}
\usepackage{amsmath}
\usepackage[letterpaper,margin=1in,includehead=true]{geometry}
\usepackage{comment}
\usepackage{graphicx}
\usepackage{fancyhdr}
\pagestyle{fancy}
\usepackage{color}
\usepackage{setspace}
\usepackage{tikz,tikz-3dplot,pgfplots}
\usepackage{comment}
\usepackage{multicol}
\usepackage{mdframed}
\usepackage{enumitem}
\usepackage{tabularx}

\newmdtheoremenv{definition}{Definition}
\newmdtheoremenv{theorem}{Theorem}
\setlength{\headheight}{15pt}

\newcounter{mycounter}  
\newenvironment{noindlist}
 {\begin{list}{\arabic{mycounter}.~~}{\usecounter{mycounter} \labelsep=0em \labelwidth=0em \leftmargin=0em \itemindent=0em}}
 {\end{list}}

%To print solutions, use \solutionstrue
%To repress solutions, use \solutionsfalse
%\sol takes two arguments. #1 is the vertical length. #2 is the text.

\newif\ifsolutions
\solutionsfalse
\ifsolutions
    \newcommand{\sol}[2]{\begin{minipage}[c][#1]{\linewidth}{\textcolor{blue}{\textbf{Solution:}}\quad \textcolor{blue}{#2}}\end{minipage}}
    \newcommand{\opsol}[1]{#1}
    \newcommand{\tblsol}[1]{\textcolor{blue}{#1}}
\else
    \newcommand{\sol}[2]{\begin{minipage}[c][#1]{\linewidth}{\vfill}\end{minipage}}
    \newcommand{\opsol}[1]{0}
    \newcommand{\tblsol}[1]{\textcolor{white}{#1}}
\fi

\newcommand{\unenumerate}[1]{\setcounter{saveenum}{\value{enumi}}\end{enumerate}
	\noindent #1 
	\begin{enumerate} \setcounter{enumi}{\value{saveenum}}}

\newcounter{saveenum}

\def\ds{\displaystyle}

\begin{document}
\lhead{Math 241: Calculus I}
\rhead{\bf Worksheet 20: FTC} 
\begin{enumerate}
    \item Use the graph of $f(x)$ below to answer the following:
    \begin{center}
    \begin{tikzpicture}
    \begin{axis}[
        xmin=0, xmax=11,
        ymin=-3, ymax=3,
        xtick={0,...,11},  
        % xticklabels= {,,},
        ytick={-3,...,3},
        % yticklabels= {,,},
        major tick length={0},
        line width=1pt,
        axis lines=center, height=2.74in, width = 5in, grid=major
        ]
        \addplot [samples=100, domain=0:2] {2};
        \addplot [samples=100, domain=2:4] {4-x};
        \addplot [samples=100, domain=4:5] {x-4};
        \addplot [samples=100, domain=5:7] {1};
        \addplot [samples=100, domain=7:11] {8-x};
    \end{axis}
    \end{tikzpicture}
    \end{center}
    \begin{enumerate}
        \item Find $\int_0^3 f(x) \ dx$. Include an illustration of this quantity on the graph.
        \vfill
        \item Complete the following table
        \begin{center}
        \begin{tabular}{|c|c|c|c|c|c|c|c|c|c|c|c|}
            \hline
            $b$ & 0 & 1 & 2 & 3 & 4 & 5 & 6 & 7 & 8 & 9 & 10 \\
            \hline 
            $\displaystyle \int_0^b f(x) \ dx$ & & & & & & & & & & & \\
            \hline
        \end{tabular}
        \end{center}
        \vfill
        \newpage
        \item If $F(x)$ is a function such that $F(0) = 0$ and $F'(x) = f(x)$, find the intervals where $F(x)$ is:
        
        \begin{tabularx}{\textwidth}{p{3in} p{3in}}
            increasing & concave up \\
            & \\
            decreasing & concave down \\
            & \\
            & linear \\
            & \\
        \end{tabularx}

        \item Use the information above to sketch a graph of $F(x)$.
        \begin{center}
        \begin{tikzpicture}
        \begin{axis}[
            xmin=0, xmax=11,
            ymin=-9, ymax=9,
            xtick={0,...,11},  
            % xticklabels= {,,},
            ytick={-9,-6,-3,0,3,6,9},
            yticklabels= {-9,-6,-3,0,3,6,9},
            major tick length={0},
            line width=1pt,
            axis lines=center, height=2.74in, width = 5in, grid=major
            ]
        \end{axis}
        \end{tikzpicture}
        \end{center}
        \item How would your graph of $F(x)$ change if $F(0) =2$ instead?
    \end{enumerate}
    \newpage
    \item Find the derivative of the function
    \[g(x) = \int_1^x \cos(t^2) \ dt.\]
    \vfill 
    \item Find the derivative of the function
    \[h(x) = \int_2^{1/x} \sin^4{t} \ dt.\]
    \vfill
    \newpage
    \item Evaluate the definite integral
    \[\int_0^1 (u+2)(u-3) \ du.\]
    \vfill
    \item Evaluate the definite integral
    \[\int_1^2 \frac{v^5 + 3v^6}{v^4} \ dv.\]
    \vfill
\end{enumerate}
\end{document}
