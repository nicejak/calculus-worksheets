\documentclass[letterpaper,11pt]{article}
\usepackage{amsmath, mathtools, comment, graphicx, fancyhdr, color, setspace, comment, multicol, hyperref, tabu}
\usepackage[letterpaper,margin=1in,includehead=true]{geometry}
\pagestyle{fancy}
\usepackage{tikz,tikz-3dplot,pgfplots}
\usepackage{cancel}

\newcounter{mycounter}  
\newenvironment{noindlist}
 {\begin{list}{\arabic{mycounter}.~~}{\usecounter{mycounter} \labelsep=0em \labelwidth=0em \leftmargin=0em \itemindent=0em}}
 {\end{list}}

%To print solutions, use \solutionstrue
%To hide solutions, use \solutionsfalse
%\sol takes two arguments. #1 is the vertical length. #2 is the text.

\newif\ifsolutions
\solutionsfalse
% \solutionstrue
\ifsolutions
    \newcommand{\sol}[2]{\begin{minipage}[c][#1]{\linewidth}{\textcolor{blue}{\textbf{Solution:}}\quad \textcolor{blue}{#2}}\end{minipage}}
    \newcommand{\opsol}[1]{#1}
    \newcommand{\tsol}[2]{\textcolor{blue}{\textbf{Solution:}}\quad\vspace{-.25 in}#2}
            \newcommand{\blankans}[1]{\underbar{\enskip \textcolor{blue}{#1}\enskip}}
\else
    \newcommand{\sol}[2]{\begin{minipage}[c][#1]{\linewidth}{\vfill}\end{minipage}}
    \newcommand{\opsol}[1]{0}
    \newcommand{\tsol}[2]{#1}
         \newcommand{\blankans}[1]{\underbar{\enskip \textcolor{white}{#1}\enskip }}
\fi

\def\ds{\displaystyle}

\begin{document}
\lhead{Math 241: Calculus I}
\chead{}
\rhead{\bf Worksheet 18: Riemann Sums} 

\begin{enumerate}

\item A girl is running at a velocity of 12 feet per second for 10 seconds, as shown in the velocity graph below.

\begin{center}
\begin{tikzpicture}
\begin{axis}[
   	xmin=-1, xmax=12,
	ymin=-1, ymax=16,
	xtick={4, 7, 10},  
	ytick={6, 10, 12},
	major tick length={0},
	line width=1pt,
 	axis lines=center, height=2 in, grid=major, ylabel=$v(t)$, xlabel=$t$
	]
    \addplot [opacity=\opsol{.25}, domain=0:10,fill,blue] {12}  \closedcycle;
    \addplot [ultra thick, samples=9, domain=0:12.5] {12};
\end{axis}
\end{tikzpicture}
\end{center}

How far does she travel during this time?

\sol{1 in}{120 feet}

This distance can be depicted graphically as a rectangle. Shade such a rectangle and explain why it gives the distance.

\sol{1 in}{The area is found by multiplying an $x$-axis value by a $y$-axis value. In this case, we multiply 10 ft/sec by 12 sec, which gives $\ds \frac{10 \text{ ft}}{1 \cancel{\text{ sec}}} \cdot 12 \cancel{\text{ sec}}=120 \text{ ft}$}

\item Now the girl changes her velocity as she runs. Her velocity graph is approximately as shown:

\begin{center}
\begin{tikzpicture}
\begin{axis}[
   	xmin=-1, xmax=12,
	ymin=-1, ymax=16,
	xtick={4, 7, 10},  
	ytick={6, 10, 12},
	major tick length={0},
	line width=1pt,
 	axis lines=center, height=2 in, grid=major, ylabel=$v(t)$, xlabel=$t$
	]
    \addplot [mark=*, smooth, black, very thick] coordinates {(0,12) (4,12)};
    \addplot [mark=*, smooth, black, very thick] coordinates {(4,10) (7,10)};
    \addplot [mark=*, smooth, black, very thick] coordinates {(7,6) (10,6)};
    \addplot [black, only marks, very thick, mark=*, mark options={scale=1, fill=white}]
    coordinates{(4,12) (7,10)};
\end{axis}
\end{tikzpicture}
\end{center}

How far does she travel this time?

\sol{1 in}{$12 \text{ ft/sec} \cdot 4 \text{ sec}+10 \text{ ft/sec} \cdot 3 \text{ sec} +6 \text{ ft/sec} \cdot 3 \text{ ft/sec}$=$96$ ft.}

\newpage

\item This time she starts off slowly and speeds up. 

\begin{center}
\begin{tikzpicture}
\begin{axis}[
   	xmin=-1, xmax=12,
	ymin=-1, ymax=16,
	xtick={12},  
	ytick={12},
	major tick length={0},
	line width=1pt,
 	axis lines=center, height=2 in, grid=major, ylabel=$v(t)$, xlabel=$t$
	]
    \addplot [ultra thick, samples=9, domain=0:12.5] {1/12*x^2};
\end{axis}
\end{tikzpicture}
\end{center}

The velocity is given by $v(t)=\frac{t^2}{12}$ (time in seconds, velocity in ft/sec). 
We can no longer exactly find the distance travelled using areas of rectangles. But we can estimate it using areas of rectangles. 

\begin{center}
\begin{tikzpicture}
\begin{axis}[
   	xmin=-1, xmax=12,
	ymin=-1, ymax=16,
	xtick={3, 6, 9, 12},  
	ytick={12},
	major tick length={0},
	line width=1pt,
 	axis lines=center, height=2 in, grid=major, ylabel=$v(t)$, xlabel=$t$
	]
    \addplot [ultra thick, samples=9, domain=0:12.5] {1/12*x^2};
    \draw [draw, thick] (axis cs:0,0) rectangle (axis cs:3,.75);
    \draw [draw, thick] (axis cs:3,0) rectangle (axis cs:6,3);
    \draw [draw, thick] (axis cs:6,0) rectangle (axis cs:9,6.75);
    \draw [draw, thick] (axis cs:9,0) rectangle (axis cs:12,12);
\end{axis}
\end{tikzpicture}
\end{center}

Find her velocity at time $t=3, 6, 9, 12$ and use it to estimate her distance travelled in the first 12 seconds. 
In your answer to this problem {\bf use fractions, not decimals.}

\sol{2 in}{Using $v(t)=\frac{t^2}{12}$, $v(3)=3/4$, $v(6)=3$, $v(9)=27/4$, and $v(12)=12$. Though her velocity is always changing, we can use each of these velocities as an estimate of the velocities on the entire interval. We'll estimate her velocity as $3/4$ ft/sec for $t=0$ to $t=3$, $3$ ft/sec for $t=3$ to $t=6$, $27/4$ ft/sec for $t=6$ to $t=9$, and $12$ ft/sec for $t=9$ to $t=12$. This gives us a distance of 
\[\frac{3}{4} \text{ ft/sec} \cdot 3 \text{ sec}+3 \text{ ft/sec} \cdot 3 \text{ sec}+\frac{27}{4} \text{ ft/sec} \cdot 3 \text{ sec}+12 \text{ ft/sec} \cdot 3 \text{ sec}=\frac{135}{2}=67.5 \text{ ft.}\]}

\newpage

\item Now, for the same velocity function $v(t)=\frac{t^2}{12}$, get a better estimate of how far she travelled using $n=6$ rectangles. Draw a graph showing the areas, and use their areas to estimate her distance travelled in the first 12 seconds.
Again {\bf use fractions, not decimals.}

\begin{center}
\begin{tikzpicture}
\begin{axis}[
   	xmin=-1, xmax=12,
	ymin=-1, ymax=16,
	xtick={2,4,...,12},  
	ytick={12},
	major tick length={0},
	line width=1pt,
 	axis lines=center, height=3 in, grid=major, ylabel=$v(t)$, xlabel=$t$
	]
    \addplot [ultra thick, samples=9, domain=0:12.5] {1/12*x^2};
    \addplot [opacity=\opsol{.25}, domain=0:2,fill,blue] {4/12}  \closedcycle;
    \addplot [opacity=\opsol{.25}, domain=2:4,fill,blue] {16/12}  \closedcycle;
    \addplot [opacity=\opsol{.25}, domain=4:6,fill,blue] {36/12}  \closedcycle;
    \addplot [opacity=\opsol{.25}, domain=6:8,fill,blue] {64/12}  \closedcycle;
    \addplot [opacity=\opsol{.25}, domain=8:10,fill,blue] {100/12}  \closedcycle;
    \addplot [opacity=\opsol{.25}, domain=10:12,fill,blue] {144/12}  \closedcycle;
\end{axis}
\end{tikzpicture}
\end{center}

\sol{1 in}{
\begin{align*}
\text{Area} &\approx \frac{12}{6} \left( \frac{(1 \cdot 2)^2}{12}+\frac{(2 \cdot 2)^2}{12}+\frac{(3 \cdot 2)^2}{12}+\frac{(4 \cdot 2)^2}{12}+\frac{(5 \cdot 2)^2}{12}+\frac{(6 \cdot 2)^2}{12} \right) \\
&\approx \frac{182}{3} \;=\; 60\frac{2}{3} \text{ ft.}
\end{align*}
}

\item  Now we will estimate the area when there are $n=37$ rectangles. 
{\bf Use fractions, not decimals.}

\begin{enumerate}
\item Width of each rectangle: 

\sol{.4 in}{$\dfrac{12}{37}$}

\item List of right-hand endpoint of each rectangle:\\ 

\blankans{$12/37$}, \blankans{$2\cdot12/37$}, \blankans{$3\cdot12/37$},...,\blankans{$36\cdot12/37$}, \blankans{$37\cdot12/37$} 


\item List of heights of each rectangle: \\\\

\blankans{$\frac{(1\cdot12/37)^2}{12}$},
\blankans{$\frac{(2\cdot12/37)^2}{12}$},
\blankans{$\frac{(3\cdot12/37)^2}{12}$},...,
\blankans{$\frac{(36\cdot12/37)^2}{12}$},
\blankans{$\frac{(37\cdot12/37)^2}{12}$}

\item List of areas of rectangles: \\\\

\blankans{$\frac{12}{37}\cdot\frac{(1\cdot12/37)^2}{12}$},
\blankans{$\frac{12}{37}\cdot\frac{(2\cdot12/37)^2}{12}$},
\blankans{$\frac{12}{37}\cdot\frac{(3\cdot12/37)^2}{12}$},...,
\blankans{$\frac{12}{37}\cdot\frac{(36\cdot12/37)^2}{12}$},
\blankans{$\frac{12}{37}\cdot\frac{(37\cdot12/37)^2}{12}$}

\item Sum of all areas:\\\\
 
\blankans{$\frac{12}{37}\cdot\frac{(1\cdot12/37)^2}{12}$}+
\blankans{$\frac{12}{37}\cdot\frac{(2\cdot12/37)^2}{12}$}+
\blankans{$\frac{12}{37}\cdot\frac{(3\cdot12/37)^2}{12}$}+...+
\blankans{$\frac{12}{37}\cdot\frac{(36\cdot12/37)^2}{12}$}+
\blankans{$\frac{12}{37}\cdot\frac{(37\cdot12/37)^2}{12}$}

\end{enumerate}


\item Now we will figure out the estimate when there are an arbitrary number of rectangles, or $n$ rectangles.

\begin{enumerate}
\item Width of each rectangle: 

\sol{.4 in}{$\dfrac{12}{n}$}

\item List of right-hand endpoint of each rectangle:\\ 

\blankans{$1 \cdot 12/n$}, \blankans{$2\cdot12/n$}, \blankans{$3\cdot12/n$},...,\blankans{$(n-1)\cdot12/n$}, \blankans{$n\cdot12/n$} 


\item List of heights of each rectangle: \\\\

\blankans{$\frac{(1\cdot12/n)^2}{12}$},
\blankans{$\frac{(2\cdot12/n)^2}{12}$},
\blankans{$\frac{(3\cdot12/n)^2}{12}$},...,
\blankans{$\frac{((n-1)\cdot12/n)^2}{12}$},
\blankans{$\frac{(n\cdot12/n)^2}{12}$}


\item List of areas of rectangles: \\\\

\blankans{$\frac{12}{n}\cdot\frac{(1\cdot12/n)^2}{12}$},
\blankans{$\frac{12}{n}\cdot\frac{(2\cdot12/n)^2}{12}$},
\blankans{$\frac{12}{n}\cdot\frac{(3\cdot12/n)^2}{12}$},...,
\blankans{$\frac{12}{n}\cdot\frac{((n-1)\cdot12/n)^2}{12}$},
\blankans{$\frac{12}{n}\cdot\frac{(n\cdot12/n)^2}{12}$}

\item Sum of all areas:\\\\
 
\blankans{$\frac{12}{n}\cdot\frac{(1\cdot12/n)^2}{12}$}+
\blankans{$\frac{12}{n}\cdot\frac{(2\cdot12/n)^2}{12}$}+
\blankans{$\frac{12}{n}\cdot\frac{(3\cdot12/n)^2}{12}$}+...+
\blankans{$\frac{12}{n}\cdot\frac{((n-1)\cdot12/n)^2}{12}$}+
\blankans{$\frac{12}{n}\cdot\frac{(n\cdot12/n)^2}{12}$}

\end{enumerate}


\newpage

\item Manipulate the sum algebraically until it is of the form: $$\textit{stuff} \cdot (1+4+9+...+n^2).$$

\sol{2.2 in}{
\begin{align*}
\text{Sum} &= \frac{12}{n} \cdot \frac{(1 \cdot 12/n)^2}{12}+\frac{12}{n} \cdot \frac{(2 \cdot 12/n)^2}{12}+\frac{12}{n} \cdot \frac{(3 \cdot 12/n)^2}{12}+ ... + \frac{12}{n} \cdot \frac{(n \cdot 12/n)^2}{12} \\
&= \frac{12}{n} \cdot \frac{(12/n)^2}{12} \cdot (1^2+2^2+3^2+...+n^2) \\
&= \frac{12^2}{n^3} \cdot (1^2+2^2+3^2+...+n^2)
\end{align*}
}
\item Simplify further by substituting $\ds 1+4+9+...+n^2=\frac{n(n+1)(2n+1)}{6}$ into your answer above. Check that it gives the same answer for $n=6$ that you got in problem 4.

\sol{2.3 in}{\[\frac{12^2}{n^3} \cdot (1^2+2^2+3^2+...+n^2)= \frac{12^2}{n^3} \cdot \frac{n(n+1)(2n+1)}{6}=\frac{24}{n^2} \cdot (n+1)(2n+1)\]
For $n=6$ this gives $\ds \frac{24}{36} \cdot (6+1)(2 \cdot 6+1)\approx 60.667$ which is the same answer we got above.}

\item As $n$ approaches infinity we find her exact distance travelled (the exact area under the curve). Take the limit as $n$ goes to infinity for your answer to the previous problem.

\sol{2in}{\[ \lim_{n\rightarrow \infty} \frac{24(n+1)(2n+1)}{n^2}=\lim_{n\rightarrow \infty} \frac{48 n^2+72 n+24}{n^2}=48\]}

Notice that you just found the area inside a region with a curved edge!
\end{enumerate}
\end{document}
