\documentclass[letterpaper,11pt]{article}
\usepackage{amsmath}
\usepackage[letterpaper,margin=1in,includehead=true]{geometry}
\usepackage{comment}
\usepackage{graphicx}
\usepackage{fancyhdr}
\pagestyle{fancy}
\usepackage{color}
\usepackage{setspace}
\usepackage{tikz,tikz-3dplot,pgfplots}
\usepackage{comment}
\usepackage{multicol}
\usepackage{mdframed}
\usepackage{enumitem}
\usepackage{tabularx}

\newmdtheoremenv{definition}{Definition}
\newmdtheoremenv{theorem}{Theorem}
\setlength{\headheight}{15pt}

\newcounter{mycounter}  
\newenvironment{noindlist}
 {\begin{list}{\arabic{mycounter}.~~}{\usecounter{mycounter} \labelsep=0em \labelwidth=0em \leftmargin=0em \itemindent=0em}}
 {\end{list}}

%To print solutions, use \solutionstrue
%To repress solutions, use \solutionsfalse
%\sol takes two arguments. #1 is the vertical length. #2 is the text.

\newif\ifsolutions
\solutionsfalse
\ifsolutions
    \newcommand{\sol}[2]{\begin{minipage}[c][#1]{\linewidth}{\textcolor{blue}{\textbf{Solution:}}\quad \textcolor{blue}{#2}}\end{minipage}}
    \newcommand{\opsol}[1]{#1}
    \newcommand{\tblsol}[1]{\textcolor{blue}{#1}}
\else
    \newcommand{\sol}[2]{\begin{minipage}[c][#1]{\linewidth}{\vfill}\end{minipage}}
    \newcommand{\opsol}[1]{0}
    \newcommand{\tblsol}[1]{\textcolor{white}{#1}}
\fi

\newcommand{\unenumerate}[1]{\setcounter{saveenum}{\value{enumi}}\end{enumerate}
	\noindent #1 
	\begin{enumerate} \setcounter{enumi}{\value{saveenum}}}

\newcounter{saveenum}

\def\ds{\displaystyle}

\begin{document}
\lhead{Math 241: Calculus I}
\rhead{\bf Worksheet 21: Indefinite Integrals} 
Somewhere along the worksheet, you might run into some trig identities, such as (but not limited to):
\begin{align*}
    \cos^2{x} & = \frac{1+ \cos{2x}}{2}\\
    \sin(2x) & = 2\sin{x}\cos{x}
\end{align*}
\begin{enumerate}
    \item Verify by differentiation that the following integrals are correct. 
    \begin{enumerate}
        \item $\displaystyle \int \frac{1}{x^2 \sqrt{1+x^2}} \ dx = -\frac{\sqrt{1+x^2}}{x} + C$
        \vfill
        \item $\displaystyle \int \cos^2{x} \ dx = \frac{1}{2} x + \frac{1}{4} \sin{2x} + C$
        \vfill
    \end{enumerate}
    \newpage
    \item Find the general indefinite integrals
    \begin{enumerate}
        \item $\displaystyle \int x^{13} + 7x^{25} \ dx$
        \vfill
        \item $\displaystyle \int \sqrt{t}(t^2 + 3t + 2) \ dt$
        \vfill
        \item $\displaystyle \int \frac{\sin(2x)}{\sin{x}} \ dx$
        \vfill
    \end{enumerate}
    \newpage

    \item Evaluate the definite integrals
    \begin{enumerate}
        \item $\displaystyle \int_1^4 \sqrt{\frac{5}{x}} \ dx$
        \vfill
        \item $\displaystyle \int_0^{\pi/4} \frac{1+\cos^2{x}}{\cos^2{x}} \ dx$
        \vfill
        \item $\displaystyle \int_{-1}^2 x - 2|x| \ dx$
        \vfill
    \end{enumerate}
    \newpage
    \item The acceleration function (in m/s$^2$) and the inital velocity is given for a particle moving along a line. Find the velocity $v(t)$ function and the distance traveled during the time interval given.
    \[a(t) = t+4, \quad v(0) = 5, \quad 0 \leq t \leq 10.\]
    \vfill
\end{enumerate}
\end{document}
