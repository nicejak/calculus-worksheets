\documentclass[letterpaper,11pt]{article}
\usepackage{amsmath}
\usepackage[letterpaper,margin=1in,includehead=true]{geometry}
\usepackage{comment}
\usepackage{graphicx}
\usepackage{fancyhdr}
\pagestyle{fancy}
\usepackage{color}
\usepackage{setspace}
\usepackage{tikz,tikz-3dplot,pgfplots}
\usepackage{comment}
\usepackage{multicol}
\usepackage{mdframed}
\usepackage{enumitem}

\newmdtheoremenv{definition}{Definition}
\newmdtheoremenv{theorem}{Theorem}
\setlength{\headheight}{15pt}

\newcounter{mycounter}  
\newenvironment{noindlist}
 {\begin{list}{\arabic{mycounter}.~~}{\usecounter{mycounter} \labelsep=0em \labelwidth=0em \leftmargin=0em \itemindent=0em}}
 {\end{list}}

%To print solutions, use \solutionstrue
%To repress solutions, use \solutionsfalse
%\sol takes two arguments. #1 is the vertical length. #2 is the text.

\newif\ifsolutions
\solutionsfalse
\ifsolutions
    \newcommand{\sol}[2]{\begin{minipage}[c][#1]{\linewidth}{\textcolor{blue}{\textbf{Solution:}}\quad \textcolor{blue}{#2}}\end{minipage}}
    \newcommand{\opsol}[1]{#1}
    \newcommand{\tblsol}[1]{\textcolor{blue}{#1}}
\else
    \newcommand{\sol}[2]{\begin{minipage}[c][#1]{\linewidth}{\vfill}\end{minipage}}
    \newcommand{\opsol}[1]{0}
    \newcommand{\tblsol}[1]{\textcolor{white}{#1}}
\fi

\newcommand{\unenumerate}[1]{\setcounter{saveenum}{\value{enumi}}\end{enumerate}
	\noindent #1 
	\begin{enumerate} \setcounter{enumi}{\value{saveenum}}}

\newcounter{saveenum}

\def\ds{\displaystyle}

\begin{document}
\lhead{Math 241: Calculus I}
\rhead{\bf Worksheet 19: Definite Integral} 
\begin{enumerate}
    \item Use the Midpoint Rule with $n=4$ to approximate the integral
    \[\int_0^8 \sin{\sqrt{x}} \ dx\]
    with the answer rounded to four decimal places.
    \vfill

    \item The definition of an integral on a range $[a,b]$ is
    \[\int_a^b f(x) \ dx = \lim_{n\to \infty} \sum_{i=1}^n f(x_i) \Delta x\]
    where $\Delta x = \frac{b-a}{n}$ and $x_i = a + i \Delta x$.
    
    Evaluate $\int_{-2}^0 (x^2 + x) \ dx$ using the definition of the integral.
    \vfill
    \newpage
    \item The graph of $f$ is shown. Evaluate each integral by interpreting it in terms of area.
    \begin{center}
    \begin{tikzpicture}
    \begin{axis}[
        xmin=-1, xmax=10,
        ymin=-4, ymax=4,
        xtick={-1,...,10},  
        % xticklabels= {,,},
        ytick={-4,...,4},
        % yticklabels= {,,},
        major tick length={0},
        line width=1pt,
        axis lines=center, height=3.64in, width = 5in, grid=major
        ]
        \addplot [samples=100, domain=0:2] {x + 1};
        \addplot [samples=100, domain=2:3] {3};
        \addplot [samples=100, domain=3:7] {-1.5*x + 7.5};
        \addplot [samples=100, domain=7:9] {0.5*x - 6.5};
    \end{axis}
    \end{tikzpicture}
    \end{center}
    \begin{enumerate}
        \item $\displaystyle \int_0^2 f(x) \ dx$
        \vfill
        \item $\displaystyle \int_0^5 f(x) \ dx$
        \vfill
        \item $\displaystyle \int_5^7 f(x) \ dx$
        \vfill
        \item $\displaystyle \int_0^9 f(x) \ dx$
        \vfill
    \end{enumerate}
    \newpage
    \item Evaluate the integrals by interpreting it in terms of area. You will benefit from sketching the graph.
    \begin{enumerate}
        \item $\displaystyle \int_{-1}^2 (1-x) \ dx$
        \vfill
        \item $\displaystyle \int_{-3}^0 (1 + \sqrt{9-x^2}) \ dx$
        \vfill
    \end{enumerate}
    \newpage
    \item Write the equation into a single integral
    \[\int_{-2}^2 f(x) \ dx + \int_2^5 f(x) \ dx - \int_{-2}^{-1} f(x) \ dx.\]
    \vfill
    \item If $\int_0^9 f(x) \ dx = 37$ and $\int_0^9 g(x) \ dx = 16$, find 
    \[\int_0^9 [2f(x) + 3g(x)] \ dx.\]
    \vfill
\end{enumerate}
\end{document}
