\documentclass[letterpaper,11pt]{article}
\usepackage{amsmath}
\usepackage[letterpaper,margin=1in,includehead=true]{geometry}
\usepackage{comment}
\usepackage{graphicx}
\usepackage{fancyhdr}
\pagestyle{fancy}
\usepackage{color}
\usepackage{multicol}

%%%%%%%%%%%%%
\usepackage{pgfplots}
\usetikzlibrary{calc}
\usepackage{wrapfig}
\usepackage{enumitem}

\newcounter{mycounter}  
\newenvironment{noindlist}
 {\begin{list}{\arabic{mycounter}.~~}{\usecounter{mycounter} \labelsep=0em \labelwidth=0em \leftmargin=0em \itemindent=0em}}
 {\end{list}}

%To print solutions, use \solutionstrue
%To repress solutions, use \solutionsfalse
%\sol takes two arguments. #1 is the vertical length. #2 is the text.

\newif\ifsolutions
\solutionsfalse
\ifsolutions
    \newcommand{\sol}[2]{\begin{minipage}[c][#1]{\linewidth}{\textcolor{blue}{\textbf{Solution:}}\quad \textcolor{blue}{#2}}\end{minipage}}
    \newcommand{\opsol}{1}
\else
    \newcommand{\sol}[2]{\begin{minipage}[c][#1]{\linewidth}{\vfill}\end{minipage}}
    \newcommand{\opsol}{0}
\fi

\newcommand{\unenumerate}[1]{\setcounter{saveenum}{\value{enumi}}\end{enumerate}
	\noindent #1 
	\begin{enumerate} \setcounter{enumi}{\value{saveenum}}}

\newcounter{saveenum}

\def\changemargin#1#2{\list{}{\rightmargin#2\leftmargin#1}\item[]}
\let\endchangemargin=\endlist 

\begin{document}

\lhead{\bf Math 241: Calculus I}
\rhead{\bf Quiz 5: Limits and Continuity}
\begin{enumerate}
\item Recall the Squeeze Theorem that states 

\textbf{Theorem:} (Squeeze Theorem) \\
If $f(x) \leq g(x) \leq h(x)$ when $x$ is near $a$ and 
\[\lim_{x\to a} f(x) = \lim_{x\to a}h(x) = L\]
then
\[\lim_{x\to a}g(x) = L.\]

Using the above theorem, prove that
\[\lim_{x\to 0} x^4 \cos\left(\frac{2}{x}\right) = 0.\]
\vfill
\newpage
\item Suppose that $f(x) = x^2$. Solve and simplify the following limit.
\[\lim_{h \to 0} \frac{f(x+h)-f(x)}{h}.\]
\vfill

\end{enumerate}

\end{document}