\documentclass{article}
\usepackage{graphicx} % Required for inserting images
\usepackage{termcal}
\usepackage[margin = 1in]{geometry}
\usepackage{hyperref}
\hypersetup{
    colorlinks=true,
    linkcolor=blue,
    filecolor=magenta,      
    urlcolor=blue,
    }

\title{Calculus 1 Syllabus}
\author{Summer 2023}
\date{\ }

\setlength\parindent{0pt}

\begin{document}

\maketitle

\textbf{Course: } MATH241-803

\textbf{Intructor: }Alan Tong \quad tongalan@hawaii.edu

\textbf{Lectures: }Keller Hall 313 \quad MTWRF 12:00pm-1:50pm

\textbf{Office Hours: }T/R 2:00-3:30pm, M-F 11:00-12:00pm

\textbf{Course Overview: }

Welcome to calculus one, likely to be a large step in mathematics, regardless of how long its been since your last math course. Your ability to succeed in calculus will depend heavily on your ability to work with functions and algebra from previous courses, pre-calculus and trigonometry. In this course, we will cover basic calculus concepts, differentiation with applications, and integration.

\textbf{Learning Goals: }

A student after successful completion of this course will have the tools needed to complete calculus two. That is, a student will be able to calculate limits properly, compute derivatives of elementary functions and graph that function with its key components, and integrate a function for application in future courses, along with picking up some big theorems.

\textbf{Prerequisites: }

A grade of C or better in Math 140 or Math215 or precalculus assessment \url{https://math.hawaii.edu/wordpress/placement-exam/}.

\textbf{Course Materials: }

I will be utilizing Calculus by James Stewart, 8th edition. I also recommend students bring a method to write notes and a pen/pencil to complete in-class worksheets and quizzes. 

\textbf{Written Homework: }

Worksheets assigned in class will become homework if the worksheet is not finished during class time or if the student has an excused absence.

\textbf{Exams: }

There will be one midterm in the middle of the summer session (6/30) and one final exam at the end of the summer session (7/21), on a 100 point scale. If there is an attendance conflict, you must resolve it \textbf{before} the exam occurs; otherwise, making up an exam is not allowed. Neither exams are cumulative, but you will find that the topics build on each other.

\textbf{Coursework and Grading: }

My teaching style will be \href{https://teaching.berkeley.edu/resources/course-design-guide/active-learning}{active learning}, which demands students attend class regularly for exercises. Submission of these exercises as worksheets will be worth majority of the student's grade on a ten point scale. At the beginning of every non-examination class, there will be a small review quiz that will be used as attendance. This review quiz will not be graded on correctness.

The grading scale is as follows
\begin{center}
    \begin{tabular}{|l | l|}
        \hline
        Homework and Worksheets & 50 \% \\
        \hline 
        Midterm & 15 \% \\
        \hline
        Final Exam & 25 \% \\
        \hline
        Attendance & 10 \% \\
        \hline
    \end{tabular}
\end{center}

\textbf{Makeup Work: }

Work missed in class is only allowed to be made up if a student informs me of their absence with proof of absence in 12-hour advance of the class, and I excuse it. Excuses will not be allowed without reason. The student will be responsible for completing their work and it must be submitted by the end of the day after the missed day, through email or in person.

Work returned may be revised and resubmitted for up to full credits.

\textbf{Term Calendar: }
\begin{calendar}{6/12/23}{6}
\setlength{\calboxdepth}{.3in}
\setlength{\calwidth}{\linewidth}
% Description of the Week.
\calday[Monday]{\classday} % Monday
\calday[Tuesday]{\classday} % Tuesday
\calday[Wednesday]{\classday} % Wednesday
\calday[Thursday]{\classday} % Thursday
\calday[Friday]{\classday} % Friday
\skipday\skipday % weekend (no class)
% Holidays
\caltext{6/12/23}{No Class\\Kamehameha Day}
\caltext{7/4/23}{No Class\\Independence Day}
% Exams
\caltext{7/3/23}{Midterm Exam (up to chapter 2)}
\caltext{7/21/23}{Final Exam}
% Text on consecutive days
\caltexton{2}{Introduction and review}
\caltextnext{Tangent and velocity \S1.4}
\caltextnext{Limit of a function \& Limit laws \S1.5,1.6}
\caltextnext{Precise definition of a limit and continuity \S1.7,1.8}

\caltextnext{Derivatives as rate of change and as a function \S2.1,2.2}
\caltextnext{Derivative formulas and derivatives of trigonometric functions \S2.3,2.4}
\caltextnext{Chain rule \S2.5}
\caltextnext{Implicit differentiation \S2.6}
% \caltextnext{Rates of change \S2.7}
\caltextnext{Related rates \S2.8}

\caltextnext{Linear approximation and differentials \S2.9}
\caltextnext{Max and min values \S3.1}
\caltextnext{The MVT \S3.2}
\caltextnext{Graphing and limits at infinity \S3.3,3.4}
\caltextnext{Summary of graphing \S3.5}
\caltextnext{}
\caltextnext{}
\caltextnext{Optimization \S3.7}
\caltextnext{Newton's method \S3.8}

\caltextnext{Antiderivatives \S3.9}
\caltextnext{Areas and distances \S4.1}
\caltextnext{The definite integral \S4.2}
\caltextnext{The FTC \S4.3}
\caltextnext{Indefinite integrals \S4.4}

\caltextnext{$u$-substitution \S4.5}
\caltextnext{Areas between curves \S5.1}
\caltextnext{Volumes \S5.2}
\caltextnext{Volumes by cylindrical shells \S5.3}
\caltextnext{Review}

\end{calendar}

{\bf Sources of help:}

All students are encouraged to come to office hours to discuss homework questions or material from class. Also consider the \href{https://natsci.manoa.hawaii.edu/learningemporium/}{Learning Emporium}.

{\bf Academic integrity:}

All students are expected to abide by the university's Conduct Code. Academic sanctions for dishonesty may include receiving an F in the assignment or receiving an F in the class. There may be additional administrative sanctions.
\newline {\small\url{https://www.hawaii.edu/policy/?action=viewPolicy&policySection=ep&policyChapter=7&policyNumber=208}}

{\bf KOKUA:} 

I am happy to work with you and the KOKUA Program (Office for Students with Disabilities), if you need course accommodations due to a disability.
KOKUA can be reached at (808) 956-7511 or (808) 956-7612 (voice/text) in room 013 of the Queen Lili`uokalani Center for Student Services.
All course modifications must be arranged through KOKUA. You are encouraged to start this process as early as possible.

\textbf{Concerns:}

If at any time during the semester you have any questions or concerns about the class, please contact me during regularly scheduled office hours or via email to make an appointment. You may also contact the mathematics department office located on the fourth floor of Keller Hall.

\end{document}
