\documentclass[letterpaper,11pt]{article}
\usepackage{amsmath}
\usepackage[letterpaper,margin=1in,includehead=true]{geometry}
\usepackage{comment}
\usepackage{graphicx}
\usepackage{fancyhdr}
\pagestyle{fancy}
\usepackage{color}
\usepackage{setspace}
\usepackage{tikz,tikz-3dplot,pgfplots}
\usepackage{comment}
\usepackage{multicol}

\setlength{\headheight}{15pt}

\newcounter{mycounter}  
\newenvironment{noindlist}
 {\begin{list}{\arabic{mycounter}.~~}{\usecounter{mycounter} \labelsep=0em \labelwidth=0em \leftmargin=0em \itemindent=0em}}
 {\end{list}}

%To print solutions, use \solutionstrue
%To repress solutions, use \solutionsfalse
%\sol takes two arguments. #1 is the vertical length. #2 is the text.

\newif\ifsolutions
\solutionsfalse
\ifsolutions
    \newcommand{\sol}[2]{\begin{minipage}[c][#1]{\linewidth}{\textcolor{blue}{\textbf{Solution:}}\quad \textcolor{blue}{#2}}\end{minipage}}
    \newcommand{\opsol}[1]{#1}
    \newcommand{\tblsol}[1]{\textcolor{blue}{#1}}
\else
    \newcommand{\sol}[2]{\begin{minipage}[c][#1]{\linewidth}{\vfill}\end{minipage}}
    \newcommand{\opsol}[1]{0}
    \newcommand{\tblsol}[1]{\textcolor{white}{#1}}
\fi

\newcommand{\unenumerate}[1]{\setcounter{saveenum}{\value{enumi}}\end{enumerate}
	\noindent #1 
	\begin{enumerate} \setcounter{enumi}{\value{saveenum}}}

\newcounter{saveenum}

\def\ds{\displaystyle}

\begin{document}
\lhead{Math 241: Calculus I}
\rhead{\bf Worksheet 16: Newton's Method} 
\textbf{Please do not use a factoring or graphing calculator on this first page.}

\begin{enumerate}

\item Find the $x$-intercepts of $x^2 + 2x -3$ using \textbf{any} method you like (except a calculator). You may factor, use the quadratic equation, or even guess.
\vfill
\item Let's change the function just barely by adding 1 to it to obtain $x^2 + 2x -2$. What are the $x$-intercepts of this equation now?
\vfill
\item Changing the function just barely again by increasing the degree to 3, we obtain $x^3 + 2x - 2$. Do not solve for the $x$-intercepts yet. Rather, explain why this new equation is difficult to find exact roots of.
\vfill

\newpage
\item Newton's method is a method for approximating the roots of an equation using geometry and derivatives. Use the graph below of $y=x^3 + 2x -2$ to help you estimate the $x$-intercept using Newton's method in the next parts

\begin{center}
\begin{tikzpicture}
\begin{axis}[
    xmin=0, xmax=2,
    ymin=-5, ymax=5,
    xtick={0,...,2},  
    % xticklabels= {,,},
    ytick={-5,...,5},
    yticklabels= {,,},
    major tick length={0},
    line width=1pt,
    axis lines=center, height=6in, width = 6in, grid=major
    ]
    \addplot [samples=100, domain=-2:2] {x^3+2*x-2};
\end{axis}
\end{tikzpicture}
\end{center}
\newpage
\begin{enumerate}
    \item Newton's method requires an initial first guess for a root. Pick an integer $x$-value on the graph to declare as your first guess $x_1$.
    \vspace{1in}
    \item Draw the linearization of the function at your first guess $x_1$. Does your $x$-intercept of the linearization on the graph approximate the root of the original equation $x^3 + 2x -2$?
    \vfill
    \item Write out the linearization equation and solve for the $x$-intercept, and label it $x_2$. You may use a calculator to help you calculate.
    \vfill
    \item Repeat the above two steps for a further approximation. So graph the second linearization and find $x_3$.
    \vfill
    \vfill
\end{enumerate}
\newpage
\item The generalization of this process arises a formula 
\[x_{n+1} = x_n - \frac{f(x_n)}{f'(x_n)}\] 
where $x_n$ is a previous estimation and $x_{n+1}$ is the successive estimation (hopefully better).

Use this equation to find the root of $x^7 + 4 = 0$ with the first guess $x_1 = -1$. Only estimate with 2 approximations (that is, find $x_3$).
\vfill
\newpage
\item Approximate $\sqrt[4]{75}$ using Newton's method.
\vfill

\end{enumerate}

\end{document}
