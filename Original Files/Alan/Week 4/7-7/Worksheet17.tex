\documentclass[letterpaper,11pt]{article}
\usepackage{amsmath}
\usepackage[letterpaper,margin=1in,includehead=true]{geometry}
\usepackage{comment}
\usepackage{graphicx}
\usepackage{fancyhdr}
\pagestyle{fancy}
\usepackage{color}
\usepackage{setspace}
\usepackage{tikz,tikz-3dplot,pgfplots}
\usepackage{comment}
\usepackage{multicol}
\usepackage{mdframed}
\usepackage{enumitem}

\newmdtheoremenv{definition}{Definition}
\newmdtheoremenv{theorem}{Theorem}
\setlength{\headheight}{15pt}

\newcounter{mycounter}  
\newenvironment{noindlist}
 {\begin{list}{\arabic{mycounter}.~~}{\usecounter{mycounter} \labelsep=0em \labelwidth=0em \leftmargin=0em \itemindent=0em}}
 {\end{list}}

%To print solutions, use \solutionstrue
%To repress solutions, use \solutionsfalse
%\sol takes two arguments. #1 is the vertical length. #2 is the text.

\newif\ifsolutions
\solutionstrue
\ifsolutions
    \newcommand{\sol}[2]{\begin{minipage}[c][#1]{\linewidth}{\textcolor{blue}{\textbf{Solution:}}\quad \textcolor{blue}{#2}}\end{minipage}}
    \newcommand{\opsol}[1]{#1}
    \newcommand{\tblsol}[1]{\textcolor{blue}{#1}}
\else
    \newcommand{\sol}[2]{\begin{minipage}[c][#1]{\linewidth}{\vfill}\end{minipage}}
    \newcommand{\opsol}[1]{0}
    \newcommand{\tblsol}[1]{\textcolor{white}{#1}}
\fi

\newcommand{\unenumerate}[1]{\setcounter{saveenum}{\value{enumi}}\end{enumerate}
	\noindent #1 
	\begin{enumerate} \setcounter{enumi}{\value{saveenum}}}

\newcounter{saveenum}

\def\ds{\displaystyle}

\begin{document}
\lhead{Math 241: Calculus I}
\rhead{\bf Worksheet 17: Antiderivatives} 
The upcoming chapter will introduce integrals, which require knowledge of antiderivatives. As the name implies, antiderivatives are derivatives going backwards. Read the definition below carefully and practice the problems below.
\begin{definition}
    A function $F$ is an \textbf{antiderivative} of another function $f$ if $F'(x) = f(x)$.
\end{definition}
\begin{enumerate}
\item Can you find an antiderivative of $g(x) = x^3$? Label your answer $G(x)$.
\vfill
\item Check that $G'(x) = g(x)$ by taking its derivative below.
\vfill
\item Can you find another antiderivative of $g(x) = x^3$ different from $G(x)$? Label your new answer $G_1(x)$.\footnote{Hint: what can be added on to a function that disappears when a derivative is taken?}
\vfill
\newpage
\item Find another distinct antiderivative of $g(x) = x^3$ and label it $G_2(x)$.
\vfill
\end{enumerate}
Problems 3 and 4 induce a theorem on antiderivatives.
\begin{theorem}
    If $F$ is an antiderivative of $f$, then the most general antiderivative of $f$ is
    \[F(x) + C\]
    where $C$ is an arbitrary constant.
\end{theorem}
This theorem implies that any function has an infinite collection of antiderivatives with any constant $C$ in $(-\infty,\infty)$. Any problem asking for an antiderivative will be expecting the $+C$ from here forth!
\vspace{.5in}
\begin{enumerate}[resume]
    \item What is the (general) antiderivative of $f(x) = x^n$ for some power $n$?
    \vfill
    \item The power rule for a derivative of $x^n$ is "bring the power to the front and lower the power by one". What is the power rule for an antiderivative of $x^n$? Write an equivalent sentence.
    \vfill
    \newpage
    \item What are the antiderivatives of $f'(x) = 2\cos{x}$?
    \vfill
    \item What are the antiderivatives of $f'(x) = x^2 + 3\sin{x}$?
    \vfill
    \item What is the specific antiderivative of $f'(x) = 1 + 3\sqrt{x}$ if $f(4) = 25$? \textit{You should not have $+C$ in your answer.}
    \vfill
    \newpage
    \item What is the specific antiderivative of $f'(x) = \frac{x+1}{\sqrt{x}}$ if $f(1) = 5$?
    \vfill
    \item A car is traveling at 50 mi/h when the brakes are fully applied, producing a constant deceleration of 22 ft/s$^2$. What is the distance traveled before the car comes to a stop?
    \vfill
\end{enumerate}
\end{document}
