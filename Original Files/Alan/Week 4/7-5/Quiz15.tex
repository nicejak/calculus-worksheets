\documentclass[letterpaper,11pt]{article}
\usepackage{amsmath}
\usepackage[letterpaper,margin=1in,includehead=true]{geometry}
\usepackage{comment}
\usepackage{graphicx}
\usepackage{fancyhdr}
\pagestyle{fancy}
\usepackage{color}
\usepackage{multicol}

%%%%%%%%%%%%%
\usepackage{pgfplots}
\usetikzlibrary{calc}
\usepackage{wrapfig}
\usepackage{enumitem}

\newcounter{mycounter}  
\newenvironment{noindlist}
 {\begin{list}{\arabic{mycounter}.~~}{\usecounter{mycounter} \labelsep=0em \labelwidth=0em \leftmargin=0em \itemindent=0em}}
 {\end{list}}

%To print solutions, use \solutionstrue
%To repress solutions, use \solutionsfalse
%\sol takes two arguments. #1 is the vertical length. #2 is the text.

\newif\ifsolutions
\solutionstrue
\ifsolutions
    \newcommand{\sol}[2]{\begin{minipage}[c][#1]{\linewidth}{\textcolor{blue}{\textbf{Solution:}}\quad \textcolor{blue}{#2}}\end{minipage}}
    \newcommand{\opsol}{1}
\else
    \newcommand{\sol}[2]{\begin{minipage}[c][#1]{\linewidth}{\vfill}\end{minipage}}
    \newcommand{\opsol}{0}
\fi

\newcommand{\unenumerate}[1]{\setcounter{saveenum}{\value{enumi}}\end{enumerate}
	\noindent #1 
	\begin{enumerate} \setcounter{enumi}{\value{saveenum}}}

\newcounter{saveenum}

\def\changemargin#1#2{\list{}{\rightmargin#2\leftmargin#1}\item[]}
\let\endchangemargin=\endlist 

\begin{document}

\lhead{\bf Math 241: Calculus I}
\rhead{\bf Quiz 15: Mean Value Theorem}
\begin{enumerate}
\item Graph the function
\[f(x) = \frac{x}{x^3-1} = \frac{x}{(x-1)(x^2+x+1)}\]
by including roots, asymptotes, local minimums and maximums, and inflection points. Use exact values when labeling these points. You do not need to find the $y$ value corresponding to each points $x$ value. There is space to graph on the back.
\newpage
\begin{center}
\begin{tikzpicture}
\begin{axis}[
    xmin=-6, xmax=6,
    ymin=-6, ymax=6,
    xtick={-6,...,6},  
    xticklabels= {,,},
    ytick={-6,...,6},
    yticklabels= {,,},
    major tick length={0},
    line width=1pt,
    axis lines=center, height=6in, width = 6in, grid=major
    ]
\end{axis}
\end{tikzpicture}
\end{center}
\end{enumerate}

\end{document}