\documentclass[letterpaper,11pt]{article}
\usepackage{amsmath}
\usepackage{amsthm}
\usepackage{mathtools}
\usepackage[letterpaper,margin=1in,includehead=true]{geometry}
\usepackage{comment}
\usepackage{graphicx}
\usepackage{fancyhdr}
\pagestyle{fancy}
\usepackage{color}
\usepackage{setspace}
\usepackage{tikz,tikz-3dplot,pgfplots}
\usepackage{comment}
\usepackage{multicol}
\usetikzlibrary{patterns}

\DeclarePairedDelimiter\abs{\lvert}{\rvert}

\setlength{\headheight}{15pt}%needed to remove fancyhdr error--not crucial%

\newtheorem{theorem}{Theorem}

\newcounter{mycounter}  
\newenvironment{noindlist}
 {\begin{list}{\arabic{mycounter}.~~}{\usecounter{mycounter} \labelsep=0em \labelwidth=0em \leftmargin=0em \itemindent=0em}}
 {\end{list}}

\newcommand{\unenumerate}[1]{\setcounter{saveenum}{\value{enumi}}\end{enumerate}
	\noindent #1 
	\begin{enumerate} \setcounter{enumi}{\value{saveenum}}}

\newcounter{saveenum}

\def\ds{\displaystyle}

%To print solutions, use \solutionstrue; To hide solutions, use \solutionsfalse.
%\sol takes two arguments. #1 is the vertical length. #2 is the text.

\newif\ifsolutions
\solutionsfalse
\ifsolutions
    \newcommand{\sol}[2]{\begin{minipage}[c][#1]{\linewidth}{\textcolor{blue}{\textbf{Solution:}}\quad \textcolor{blue}{#2}}\end{minipage}}
    \newcommand{\opsol}[1]{#1}
    \newcommand{\tblsol}[1]{\textcolor{blue}{#1}}
\else
    \newcommand{\sol}[2]{\begin{minipage}[c][#1]{\linewidth}{\vfill}\end{minipage}}
    \newcommand{\opsol}[1]{0}
    \newcommand{\tblsol}[1]{\textcolor{white}{#1}}
\fi

\newcommand{\ww}{WeBWorK }
\newcommand{\ddx}{\frac{d}{dx}}
\newcommand{\dydx}{\frac{dy}{dx}}
\newcommand{\be}{\begin{enumerate}}
\newcommand{\ee}{\end{enumerate}}
\newcommand{\vs}[1]{\vspace{#1 pt}}

\begin{document}
\lhead{Math 241: Calculus I}
\rhead{\bf Worksheet 12} 

You will find the following two theorems useful.

\begin{theorem}[Rolle's Theorem]
    Let $f$ be a function that satisfies the following three hypotheses:
    \begin{enumerate}
        \item $f$ is continuous on the closed interval $[a,b]$.
        \item $f$ is differentiable on the open interval $(a,b)$.
        \item $f(a) = f(b)$.
    \end{enumerate}
    Then there is a number $c$ in $(a,b)$ such that $f'(c) = 0$.
\end{theorem}

\begin{theorem}[Mean Value Theorem]
    Let $f$ be a function that satisfies the following hypotheses:
    \begin{enumerate}
        \item $f$ is continuous on the closed interval $[a,b]$.
        \item $f$ is differentiable on the open interval $(a,b)$.
    \end{enumerate}
    Then there is a number $c$ in $(a,b)$ such that
    \[f'(c) = \frac{f(b) - f(a)}{b-a}.\]
    That is, the there is a point $c$ such that the instantaneous slope is the average slope over $a$ to $b$.
\end{theorem}

\begin{enumerate}
    \item Draw the graph of a function defined on $[0,8]$ such that $f(0) = f(8) = 3$ and the function does not satisfy the conclusion of Rolle's Theorem on $[0,8]$.
    \begin{center}
    \begin{tikzpicture}
    \begin{axis}[
        xmin=-0.5, xmax=8.5,
        ymin=-4.5, ymax=4.5,
        xtick={0,...,8},  
        ytick={-4,...,4},
        major tick length={0},
        line width=1pt,
        axis lines=center, height=3in, width = 5.5in, grid=major
        ];
    \end{axis}
    \end{tikzpicture}
    \end{center}
    \newpage
    \item The graph of a function $g$ is shown below.
    \begin{center}
    \begin{tikzpicture}
    \begin{axis}[
        xmin=-0.5, xmax=8.5,
        ymin=-0.5, ymax=12,
        % xtick={0,...,8},  
        % ytick={0,...,12},
        major tick length={0},
        line width=1pt,
        axis lines=center, height=3in, width = 5.5in, grid=major
        ]
        \addplot[
            domain = 0:8,
            samples = 100,
        ]
        {(1/32)*(x+1)^2*(x-6)^2 + 1};
    \end{axis}
    \end{tikzpicture}
    \end{center}
    \begin{enumerate}
        \item Verify that $g$ satisfies the Mean Value Theorem on the interval $[0,8]$.
        \vfill
        \item Estimate the value(s) of $c$ that satisfy the conclusion of the Mean Value Theorem on the interval $[0,8]$ and plot the point on the graph above.
        \vfill
        \item Estimate the value(s) of $c$ that satisfy the conclusion of the Mean Value Theorem on the interval $[2,6]$ and plot the point on the graph above.
        \vfill
    \end{enumerate}
    \newpage
    \item Verify that the function satisfies the hypotheses of Rolle's Theorem on the given interval. Then find all numbers $c$ that satisfy the conclusion of Rolle's Theorem.
    \begin{enumerate}
        \item $f(x) = x^3 - 2x^2 -4x +2, \quad [-2,2]$.
        \vfill
        \item $f(x) = x + \frac{1}{x}, \quad [\frac{1}{2}, 2]$.
        \vfill
    \end{enumerate}
    \item Verify that the function satisfies the hypotheses of Mean Value Theorem on the given interval. Then find all numbers $c$ that satisfy the conclusion of Mean Value Theorem.
    \begin{enumerate}
        \item $f(x) = x^3 - 3x + 2, \quad [-2,2]$.
        \vfill
        \item $f(x) = \frac{1}{x}, \quad [1,3]$.
        \vfill 
    \end{enumerate}
    \newpage
    \item Show that the equation $f(x) = 2x - 1 - \sin{x} = 0$ has exactly one root.
    \vfill
    \item A number $a$ is called a \textbf{fixed point} of a function $f$ if $f(a) = a$. Prove that if $f'(x) \neq 1$ for all real numbers $x$, then $f$ has at most one fixed point.
    \vfill
\end{enumerate}
\end{document}
