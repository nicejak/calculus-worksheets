\documentclass[letterpaper,11pt]{article}
\usepackage{amsmath}
\usepackage{mathtools}
\usepackage[letterpaper,margin=1in,includehead=true]{geometry}
\usepackage{comment}
\usepackage{graphicx}
\usepackage{fancyhdr}
\pagestyle{fancy}
\usepackage{color}
\usepackage{setspace}
\usepackage{tikz,tikz-3dplot,pgfplots}
\usepackage{comment}
\usepackage{multicol}
\usetikzlibrary{patterns}

\DeclarePairedDelimiter\abs{\lvert}{\rvert}

\setlength{\headheight}{15pt}%needed to remove fancyhdr error--not crucial%



\newcounter{mycounter}  
\newenvironment{noindlist}
 {\begin{list}{\arabic{mycounter}.~~}{\usecounter{mycounter} \labelsep=0em \labelwidth=0em \leftmargin=0em \itemindent=0em}}
 {\end{list}}

\newcommand{\unenumerate}[1]{\setcounter{saveenum}{\value{enumi}}\end{enumerate}
	\noindent #1 
	\begin{enumerate} \setcounter{enumi}{\value{saveenum}}}

\newcounter{saveenum}

\def\ds{\displaystyle}

%To print solutions, use \solutionstrue; To hide solutions, use \solutionsfalse.
%\sol takes two arguments. #1 is the vertical length. #2 is the text.

\newif\ifsolutions
\solutionsfalse
\ifsolutions
    \newcommand{\sol}[2]{\begin{minipage}[c][#1]{\linewidth}{\textcolor{blue}{\textbf{Solution:}}\quad \textcolor{blue}{#2}}\end{minipage}}
    \newcommand{\opsol}[1]{#1}
    \newcommand{\tblsol}[1]{\textcolor{blue}{#1}}
\else
    \newcommand{\sol}[2]{\begin{minipage}[c][#1]{\linewidth}{\vfill}\end{minipage}}
    \newcommand{\opsol}[1]{0}
    \newcommand{\tblsol}[1]{\textcolor{white}{#1}}
\fi

\newcommand{\ww}{WeBWorK }
\newcommand{\ddx}{\frac{d}{dx}}
\newcommand{\dydx}{\frac{dy}{dx}}
\newcommand{\be}{\begin{enumerate}}
\newcommand{\ee}{\end{enumerate}}
\newcommand{\vs}[1]{\vspace{#1 pt}}

\begin{document}
\lhead{Math 241: Calculus I}
\rhead{\bf Worksheet 11} 
\begin{enumerate}
    \item Sketch a graph of a function that is continuous on $[1,5]$ and has the following properties:
    \begin{enumerate}
        \item absolute maximum at 5, absolute minimum at 2, local maximum at 3, local minima at 2 and 4.
        \begin{center}
        \begin{tikzpicture}
        \begin{axis}[
           	xmin=0, xmax=5.5,
        	ymin=-3, ymax=3,
        	xtick={0,...,5},  
        	ytick={-3,...,3},
        	major tick length={0},
        	line width=1pt,
         	axis lines=center, height=3in, grid=major
        	];
        \end{axis}
        \end{tikzpicture}
        \end{center}
        \item absolute maximum at 2, absolute minimum at 5, 4 is a critical number but there is no local maximum or minimum there.
        \begin{center}
        \begin{tikzpicture}
        \begin{axis}[
           	xmin=0, xmax=5.5,
        	ymin=-3, ymax=3,
        	xtick={0,...,5},  
        	ytick={-3,...,3},
        	major tick length={0},
        	line width=1pt,
         	axis lines=center, height=3in, grid=major
        	];
        \end{axis}
        \end{tikzpicture}
        \end{center}
    \end{enumerate}
    \newpage
    \item \begin{enumerate}
        \item Sketch the graph of a function on $[-1,2]$ that has an absolute maximum but no absolute minimum.
        \begin{center}
        \begin{tikzpicture}
        \begin{axis}[
           	xmin=-1.5, xmax=2.5,
        	ymin=-3, ymax=3,
        	xtick={-1,...,2},  
        	ytick={-3,...,3},
        	major tick length={0},
        	line width=1pt,
         	axis lines=center, height=3 in, grid=major
        	];
        \end{axis}
        \end{tikzpicture}
        \end{center}
        \item Sketch the graph of a function on $[-1,2]$ that is discontinuous but has both an absolute maximum and and absolute minimum.
        \begin{center}
        \begin{tikzpicture}
        \begin{axis}[
           	xmin=-1.5, xmax=2.5,
        	ymin=-3, ymax=3,
        	xtick={-1,...,2},  
        	ytick={-3,...,3},
        	major tick length={0},
        	line width=1pt,
         	axis lines=center, height=3 in, grid=major
        	];
        \end{axis}
        \end{tikzpicture}
        \end{center}
    \end{enumerate}
    \newpage
    \item What are the critical points of the function $f(x) = 2x^3 - 3x^2 -36x$?

    \sol{}{$f'(x) = 6x^2 -6x -36 = 6(x^2 -x -6) = 6(x-3)(x+2)$ so then the CP are $x=3,-2$.}
    \vfill
    \item Find the absolute maximum and absolute minimum values of $f(x) = 2x^3 - 3x^2 -12x + 1$ on the interval $[-2,3]$.

    \sol{}{$f'(x) = 6x^2 - 6x -12 = 6(x^2-x-2) = 6(x-2)(x+1)$ so the CP are $x=2,-1,-2,3$. The values of $f$ are given as $f(-2) = -3, f(-1) = 8, f(2) = -19, f(3) = -8$. Therefore, the absolute maximum is $f(-1) = 8$ and the absolute minimum is $f(2)=-19$.}
    \vfill
    \item Find the absolute maximum and absolute minimum values of $f(x) = \frac{1}{x}$ on the interval $[-1,1]$.

    \sol{}{$f$ does not have an absolute minimum or maximum value on the interval.}
    \vfill
\end{enumerate}
\end{document}
