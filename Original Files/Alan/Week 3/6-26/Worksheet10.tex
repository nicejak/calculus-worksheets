\documentclass[letterpaper,11pt]{article}
\usepackage{amsmath}
\usepackage{mathtools}
\usepackage[letterpaper,margin=1in,includehead=true]{geometry}
\usepackage{comment}
\usepackage{graphicx}
\usepackage{fancyhdr}
\pagestyle{fancy}
\usepackage{color}
\usepackage{setspace}
\usepackage{tikz,tikz-3dplot,pgfplots}
\usepackage{comment}
\usepackage{multicol}
\usetikzlibrary{patterns}

\DeclarePairedDelimiter\abs{\lvert}{\rvert}

\setlength{\headheight}{15pt}%needed to remove fancyhdr error--not crucial%



\newcounter{mycounter}  
\newenvironment{noindlist}
 {\begin{list}{\arabic{mycounter}.~~}{\usecounter{mycounter} \labelsep=0em \labelwidth=0em \leftmargin=0em \itemindent=0em}}
 {\end{list}}

\newcommand{\unenumerate}[1]{\setcounter{saveenum}{\value{enumi}}\end{enumerate}
	\noindent #1 
	\begin{enumerate} \setcounter{enumi}{\value{saveenum}}}

\newcounter{saveenum}

\def\ds{\displaystyle}

%To print solutions, use \solutionstrue; To hide solutions, use \solutionsfalse.
%\sol takes two arguments. #1 is the vertical length. #2 is the text.

\newif\ifsolutions
\solutionsfalse
\ifsolutions
    \newcommand{\sol}[2]{\begin{minipage}[c][#1]{\linewidth}{\textcolor{blue}{\textbf{Solution:}}\quad \textcolor{blue}{#2}}\end{minipage}}
    \newcommand{\opsol}[1]{#1}
    \newcommand{\tblsol}[1]{\textcolor{blue}{#1}}
\else
    \newcommand{\sol}[2]{\begin{minipage}[c][#1]{\linewidth}{\vfill}\end{minipage}}
    \newcommand{\opsol}[1]{0}
    \newcommand{\tblsol}[1]{\textcolor{white}{#1}}
\fi

\newcommand{\ww}{WeBWorK }
\newcommand{\ddx}{\frac{d}{dx}}
\newcommand{\dydx}{\frac{dy}{dx}}
\newcommand{\be}{\begin{enumerate}}
\newcommand{\ee}{\end{enumerate}}
\newcommand{\vs}[1]{\vspace{#1 pt}}

\begin{document}
\lhead{Math 241: Calculus I}
\rhead{\bf Worksheet 10} 
Given a function $f(x)$ and a point $x = a$, the linearization of $f$ at $a$ is the tangent line at $(a,f(a))$. In other words, the linearization equation is given by
\[L(x) = f'(a)[x-a] + f(a).\]
\begin{enumerate}
\item Find the linearization $L(x)$ of the function $f(x) = \sqrt{x}$ at the point $a=4$.

\sol{}{$L(x) = \frac{1}{4} (x-4) + 2$}.
\vfill
\item Find the linearization of $f(x) = \sqrt{1-x}$ at $a=0$ and use it to approximate $\sqrt{0.99}$. Use a calculator to check that your answer is close to the real value of $\sqrt{0.99}$.

\sol{}{$L(x) = \frac{-1}{2} x + 1$ so $\sqrt{0.99} = f(0.01) = L(0.01) = -0.005 + 1 = 0.995$}
\vfill
\newpage
\item Use linear approximation to estimate $(1.9999)^4.$

\sol{}{$f(x) = x^4$ at $a=2$. Then $L(x) = 4(2)^3(x-2) + 2^4 = 32(x-2) + 16 = 32x - 48$. Estimating by letting $x=1.9999$, we have that $L(1.9999) = 32\times 1.9999 - 48 = 15.9968$.}
\vfill
\item Use linear approximation to estimate $\sqrt[3]{1001}$.

\sol{}{$f(x) = \sqrt[3]{x}$ at $a=1000$. Then $L(x) = \frac{1}{3(\sqrt[3]{(1000)})^2} (x-1000)+ 10 = \frac{1}{300} (x-1000) + 10$. Letting $x = 1001$, we obtain $L(1001) = \frac{1}{300} + 10 = 10.00\overline{3}$.} 
\vfill

\newpage
\item The edge of a cube was found to be $30$cm with a possible error in measurement of $0.1$cm. Use differentials to estimate the maximum possible error in computing:
\begin{enumerate}
    \item the volume of the cube.

    \sol{}{$V = x^3$ so then $dV = 3x^2 dx$. Letting $x = 30$ and $dx = 0.1$, we obtain $dV = 3\cdot 30^2 \cdot 0.1= 270$. Thus, the maximum error in volume is $270$cm cubed.}
    \vfill
    \item the surface area of the cube.

    \sol{}{$A = 6x^2$ so then $dA = 12x dx$. Letting $x=30$ and $dx=  0.1$, we obtain $dA = 12\cdot 30\cdot 0.1 = 36$. Thus, the maximum error in surface area is $36$cm squared.}
    \vfill
\end{enumerate}
\end{enumerate}

\end{document}
