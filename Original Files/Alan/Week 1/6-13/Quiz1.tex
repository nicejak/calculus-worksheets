\documentclass[letterpaper,11pt]{article}
\usepackage{amsmath}
\usepackage[letterpaper,margin=1in,includehead=true]{geometry}
\usepackage{comment}
\usepackage{graphicx}
\usepackage{fancyhdr}
\pagestyle{fancy}
\usepackage{color}
\usepackage{multicol}

%%%%%%%%%%%%%
\usepackage{pgfplots}
\usetikzlibrary{calc}
\usepackage{wrapfig}
\usepackage{enumitem}

\newcounter{mycounter}  
\newenvironment{noindlist}
 {\begin{list}{\arabic{mycounter}.~~}{\usecounter{mycounter} \labelsep=0em \labelwidth=0em \leftmargin=0em \itemindent=0em}}
 {\end{list}}

%To print solutions, use \solutionstrue
%To repress solutions, use \solutionsfalse
%\sol takes two arguments. #1 is the vertical length. #2 is the text.

\newif\ifsolutions
\solutionsfalse
\ifsolutions
    \newcommand{\sol}[2]{\begin{minipage}[c][#1]{\linewidth}{\textcolor{blue}{\textbf{Solution:}}\quad \textcolor{blue}{#2}}\end{minipage}}
    \newcommand{\opsol}{1}
\else
    \newcommand{\sol}[2]{\begin{minipage}[c][#1]{\linewidth}{\vfill}\end{minipage}}
    \newcommand{\opsol}{0}
\fi

\newcommand{\unenumerate}[1]{\setcounter{saveenum}{\value{enumi}}\end{enumerate}
	\noindent #1 
	\begin{enumerate} \setcounter{enumi}{\value{saveenum}}}

\newcounter{saveenum}

\def\changemargin#1#2{\list{}{\rightmargin#2\leftmargin#1}\item[]}
\let\endchangemargin=\endlist 

\begin{document}

\lhead{\bf Math 241: Calculus I}
\rhead{\bf Quiz 1: Introduction}

This summer, there will be six weeks where we all spend ten hours a week with each other! For that reason, we should get to know each other decently well. To do that, we ask the five Ws and the H. Find one other person and interview them to complete this quiz!
\begin{enumerate}
\item Who are you?
\vfill
\item What is your major?
\vfill
\item When did you enter college?
\vfill
\item Where are you from?
\vfill
\item Why are you in this class?
\vfill
\item How is your day?
\vfill
\end{enumerate}

\end{document}


% \documentclass[11pt,letterpaper]{article}
% \usepackage[lmargin=1in,rmargin=1in,tmargin=1in,bmargin=1in]{geometry}

% % -------------------
% % Packages
% % -------------------
% \usepackage{
% 	amsmath,			% Math Environments
% 	amssymb,			% Extended Symbols
% 	enumerate,		    % Enumerate Environments
% 	graphicx,			% Include Images
% 	lastpage,			% Reference Lastpage
% 	multicol,			% Use Multi-columns
% 	multirow			% Use Multi-rows
% }


% % -------------------
% % Font
% % -------------------
% \usepackage[T1]{fontenc}
% \usepackage{charter}


% % -------------------
% % Heading Commands
% % -------------------
% \newcommand{\class}{Math 241}
% \newcommand{\term}{Summer 2023}
% \newcommand{\instructor}{Alan Tong}
% \newcommand{\head}[2]{%
% \thispagestyle{empty}
% \vspace*{-0.5in}
% \noindent\begin{tabular*}{\textwidth}{l @{\extracolsep{\fill}} r @{\extracolsep{6pt}} l}
% 	\textbf{#1} & \textbf{Name:} & \makebox[8cm]{\hrulefill} \\
% 	\textbf{#2} & & \\
% 	\textbf{\class:\; \term} & & 
% \end{tabular*} \\
% \rule[2ex]{\textwidth}{2pt} %
% }


% % -------------------
% % Commands
% % -------------------
% \newcommand{\prob}{\noindent\textbf{Problem. }}
% \newcounter{problem}
% \newcommand{\problem}{
% 	\stepcounter{problem}%
% 	\noindent \textbf{Problem \theproblem. }%
% }
% \newcommand{\pointproblem}[1]{
% 	\stepcounter{problem}%
% 	\noindent \textbf{Problem \theproblem.} (#1 points)\,%
% }
% \newcommand{\pspace}{\par\vspace{\baselineskip}}
% \newcommand{\ds}{\displaystyle}


% % -------------------
% % Header & Footer
% % -------------------
% \usepackage{fancyhdr}

% \fancypagestyle{pages}{
% 	%Headers
% 	\fancyhead[L]{}
% 	\fancyhead[C]{}
% 	\fancyhead[R]{}
% \renewcommand{\headrulewidth}{0pt}
% 	%Footers
% 	\fancyfoot[L]{}
% 	\fancyfoot[C]{}
% 	\fancyfoot[R]{}
% \renewcommand{\footrulewidth}{0.0pt}
% }
% \headheight=0pt
% \footskip=14pt

% \pagestyle{pages}

% \renewcommand{\arraystretch}{1.5}

% % -------------------
% % Content
% % -------------------
% \begin{document}
% \head{Quiz 1}{Week \#: 6/13/2023}
% \textbf{Introduction:} \par \noindent This summer, there will be six weeks where we all spend ten hours a week with each other! For that reason, we should get to know each other decently well. To do that, we ask the five Ws and the H. Find one other person and interview them to complete this quiz! \pspace

% \problem Who are you?
% \vfill

% \problem What is your major?
% \vfill

% \problem When did you enter college?
% \vfill

% \problem Where are you from?
% \vfill

% \problem Why are you in this class?
% \vfill

% \problem How is your day?
% \vfill
% \end{document}			