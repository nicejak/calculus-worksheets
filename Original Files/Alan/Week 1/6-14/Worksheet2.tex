\documentclass[letterpaper,11pt]{article}
\usepackage{amsmath}
\usepackage[letterpaper,margin=1in,includehead=true]{geometry}
\usepackage{comment}
\usepackage{graphicx}
\usepackage{fancyhdr}
\pagestyle{fancy}
\usepackage{color}
\usepackage{multicol}

%%%%%%%%%%%%%
\usepackage{pgfplots}
\usetikzlibrary{calc}
\usepackage{wrapfig}
\usepackage{enumitem}

\newcounter{mycounter}  
\newenvironment{noindlist}
 {\begin{list}{\arabic{mycounter}.~~}{\usecounter{mycounter} \labelsep=0em \labelwidth=0em \leftmargin=0em \itemindent=0em}}
 {\end{list}}

%To print solutions, use \solutionstrue
%To repress solutions, use \solutionsfalse
%\sol takes two arguments. #1 is the vertical length. #2 is the text.

\newif\ifsolutions
\solutionstrue
\ifsolutions
    \newcommand{\sol}[2]{\begin{minipage}[c][#1]{\linewidth}{\textcolor{blue}{\textbf{Solution:}}\quad \textcolor{blue}{#2}}\end{minipage}}
    \newcommand{\opsol}{1}
\else
    \newcommand{\sol}[2]{\begin{minipage}[c][#1]{\linewidth}{\vfill}\end{minipage}}
    \newcommand{\opsol}{0}
\fi

\newcommand{\unenumerate}[1]{\setcounter{saveenum}{\value{enumi}}\end{enumerate}
	\noindent #1 
	\begin{enumerate} \setcounter{enumi}{\value{saveenum}}}

\newcounter{saveenum}

\def\changemargin#1#2{\list{}{\rightmargin#2\leftmargin#1}\item[]}
\let\endchangemargin=\endlist 

\begin{document}

\lhead{\bf Math 241: Calculus I}
\rhead{\bf Worksheet 2: Tangents and Velocity}

\noindent \textbf{Definition:} (Tangent)

\noindent The word tangent is derived from the Latin word \textit{tangens}, which means touching. Therefore, a tangent to a curve is a line that touches a curve. In other words, a tangent line should have the same direction as the curve at the point of contact.

\begin{enumerate}
\item Consider the graph of a unit circle.
\begin{center}
    \begin{tikzpicture}
\begin{axis}[
    axis lines = center,
    xlabel = \(x\),
    ylabel = {\(f(x)\)},
    xmin=-1.5, xmax=1.5,
    ymin=-1.5, ymax=1.5,
    ymajorgrids=true,
    xmajorgrids=true,
    xtick={-1.5,-1,-0.5,0.5,1,1.5},
    ytick={-1.5,-1,-0.5,0.5,1,1.5},
]
\addplot [
    domain=-1:1, 
    samples=100, 
    color=black,
]
{sqrt(1-x^2)};
\addplot [
    domain=-1:1, 
    samples=100, 
    color=black,
]
{-sqrt(1-x^2)};
\end{axis}
\end{tikzpicture}
\end{center}
\begin{enumerate}[label = \alph*.]
    \item What is the equation of the tangent at $(0,1)$? \\
    \ifsolutions \textcolor{blue} { $y=1$ } \fi
    \vspace{1.5in}
    \item What is the equation of the tangent at $(1,0)$? \\
    \ifsolutions \textcolor{blue} { $x=1$ } \fi
    \vspace{1.5in}
    \newpage
    \item What is the equation of the tangent at $(\frac{\sqrt{2}}{2}, \frac{\sqrt{2}}{2})$? You may need to make an educated guess on the slope $m$.\\
    \ifsolutions \textcolor{blue} { $y=-1(x-\frac{\sqrt{2}}{2}) + \frac{\sqrt{2}}{2} = \sqrt{2} - x$ } \fi
    \vspace{1.5in}
    \item Regarding the position and slope, what are the two rules that a tangent line must follow?\\
    \ifsolutions \textcolor{blue} { The line must touch the point and the slope must match the curve. } \fi
    \vspace{1.5in}
\end{enumerate}
\item Consider the function $y=x^2$ and try to draw its tangent line at the point $(1,1)$.
\begin{center}
    \begin{tikzpicture}
\begin{axis}[
    axis lines = center,
    xlabel = \(x\),
    ylabel = {\(f(x)\)},
    xmin=-3, xmax=3,
    ymin=-3, ymax=3,
    ymajorgrids=true,
    xmajorgrids=true,
    xtick={-3,-2,-1,1,2,3},
    ytick={-3,-2,-1,1,2,3},
]
\addplot [
    domain=-3:3, 
    samples=100, 
    color=black,
]
{x^2};
% \addplot [
%     domain=-3:3, 
%     samples=100, 
%     color=red,
%     ]
%     {2*(x-1)+1};
\end{axis}
\end{tikzpicture}
\end{center}
\newpage
\begin{enumerate}[label = \alph*.]
    \item Although a sketch is nice, a larger goal is to create an equation for the tangent line (perhaps so that a computer may automate the process). First, discover a process to \textbf{estimate} the tangent line using the point-slope equation $y - y_1 = m(x-x_1)$ and the average slope equation $m = \frac{y_1-y_2}{x_1-x_2}$. Then, describe the process below.\\
    \ifsolutions \textcolor{blue} { Insert the point $(1,1)$ in for $(x_1, y_1)$. Then, estimate the slope $m$ near $(1,1)$ using a close point, such as $(1.1,1.21)$. } \fi
    \vfill
    \item Why cannot the slope be exact? When trying to solve for an exact solution, what goes wrong? \\
    \ifsolutions \textcolor{blue} { When using the two same points for the slope, you will divide by zero. } \fi
    \vfill
\end{enumerate}
\newpage
\item The point $P = (0.5,0)$ lies on the curve $y = \cos(\pi x)$.
\begin{enumerate}[label = \alph*.]
    \item Let $Q$ be the point $Q = (x,\cos(\pi x))$. Use a calculator to find the slope of the secant line $\overline{PQ}$ (correct to six decimal places) for the following values of $x$
    \begin{center}
    \begin{tabular}{|p{2in}|p{2in}|}
        \hline
        $x$ & Slope\\
        \hline
        \hline
        0 & \ifsolutions \textcolor{blue} {-2} \fi \\
        \hline
        0.4 & \ifsolutions \textcolor{blue} {-3.090170} \fi \\
        \hline
        0.49 & \ifsolutions \textcolor{blue} {-3.141076} \fi \\
        \hline
        0.499 & \ifsolutions \textcolor{blue} {-3.141587} \fi \\
        \hline
        1 & \ifsolutions \textcolor{blue} {-2} \fi \\
        \hline
        0.6 & \ifsolutions \textcolor{blue} {-3.090170} \fi\\
        \hline
        0.51 & \ifsolutions \textcolor{blue} {-3.141080} \fi \\
        \hline
        0.501 & \ifsolutions \textcolor{blue} {-3.141587} \fi\\
        \hline
    \end{tabular}
    \end{center}
    \item Using the results from a., guess the value of the slope of the tangent line to the curve at $P$.\\
    \ifsolutions \textcolor{blue} {$m = -\pi$} \fi
    \vfill
    \item Using the slope from b., find an equation of the tangent line to the curve at $P$.\\
    \ifsolutions \textcolor{blue} {$y = -\pi(x-0.5) = -\pi x + \frac{\pi}{2}$} \fi
    \vfill
    \newpage
    \item The secant line is the line that goes through the same point as the tangent does, but is perpendicular. Find the equation of the secant line.\\
    \ifsolutions \textcolor{blue} {$y = \frac{1}{\pi}(x-0.5) = \frac{1}{\pi} x - \frac{1}{2\pi}$} \fi
    \vfill
    \item Sketch the original curve $y$, as well as the tangent line and the secant line.
    \begin{center}
    \begin{tikzpicture}
    \begin{axis}[
        axis lines = center,
        xlabel = \(x\),
        ylabel = {\(f(x)\)},
        xmin=-2, xmax=2,
        ymin=-2, ymax=2,
        ymajorgrids=true,
        xmajorgrids=true,
        xtick={-2,-1,1,2},
        ytick={-2,-1,1,2},
    ]
    \ifsolutions {\addplot [
        domain=-3:3, 
        samples=100, 
        color=black,
    ]
    {cos(deg(3.141592*x))};
    \addplot [
        domain = -3:3,
        samples = 100,
        color = blue,
        ]
    {-3.141592*x + 3.141592/2};
    \addplot [
        domain = -3:3,
        samples = 100,
        color = red,
        ]
    {(1/3.141592)*x + 1/(2*3.141592)};
    }\fi
    \end{axis}
    \end{tikzpicture}
    \end{center}
    \vfill
\end{enumerate}
\newpage
\item Average velocity and instantaneous velocity are \textbf{not} the same. Explain the difference between the two, and which one should be easier to find.\\
\ifsolutions \textcolor{blue} { Average velocity is calculated across a range, while instantaneous velocity is calculated at only one point. Average velocity is easier to calculate with the current methods at our disposal. } \fi
\vfill
\item Over four centuries ago, Galileo observed that a free falling body has distance fallen proportional to the square of time it has been falling. The exact equation (neglecting air resistance) is
\[s(t) = 4.9t^2 \text{ meters}.\]
\begin{enumerate}[label = \alph*.]
    \item Calculate the average velocities for the given time frames. Feel free to use a calculator.\\
        \begin{center}
            \begin{tabular}{|p{2in}|p{2in}|}
                \hline
                Time interval & Average velocity \\
                \hline
                \hline
                $5 \leq t \leq 6$ & \ifsolutions \textcolor{blue} { 53.9 } \fi\\
                \hline
                $5 \leq t \leq 5.1$& \ifsolutions \textcolor{blue} { 49.49 } \fi\\
                \hline
                $5 \leq t \leq 5.05$& \ifsolutions \textcolor{blue} { 49.245 } \fi \\
                \hline
                $5 \leq t \leq 5.01$& \ifsolutions \textcolor{blue} { 49.049 } \fi\\
                \hline
                $5 \leq t \leq 5.001$& \ifsolutions \textcolor{blue} { 49.0049 } \fi\\
                \hline
            \end{tabular}
        \end{center}
    \vfill
    \newpage
    \item Neglecting air resistance, what is the instantaneous velocity of a falling ball at 5 seconds?\\
    \ifsolutions \textcolor{blue} { 49 m/s } \fi
    \vfill
    \item The difference quotient for a function $f$ is defined to be
    \[Q = \frac{f(x+h) -f(x)}{h}.\]
    Hopefully, you have seen this before :\^{}). Why do I mention this? Is there an "unsimplified" version of this equation that makes more sense?\\
    \ifsolutions \textcolor{blue} { The difference quotient is the process in which the instantaneous velocity or tangent slopes are calculated. When $h\to 0$, we approach the instantaneous velocity or tangent slope exactly. } \fi
    \vfill
\end{enumerate}

\end{enumerate}

\end{document}
