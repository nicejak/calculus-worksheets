\documentclass[letterpaper,11pt]{article}
\usepackage{amsmath}
\usepackage[letterpaper,margin=1in,includehead=true]{geometry}
\usepackage{comment}
\usepackage{graphicx}
\usepackage{fancyhdr}
\pagestyle{fancy}
\usepackage{color}
\usepackage{multicol}

%%%%%%%%%%%%%
\usepackage{pgfplots}
\usetikzlibrary{calc}
\usepackage{wrapfig}
\usepackage{enumitem}

\newcounter{mycounter}  
\newenvironment{noindlist}
 {\begin{list}{\arabic{mycounter}.~~}{\usecounter{mycounter} \labelsep=0em \labelwidth=0em \leftmargin=0em \itemindent=0em}}
 {\end{list}}

%To print solutions, use \solutionstrue
%To repress solutions, use \solutionsfalse
%\sol takes two arguments. #1 is the vertical length. #2 is the text.

\newif\ifsolutions
\solutionstrue
\ifsolutions
    \newcommand{\sol}[2]{\begin{minipage}[c][#1]{\linewidth}{\textcolor{blue}{\textbf{Solution:}}\quad \textcolor{blue}{#2}}\end{minipage}}
    \newcommand{\opsol}{1}
\else
    \newcommand{\sol}[2]{\begin{minipage}[c][#1]{\linewidth}{\vfill}\end{minipage}}
    \newcommand{\opsol}{0}
\fi

\newcommand{\unenumerate}[1]{\setcounter{saveenum}{\value{enumi}}\end{enumerate}
	\noindent #1 
	\begin{enumerate} \setcounter{enumi}{\value{saveenum}}}

\newcounter{saveenum}

\def\changemargin#1#2{\list{}{\rightmargin#2\leftmargin#1}\item[]}
\let\endchangemargin=\endlist 

\begin{document}

\lhead{\bf Math 241: Calculus I}
\rhead{\bf Worksheet 4: Continuity and the IVT}

\begin{enumerate}
    \item \begin{enumerate}[label = \alph*.]
        \item From the graph of $f$ below, state the numbers at which $f$ is discontinuous and explain why.
        \begin{center}
        \begin{tikzpicture}
        \begin{axis}[
            axis lines = center,
            xlabel = \(x\),
            ylabel = {\(f(x)\)},
            xmin=-5, xmax=8,
            ymin=-1.1, ymax=3.1,
            ymajorgrids=true,
            xmajorgrids=true,
            xtick={-4,-2,0,2,4,6},
            ytick={},
        ]
        \addplot [
            domain=-5:-2, 
            samples=100, 
            color=black,
        ]
        {x/2+3};
    
        \addplot [
            domain=-2:0, 
            samples=100, 
            color=black,
        ]
        {x/2+2};
    
        \addplot [
            domain=0:2, 
            samples=100, 
            color=black,
        ]
        {1/(x+0.5)};
    
        \addplot [
            domain=2:3.99, 
            samples=100, 
            color=black,
        ]
        {1/(x-4)+3};
    
        \addplot [
            domain=4:8, 
            samples=100, 
            color=black,
        ]
        {8/x};
        
        \addplot[
            only marks,
            color=black,
            mark=o,
            ]
            coordinates {
            (-4,1)(-2,1)(2,1/2.5)
            };
    
        \addplot[
            only marks,
            color=black,
            mark=*,
            ]
            coordinates {
            (-2,2)(2,2.5)(4,2)
            };
        \end{axis}
        \end{tikzpicture}
        \end{center}
        \sol{0in}{$x=-4,-2,2,4$}
        \vfill

        \item For each of the numbers stated in part a., determine whether $f$ is continuous from the right, continuous from the left, or neither.
        
        \sol{0in}{$x=-4$, neither. $x=-2$, left. $x=2$, right. $x=4$, right.}
        \vfill
    \end{enumerate}
    \item Consider the function
    \[f(x) = \begin{cases}
        \frac{x^2-x}{x^2-1}, & x \neq 1 \\
        1, & x=1
    \end{cases}.\]
    Is the function continuous everywhere? If not, then where is the function discontinuous? Explain why for either case.
    
    \sol{0in}{The function is not continuous at $x=1$ since $\lim_{x\to 1} f(x) = \frac{1}{2}$ while $f(1) = 1$.}
    \vfill

    \newpage
    \item Consider
    \[f(x) = \begin{cases}
        x+2, & x < 0 \\
        2x^2, & 0 \leq x \leq 1 \\
        2 - x, & x > 1
    \end{cases}.\]
    Find the numbers at which $f$ is discontinuous. At which of these numbers is $f$ continuous from the right, continuous from the left, or neither.

    \sol{0in}{$f$ is discontinuous at both $x=0,1$. At $x=0$, $f$ is right continuous. At $x=1$, $f$ is left continuous.}
    \vfill

    \item Use the definition of continuity and the properties of limits to show that the function $f$ is continuous on the given interval.
    \[f(x) = x+\sqrt{x-4}, \quad [4,\infty).\]
    \\\\
    \sol{0in}{Let $a \in [4,\infty)$. Then
    \[\lim_{x\to a} x + \sqrt{x-4} = \lim_{x\to a} x + \lim_{x\to a} \sqrt{x-4} = a + \sqrt{a-4} = f(a).\]
    So $f$ is continuous on the interval given.}
    \vfill
    
    \item Use continuity to solve the limit.
    \[\lim_{x\to\pi} \sin(x+\sin{x}).\]

    \sol{0in}{
    \[\lim_{x\to \pi} \sin(x+\sin{x}) = \sin(\lim_{x\to \pi} x + \sin{x}) = \sin(\pi + \sin(\pi)) = 0\]}
    \vfill
    
    \newpage
    
    \item For what value of the constant $c$ is the function $f$ continuous on $(-\infty, \infty)$?
    \[f(x) = \begin{cases}
        cx^2 + 2x, & x < 2\\
        x^3 - cx, & x \geq 2
    \end{cases}.\]
    \sol{1in}{Let $\lim{x\to 2^-} f(x) = \lim{x\to 2^+} f(x)$. Then
    \[4c + 4 = 8 - 2c \implies c = 2/3.\]}
    \vfill

    \item Use intermediate value theorem to show that there is a root of the given equation on the specified interval.
    \begin{enumerate}[label = \alph*.]
        \item $x^4 + x-3=0, \quad (1,2)$.
        
        \sol{.5in}{Let $f(x) = x^4 + x - 3$. Then $f(1) = -1$ and $f(2) = 15$. Then by the intermediate value theorem, since $f$ is continuous on $(1,2)$ and $-1 < 0 < 15$, there exists a value $c \in (1,2)$ such that $f(c) = 0$.}
        \vfill
        \item $x^5 - x^2 - 4 = 0, \quad (-\infty,\infty)$.

        \sol{.5in}{Let $f(x) = x^5 - x^2 - 4$. Then $f(0) = -4$ and $f(2) = 24$. Then by the intermediate value theorem, since $f$ is continuous everywhere and $-4 < 0 < 24$, then there exists a value $c\in (0,2)$ such that $f(c) = 0$.}
        \vfill
    \end{enumerate}

    \newpage

    \item \begin{enumerate}[label = \alph*.]
        \item Show that the absolute value function $F(x) = |x|$ is continuous everywhere.

        \sol{0.5in}{When $x < 0$, $F(x) = -x$, which is continuous. When $x > 0$, $F(x) = x$, which is also continuous. We must only check that the limit matches the value. $\lim_{x\to 0} F(x) = 0 = F(0)$, so $F$ is continuous everywhere.}
        \vfill
        \item Explain why the result above shows that if $f$ is a continuous function on an interval, then $|f|$ is also continuous.

        \sol{0.5in}{$|f|$ is continuous on its domain since we can compose two continuous functions to obtain a continuous function.}
        \vfill
    \end{enumerate}
    
    \item Is there a number that is exactly 1 more than its cube?
    
    \sol{0.5in}{Yes.}
    \vfill
\end{enumerate}

\end{document}
