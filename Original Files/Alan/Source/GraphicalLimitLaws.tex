\documentclass[letterpaper,11pt]{article}
\usepackage[margin=.5 in]{geometry}
\usepackage{calc}

\usepackage{amsmath, mathtools, comment, graphicx, fancyhdr, color, setspace, multicol, hyperref}
\usepackage{tikz,tikz-3dplot,pgfplots}

\usepackage{amssymb}
\usepackage{amsthm}
\usepackage{latexsym}
\usepackage{epsfig}
\usepackage{amscd}
\usepackage{arydshln}
			
\parskip = 0.2in

\def\ds{\displaystyle}

\newif\ifsolutions
\solutionsfalse
\ifsolutions
    \newcommand{\sol}[2]{\begin{minipage}[c][#1]{\linewidth}{\textcolor{blue}{\textbf{Solution:}}\quad\textcolor{blue}{#2}}\end{minipage}}
\else
    \newcommand{\sol}[2]{\begin{minipage}[c][#1]{\linewidth}{\vfill}\end{minipage}}
\fi

\newcommand{\fgraff}{
\begin{minipage}[l][.30\textwidth]{3 in}{
%\begin{center}
\begin{tikzpicture}
\begin{axis}[
   	xmin=-2.25, xmax=3.25,
	ymin=-3.25, ymax=3.25,
	major tick length={0},
	xtick={-3,-2,...,3}, ytick={-3,-2,...,2,3},
	line width=1pt, title={\textbf{Graph of $f$}},
 	axis lines=center, height=2 in, width=3 in, grid=major,
 	restrict y to domain=-3.25:3.25
	]
	\addplot [ black, smooth, very thick, domain = -2:-1] {-(x-1)^2+3};
	
	\addplot [ black, smooth, very thick, domain = -1:1] {-(x-1)^2+2};
	\addplot [ black, smooth, very thick, domain = 1:3.25] {-(x-1)^2+2};
		
 \addplot [black, only marks, very thick, mark=*, mark options={scale=1, fill=white}]
    coordinates{(1,2)(-1,-2)};
  \addplot [black, only marks, very thick, mark=*] coordinates{(1,1)(-1,-1)};
\end{axis}
\end{tikzpicture}
%\end{center}
}
\end{minipage}}

\newcommand{\ggraff}{
\begin{minipage}[l][.30\textwidth]{3 in}{
%\begin{center}
\begin{tikzpicture}
\begin{axis}[
   	xmin=-2.25, xmax=4.25,
	ymin=-3.25, ymax=3.25,
	major tick length={0},
	xtick={-1,0,...,4}, ytick={-3,-2,...,3},
	line width=1pt, title={\textbf{Graph of $g$}},
 	axis lines=center, height=2 in, width=3 in, grid=major,
 	restrict y to domain=-3.25:3.25
	]
	\addplot [black, smooth, very thick] plot coordinates {(-1,-1)(2,2)};
	\addplot [black, smooth, very thick] plot coordinates {(-2,-3)(-1,-2)};
	\addplot [black, smooth, very thick] plot coordinates {(2,3)(4,-3)};
	\addplot [black, only marks, very thick, mark=*, mark options={scale=1, fill=white}] plot coordinates {(2,3)(-1,-2)(-1,-1)};
 \addplot [black, only marks, very thick, mark=*]
    coordinates{(2,2)};
 
\end{axis}
\end{tikzpicture}
%\end{center}
}
\end{minipage}}

% I clipped this macro from an on-line post - used to raise quotes around a fraction
\newsavebox{\mathbox}\newsavebox{\mathquote}
\makeatletter
\newcommand{\mathquotes}[1]{% \mathquotes{<stuff>}
  \savebox{\mathquote}{\text{``}}% Save quotes
  \savebox{\mathbox}{$\displaystyle #1$}% Save <stuff>
  \raisebox{\dimexpr\ht\mathbox-\ht\mathquote\relax}{``}#1\raisebox{\dimexpr\ht\mathbox-\ht\mathquote\relax}{''}
}
\makeatother

\begin{document}

\noindent \textbf{Graphical limits using limit laws} \hfill \textbf{Name: \rule{4cm}{.5 pt}}


\begin{center}
\begin{tabular}{l r}
\fgraff & \ggraff
\end{tabular}
\end{center}

\vspace{-.6 in}

\begin{enumerate}
\item $\ds \lim_{x\rightarrow 0} (f(x)+g(x))$

\sol{.7in}{$\ds \lim_{x\rightarrow 0} (f(x)+g(x)) = \lim_{x\rightarrow 0} f(x) + \lim_{x\rightarrow 0} g(x) = 1 + 0 = 1$}

\item $\ds \lim_{x\rightarrow 1} (f(x)g(x))$

\sol{.7in}{$\ds \lim_{x\rightarrow 1} (f(x)g(x)) = \lim_{x\rightarrow 1} f(x) \cdot \lim_{x\rightarrow 1} g(x) = 2\cdot 1 = 2$}

\item $\ds \lim_{x\rightarrow 1} (f(x)+g(x))$

\sol{.7in}{$\ds \lim_{x\rightarrow 1} (f(x)+g(x)) = \lim_{x\rightarrow 1} f(x) + \lim_{x\rightarrow 1} g(x) = 2+ 1 = 3$}

\item $\ds \lim_{x\rightarrow 2^+} (2f(x)+3g(x))$

\sol{.7in}{$\ds \lim_{x\rightarrow 2^+} (2f(x)+3g(x)) = 2\cdot\lim_{x\rightarrow 2^+} f(x) + 3\cdot\lim_{x\rightarrow 2^+} g(x) = 2\cdot 1 +3\cdot3 = 11$}

\item $\ds \lim_{x\rightarrow 2^-} (x^2+(\ln{x})\cdot g(x))$

\sol{.7in}{$\ds \lim_{x\rightarrow 2^-} (x^2+(\ln{x})\cdot g(x)) = \lim_{x\rightarrow 2^-}x^2 + \lim_{x\rightarrow 2^-} (\ln{x}\cdot g(x)) = 4 + \lim_{x\rightarrow 2^-}\ln{x}\cdot\lim_{x\rightarrow 2^-}g(x) = 4 + (\ln{2})\cdot2= 4 + \ln{4}$}

\item $\ds \lim_{x\rightarrow 2} (f(x)- g(x))$

\sol{.9in}{First evaluate the two one-sided limits:\\
$\ds\lim_{x\rightarrow 2^+} (f(x)-g(x)) = \lim_{x\rightarrow 2^+} f(x)-\lim_{x\rightarrow 2^+} g(x) = 1-3=-2$\\
$\ds\lim_{x\rightarrow 2^-} (f(x)-g(x)) = \lim_{x\rightarrow 2^-} f(x)-\lim_{x\rightarrow 2^-} g(x) = 1-2=-1$\\
We see that $\ds\lim_{x\rightarrow 2^+} (f(x)-g(x)) \neq \lim_{x\rightarrow 2^-} (f(x)-g(x))$, so $\ds\lim_{x\rightarrow 2} (f(x)-g(x))$ does not exist.
}

\newpage

\begin{center}
\begin{tabular}{l r}
\fgraff & \ggraff
\end{tabular}
\end{center}

\vspace{-.6 in}

\item $\ds \lim_{x\rightarrow 3} 
\frac{g(x)}{f(x)}$

\sol{.7in}{$\ds \lim_{x\rightarrow 3} \frac{g(x)}{f(x)} = \frac{\lim_{x\rightarrow 3}g(x)}{\lim_{x\rightarrow 3}f(x)} = \frac{0}{-2}=0$}

\item $\ds \lim_{x\rightarrow 3^+} \frac{f(x)}{g(x)}$

\sol{.8in}{We anticipate an issue because the limit of the denominator is 0, so we'll check the limits of the numerator and denominator separately. $\ds\lim_{x\rightarrow 3^+} f(x) = -2$ and $\ds\lim_{x\rightarrow 3^+} g(x)=0$. The sign of the denominator is negative as $x$ approaches 3 from the right.  We have a non-zero limit divided by a number approaching 0 from below (which we can think of as the form 
$\ds\mathquotes{\frac{-2}{0^-}}$), so $\ds\lim_{x\rightarrow 3^+} \frac{f(x)}{g(x)} = +\infty$.}

\item $\ds \lim_{x\rightarrow 3} \frac{f(x)}{g(x)}$

\sol{1.3in}{From the previous problem, we know that we are dealing with a limit involving infinity, which tells us that we need to consider two one-sided limits.  We already know that the limit from the right is $+\infty$, so next we'll look at the limit from the left. The limit of the numerator is $\ds\lim_{x\rightarrow 3^-} f(x) = -2$ and the limit of the denominator is $\ds\lim_{x\rightarrow 3^-} g(x)=0$. This time the sign of the denominator is positive as $x$ approaches 3 from the left. We have a non-zero limit divided by a number approaching 0 from below (which we can think of as the form $\ds\mathquotes{\frac{-2}{0^+}}$), so $\ds\lim_{x\rightarrow 3^-} \frac{f(x)}{g(x)} = -\infty$. Finally, $\ds \lim_{x\rightarrow 3^+} \frac{f(x)}{g(x)} \neq \lim_{x\rightarrow 3-} \frac{f(x)}{g(x)}$, so $\ds \lim_{x\rightarrow 3} \frac{f(x)}{g(x)}$ does not exist.}

\item $\ds \lim_{x\rightarrow 1} \sqrt{1+f(x)+g(x)}$

\sol{.7in}{$\ds \lim_{x\rightarrow 1} \sqrt{1+f(x)+g(x)} = \sqrt {\lim_{x\rightarrow 1} (1+f(x)+g(x))} = \sqrt{1+ \lim_{x\rightarrow 1}f(x) + \lim_{x\rightarrow 1} g(x)} = {\sqrt{1 + 2+1}=2}$}

\item $\ds \lim_{x\rightarrow -1} (f(x)+g(x))$

\sol{2in}{Since neither $\ds\lim_{x\rightarrow -1} f(x)$ nor $\ds \lim_{x\rightarrow -1} g(x)$ exists, we cannot use limit laws to break apart the limit.  The jumps in both graphs at $x=-1$ hint to us to try two one-sided limits. Since these limits exist, we can then use the limit laws to break apart each of the one-sided limits.
$$\lim_{x\rightarrow-1^+} (f(x)+g(x)) = \lim_{x\rightarrow-1^+} f(x)+ \lim_{x\rightarrow-1^+} g(x) = -2+-1 = -3$$
$$\lim_{x\rightarrow-1^-} (f(x)+g(x)) = \lim_{x\rightarrow-1^-} f(x)+ \lim_{x\rightarrow-1^-} g(x) = -1+-2 = -3$$
Since $\ds\lim_{x\rightarrow-1^+} (f(x)+g(x)) = \lim_{x\rightarrow-1^-} (f(x)+g(x)) = -3$, we have $\ds\lim_{x\rightarrow -1} (f(x)+g(x)) = -3$.  This is a little surprising - the jump discontinuities of the two functions manage to cancel each other out, and $f(x)+g(x)$ does have a limit at $x=-1$.}


\end{enumerate}

\end{document}
