\documentclass[letterpaper,11pt]{article}
\usepackage{amsmath}
\usepackage[letterpaper,margin=1in,includehead=true,top=.75in, bottom=.5in]{geometry}
\usepackage{comment}
\usepackage{graphicx}
\usepackage{fancyhdr}
\pagestyle{fancy}
\usepackage{color}
\usepackage{setspace}
\usepackage{tikz,tikz-3dplot,pgfplots}
\pagenumbering{gobble}

\newcounter{mycounter}  
\newenvironment{noindlist}
 {\begin{list}{\arabic{mycounter}.~~}{\usecounter{mycounter} \labelsep=0em \labelwidth=0em \leftmargin=0em \itemindent=0em}}
 {\end{list}}

%To print solutions, use \solutionstrue
%To repress solutions, use \solutionsfalse
%\sol takes two arguments. #1 is the vertical length. #2 is the text.

\newif\ifsolutions
\solutionsfalse
\ifsolutions
    \newcommand{\sol}[2]{\begin{minipage}[c][#1]{\linewidth}{\textcolor{blue}{\textbf{Solution:}}\quad \textcolor{blue}{#2}}\end{minipage}}
    \newcommand{\fsol}[2]{\includegraphics[scale=#1]{#2sol}}
    \newcommand{\tsol}[2]{\textcolor{blue}{\textbf{Solution:}}\quad\vspace{-.25 in}#2}
\else
    \newcommand{\sol}[2]{\begin{minipage}[c][#1]{\linewidth}{\vfill}\end{minipage}}
    \newcommand{\fsol}[2]{\includegraphics[scale=#1]{#2}}
    \newcommand{\tsol}[2]{#1}
\fi

\newcommand{\unenumerate}[1]{\setcounter{saveenum}{\value{enumi}}\end{enumerate}
	\noindent #1 
	\begin{enumerate} \setcounter{enumi}{\value{saveenum}}}

\newcounter{saveenum}

\def\ds{\displaystyle}

\begin{document}
\lhead{Math 1300: Calculus I}
\chead{}
\rhead{\bf Graphing Using Calculus and Technology} 

\noindent Goal: Use technology to produce a graph of $\displaystyle f(x)=\frac{x^2-4}{2x^3+x+1}$, with all key features labeled.
\begin{enumerate}
\item Without technology, find the following:\\\\
Zeroes of $f(x)$: \underline{\hspace{2in}}\\\\
Horizontal asymptote of $f(x)$: \underline{\hspace{2in}}\\\\
To find vertical asymptotes, solve the equation:\underline{\hspace{2in}}\\

\item  Use technology to find the vertical asymptote(s) of $f(x)$:
\vspace{1in}

\item Use technology to produce a first graph of $f(x)$. Label the scale on the axes.

\tsol{
\begin{center}
\begin{tikzpicture}
\begin{axis}[
    axis lines=center, height=3in, grid=none,
    	yticklabels={,,},
    	xticklabels={,,},
    	scaled y ticks = false,
      	ymin=-5, ymax=5,xmin=-10,xmax=10
	]
    \addplot [opacity=0,blue, smooth, very thick, samples=100, domain=-10:10] {(x^2-4)/(2*x^3+x+1)};
\end{axis}
\end{tikzpicture}
\end{center}}
{
\begin{center}
\begin{tikzpicture}
\begin{axis}[
   axis lines=center, height=3in, grid=none,
    	scaled y ticks = false,
      	ymin=-5, ymax=5,xmin=-10,xmax=10
	]
	    \addplot [blue, smooth, very thick, samples=100, domain=-10:10] {(x^2-4)/(2*x^3+x+1)};
\end{axis}
\end{tikzpicture}
\end{center}}

\item From the graph, do you think there are local extrema? Explain. Yes/No/Not sure yet

\sol{.6in}{It looks like the curve is positive past about $x=2$, then there is a horizontal asymptote.  So there must be a local max somewhere to the right of that positive $x$-intercept. Similarly, it looks like there is local min somewhere to the left of the negative $x$-intercept}.

\item From the graph, do you think there are inflection points? Explain.  Yes/No/Not sure yet

\sol{.6in}{If there really is a local maximum and a local minimum, then there also must be an inflection point to the right of the local max and to the left of the local min.  I can't tell if there are more than those two.}

\newpage

\item Use technology to calculate $f'(x)$. Then use technology to find all critical numbers of $f(x)$.\\\\
$f'(x) = $\\\\
Critical numbers of $f'(x)$: \\

\item  Use technology to graph $f'(x)$, then use this graph to classify the critical numbers you found:\\\\
$f(x)$ has a local maximum/minimum value of \underline{\hspace{1in}} at $x=$\underline{\hspace{1in}} because $f'(x)$ switches from \underline{\hspace{1in}} to \underline{\hspace{1in}} there.\\\\
$f(x)$ has a local maximum/minimum value of \underline{\hspace{1in}} at $x=$\underline{\hspace{1in}} because $f'(x)$ switches from \underline{\hspace{1in}} to \underline{\hspace{1in}} there.\\

\item  From the graph of $f'(x)$, how many inflection points do you predict? Explain.\\  none/two/four or more/not sure yet

\sol{1in}{I see that $f'(x)$ has a local min and local max between $x=0$ and $x=1$. So $f(x)$ has at least two inflection points. It also looks to me like $f'(x)$ is negative to the right of about $x=4$, then has a horizontal asymptote.  So it has to have another local min, it looks like it is to the right of $x=6$. Similarly, it looks like there is another local min at least to the left of $x=-6$. So I think there are at least four inflection points.}

\item  Use technology to calculate $f''(x)$. Then use technology to find where $f''(x)=0$. Explain why the $x$-values where $f''(x)$ does not exist are not important here.\\\\
$f''(x)=$\\\\
Zeroes of $f''(x)$ (hint: numerator of $f''(x)$ must be zero):\\

Why zeroes of denominator of $f''(x)$ are not important:\\
\item  For each of the zeroes of $f''(x)$ you found above, determine whether or not $f(x)$ has an inflection point there. Use the graph of $f''(x)$ to justify your conclusions.

\sol{1.5in}{}
\vfill
\item Return to the graph on the first page and label all key points. Draw a new graph if changing the scale is helpful.


\end{enumerate}




\end{document}
