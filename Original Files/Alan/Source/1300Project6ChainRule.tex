\documentclass[letterpaper,11pt]{article}
\usepackage{amsmath, mathtools, comment, graphicx, fancyhdr, color, setspace, comment, multicol, hyperref, tabu}
\usepackage[letterpaper,margin=1in,includehead=true]{geometry}
\pagestyle{fancy}
\usepackage{tikz,tikz-3dplot,pgfplots}

\newcounter{mycounter}  
\newenvironment{noindlist}
 {\begin{list}{\arabic{mycounter}.~~}{\usecounter{mycounter} \labelsep=0em \labelwidth=0em \leftmargin=0em \itemindent=0em}}
 {\end{list}}

%To print solutions, use \solutionstrue
%To hide solutions, use \solutionsfalse
%\sol takes two arguments. #1 is the vertical length. #2 is the text.

\newif\ifsolutions
\solutionsfalse
\ifsolutions
    \newcommand{\sol}[2]{\begin{minipage}[c][#1]{\linewidth}{\textcolor{blue}{\textbf{Solution:}}\quad \textcolor{blue}{#2}}\end{minipage}}
    \newcommand{\fsol}[2]{\includegraphics[scale=#1]{#2sol}}
    \newcommand{\tsol}[2]{\textcolor{blue}{\textbf{Solution:}}\quad\vspace{-.25 in}#2}
\else
    \newcommand{\sol}[2]{\begin{minipage}[c][#1]{\linewidth}{\vfill}\end{minipage}}
    \newcommand{\fsol}[2]{\includegraphics[scale=#1]{#2}}
    \newcommand{\tsol}[2]{#1}
\fi

\def\ds{\displaystyle}

\begin{document}
\lhead{Math 1300: Calculus I}
\chead{}
\rhead{\bf Week 6 Day 4: Chain Rule} 

\begin{enumerate}

\item Let $f(x)=(3x^2+1)^2$. We are going to find the derivative of $f(x)$ in three ways and then compare the answers.

\begin{enumerate}
\item Algebraically multiply out the expression for $f(x)$ and then take the derivative.

\sol{1.5in}{\[f(x)=(3x^2+1)^2=9x^4+6x^2+1 \text{ so } f'(x)=36x^3+12x\]}

\item View $f(x)$ as a product of two functions, $f(x)=(3x^2+1)(3x^2+1)$ and use the product rule to find $f'(x)$.

\sol{1.5 in}{Let $u=3x^2+1$ and $v=3x^2=1$, then $(uv)'=u'v+v'u=(6x)(3x^2+1)+(6x)(3x^2+1)=36x^3+12x$}

\item Apply the chain rule directly to the expression $f(x)=(3x^2+1)^2$

\sol{1.5 in}{$f'(x)=2(3x^2+1)(6x)=36x^3+12x$}

\item Are your answers in parts a, b, and c the same? Why or why not?

\sol{1.5 in}{All the answers are the same because it doesn't matter which method you use to take a derivative. If done correctly, they should all give the same answer.}

\end{enumerate}

\newpage

\item Let $f(x)=\sin(2x).$ We are going to find the derivative of $f(x)$ in two ways and then compare answers.  You will need the double angle formulas for this problem:
\begin{itemize}
\item $\sin{2x}=2\sin{x}\cos{x}$
\item $\cos{2x}=\cos^2{x}-\sin^2{x}$

\end{itemize}

\begin{enumerate}
\item Rewrite $\sin(2x)$ using the double-angle formula, then apply the product rule to find $f'(x)$.

\sol{1.5in}{$f(x)=\sin{2x}=2\sin(x)\cos(x)$, by the sine double angle-formula. Let $u=2\sin(x)$ and $v=\cos(x)$, then $f'(x)=2\cos(x)\cos(x)-2\sin(x)\sin(x)=2(\cos^2(x)-\sin^2(x)$)}

\item Apply the chain rule directly to the expression $f(x)=\sin(2x)$ to find its derivative a second way.

\sol{1.2in}{$f'(x)=\cos(2x) \cdot 2=2\cos(2x)$}

\item Are your answers in parts a and b the same? Why or why not?

\sol{1.2 in}{
Part (a) gives = $f'(x)=2(\cos^2(x)-\sin^2(x))$, by the cosine double-angle formula, $f'(x)=2\cos(2x)$. The two answers are the same because it doesn't matter which method you use to take a derivative. If done correctly, they should all give the same answer.}

\end{enumerate}


\item Suppose $f$ is differentiable and that $g(x) = (f(\sqrt{x}))^3$. 
\begin{enumerate}
\item Calculate $g'(x)$ (your answer will include $f$ and $f'$).

\sol{.5in}{Applying the chain rule twice we get $g'(x)=3(f(\sqrt{x}))^2f'(\sqrt{x})\frac{1}{2\sqrt{x}}$.} 

\item If $f(2)=1$ and $f'(2)=-2$, calculate $g'(4)$.

\sol{.8in}{
$g'(4)=3(f(\sqrt{4}))^2f'(\sqrt{4})\frac{1}{2\sqrt{4}} = 3(f(2))^2f'(2)\frac{1}{4}=3\cdot1^2\cdot-2\cdot\frac{1}{4}=-\frac{3}{2}$.}
\end{enumerate}
\newpage


\item Let $f(x)$ and $g(x)$ be two functions. Values of $f(x)$ and $f'(x)$ are given in the table below and the graph of $g(x)$ is as shown.

\begin{multicols}{3}

\begin{tabular}{c|c c c}
$x$ & 1 & 2 & 3 \\ \hline
$f(x)$ & 3 & 2 & 1 \\ \hline
$f'(x)$ & 4 & 5 & 6 \\
\end{tabular}

\begin{center}
\begin{tikzpicture}
\begin{axis}[
   	xmin=-1, xmax=3.5,
	ymin=-1, ymax=4.5,
	xtick={0,...,3},  
	ytick={0,...,4},
	major tick length={0},
	line width=1pt,
 	axis lines=center, height=2 in, grid=major, ylabel=$g(x)$
	]
    \addplot [ultra thick, samples=9, domain=0:2] {2*x};
    \addplot [ultra thick, samples=9, domain=2:3] {12-4*x};
    \addplot [only marks] table {
0 0
2 4
3 0  
};
\end{axis}
\end{tikzpicture}
\end{center}

\end{multicols}

\begin{enumerate}
\item Let $h(x)=g(f(x))$. Find $h'(3)$.

\sol{1 in}{$h'(3)=g'(f(3)) \cdot f'(3)=g'(1) \cdot f'(3)=2 \cdot 6=12$}

\item Let $k(x)=f(g(x))$.  Find $k'(1)$.

\sol{1in}{$k'(1)=f'(g(1)) \cdot g'(1)=f'(2) \cdot g'(1)=5 \cdot 2=10$.}

\end{enumerate}
\item The US population on July 1 of 2010 was 309.33 million.  The population was 311.59 million on July 1 of 2011.
\begin{enumerate}
\item  Find an exponential model $p(t)$ to fit this data.  Let $t=0$ on July 1, 2010.

\sol{1 in}{We're looking for a function of the form $p(t)=Ae^{kt}$, in millions of people.  Substituting $p(0)=309.33$, we see that $A = 309.33$.  Substituting $p(1) = 311.59$ gives $311.59 = 309.33\cdot e^{k}$.  Solving gives $e^k = 311.59/309.33$, so $k = \ln{(311.59/309.33)} \approx .00728$.  This is an annual growth rate of .728\%.  Our model is $p(t)=309.33\cdot e^{.00728t}$.}

\item  Use your model to estimate the US population on November 1 of 2013.

\sol{.5in}{Substituting $t=3.33$ into $p(t)=309.33e^{.00728t}$ gives $p(3)\approx 316.92$ million people.  The actual value was approximately 316.98 million.}

\item Find $p'(3)$.  Interpret the meaning of this number, including units.

\sol{1 in}{First take the derivative: $p'(t) = .00728\cdot 309.33 e^{.00728t}$.  Substituting $t=3$ gives $p'(3)=.00728\cdot 309.33 e^{.00728\cdot 3} \approx 2.3$.  This means that on July 1 in the year 2013 the rate of change of the US population was approximately 2.3 million people per year.}
\end{enumerate}
\newpage

%\item Suppose $f(x)$ is a differentiable function, $g(x)=f(x)^2$, and $h(x)=f(x)^3 = f(x)g(x)$.

%\begin{enumerate}
%\item Use the product rule to find a formula for $g'(x)$.

%\sol{1.5 in}{
%\begin{align*}
%g(x)&=f(x)f(x) \\
%g'(x)&=f'(x)f(x)+f(x)f'(x) \\
%&=2f(x)f'(x)
%\end{align*}}
%\item Use the product rule and the previous answer to find a formula for $h'(x)$.   Compare your answer to what you would get using the chain rule.

%\sol{1.5in}{
%\begin{align*}
%h'(x)&=[f(x)f(x)]f(x) \\
%h'(x)&=[f(x)f(x)]'f(x)+[f(x)f(x)]f'(x) \\
%&=2f(x)f'(x)f(x)+f(x)f(x)f'(x) \\
%&=3f(x)f(x)f'(x)=3f(x)^2f'(x)
%\end{align*}}
%\item What would you guess is the derivative of $f(x)^n$?  Does this match the result you get using the chain rule?

%\sol{.5 in}{$[f(x)^n]'=nf(x)^{n-1}f'(x)$}

%\end{enumerate}



\item  Chains, Inc. is in the business of making and selling chains.  Let $c(t)$ be the number of miles of chain produced after $t$ hours of production.  Let $p(c)$ be the profit as a function of the number of miles of chain produced, and let $q(t)$ be the profit as a function of the number of hours of production.  
\begin{enumerate}
\item  Suppose the company can produce three miles of chain per hour, and suppose their profit on the chains is \$4000 per mile of chain.  Find each of the following (include units).

$c(t)= $ \ifsolutions \textcolor{blue}{ $3t$ miles}\fi\\

$c'(t)= $ \ifsolutions \textcolor{blue}{ $3$ miles/hour}\fi\\

Meaning of $c'(t)$: \ifsolutions \textcolor{blue}{ 3 feet of chain are produced per hour.}\fi\\

$p(c)= $ \ifsolutions \textcolor{blue}{ $4000c$ dollars}\fi\\

$p'(c)= $ \ifsolutions \textcolor{blue}{ $4000$ dollars/mile}\fi\\

Meaning of $p'(c)$: \ifsolutions \textcolor{blue}{ $4000$ dollars of profit are earned per mile of chain produced.}\fi\\

$q(t) =$ \ifsolutions \textcolor{blue}{ $12000t$ dollars}\fi\\

$q'(t)=$ \ifsolutions \textcolor{blue}{ $12000$ dollars/hour}\fi\\

Meaning of $q'(t)$: \ifsolutions \textcolor{blue}{ $12000$ dollars of profit are earned per hour of production.}\fi\\

How does $q'(t)$ relate to $p'(c)$ and $c'(t)$? 

\sol{.3in}{By the chain rule $q'(t)$ = $p'(c(t))c'(t)$.  So $q'(t) = 4000$ dollars/mile  $\cdot 3$ miles/hour = 12000 dollars/hours.}
\item  In this part, the production and profit functions are no longer linear.  Instead $p(c)$ is modeled by the formula $p(c)=100-100\cos{(\frac{\pi}{38} c)}$ (where $p$ is measured in thousands of dollars and $c$ is measured in miles of chain), and $c(t)$ is defined numerically below:

\medskip
\begin{tabular}{|c|c|c|c|c|c|} \hline
$t$ (in hours) & 2 & 4  & 6  & 8 &10\\ \hline
$c$ (in miles) & 6 & 14 & 24 & 38 & 52\\ \hline
\end{tabular}

\medskip
Estimate $q'(4)$ and $q'(8)$.  What conclusions should you draw about production?

\sol{1.6in}{First note that $p'(c)=\frac{100\pi}{38}\sin{(\frac{\pi}{38}c)}$.\\
Using the chain rule, $q'(4)=p'(c(4))c'(4)$. Estimating numerically, $c'(4) \approx \frac{24-6}{6-2}=\frac{18}{4}$.\\
So $q'(4) \approx \frac{100\pi}{38}\sin{(\frac{\pi}{38}\cdot 14)}\cdot \frac{18}{4}\approx 34.07$ thousand dollars/hour.  This means that after 4 hours of production, the profit increases a rate of about $\$34,000$ dollars per added hour of production.  We should keep the factory running.  Similarly, we find that $q'(8) = p'(c(8))c'(8) \approx \frac{100\pi}{38}\sin{(\frac{\pi}{38}\cdot 38)}\cdot \frac{28}{4}\approx 0$.  The profit is no longer increasing as we increase the number of hours of production.  We should determine if this is a maximum, and possibly shut down the factory after 8 hours.
}
\end{enumerate}

\end{enumerate}
\end{document}
