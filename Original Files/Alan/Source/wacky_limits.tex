\documentclass[letterpaper,11pt]{article}
\usepackage[margin=1 in]{geometry}
\usepackage{calc}

\usepackage{amsmath, mathtools, comment, graphicx, fancyhdr, color, setspace, multicol, hyperref}
\usepackage{tikz,tikz-3dplot,pgfplots}

\usepackage{amssymb}
\usepackage{amsthm}
\usepackage{latexsym}
\usepackage{epsfig}
\usepackage{amscd}
\usepackage{arydshln}
			
\parskip = 0.2in

\def\ds{\displaystyle}

\newif\ifsolutions
\solutionstrue
\ifsolutions
    \newcommand{\tsol}[1]{\textcolor{red}{#1}}
    \newcommand{\sol}[1]{\textcolor{blue}{\textbf{Solution:}}\quad \textcolor{blue}{#1}}
\else
    \newcommand{\tsol}[1]{\textcolor{white}{#1}}
        \newcommand{\sol}[1]{\vfill}
\fi

\newcommand{\fgraff}{
\begin{minipage}[l][.30\textwidth]{3 in}{
%\begin{center}
\begin{tikzpicture}
\begin{axis}[
   	xmin=-3.25, xmax=3.25,
	ymin=-2.25, ymax=2.25,
	major tick length={0},
	xtick={-3,-2,...,3}, ytick={-2,-1,...,2},
	line width=1pt, title={\textbf{Graph of $f$}},
 	axis lines=center, height=2 in, width=3 in, grid=major,
 	restrict y to domain=-2.25:2.25
	]
	\addplot [mark=*, black, smooth, very thick] plot coordinates {(-3,2)(-2,1)};
	\addplot [mark=*, black, smooth, very thick] plot coordinates {(-2,3)};
	\addplot [mark=*, black, smooth, very thick] plot coordinates {(-2,0)(-1,1)};
	\addplot [mark=*, black, smooth, very thick] plot coordinates {(-1,2)(1,2)};
	\addplot [mark=*, black, smooth, very thick] plot coordinates {(2,-1)(3,-1)};
    \addplot [black, smooth, very thick, samples=299, domain=1:2] {(x-1)^2};
    \addplot [black, only marks, very thick, mark=*, mark options={scale=1, fill=white}]
    coordinates{(1,0) (2,1) (-1,2) (-2,1) (-2,0)};
    \addplot [black, only marks, very thick, mark=*] coordinates{(-2,2)};
\end{axis}
\end{tikzpicture}
%\end{center}
}
\end{minipage}}

\newcommand{\ggraff}{
\begin{minipage}[l][.30\textwidth]{3 in}{
%\begin{center}
\begin{tikzpicture}
\begin{axis}[
   	xmin=-3.25, xmax=3.25,
	ymin=-2.25, ymax=2.25,
	major tick length={0},
	xtick={-3,-2,...,3}, ytick={-2,-1,...,2},
	line width=1pt, title={\textbf{Graph of $g$}},
 	axis lines=center, height=2 in, width=3 in, grid=major,
 	restrict y to domain=-2.25:2.25
	]
	\addplot [mark=*, black, smooth, very thick] plot coordinates {(-3,-1)(-1,-1)};
	\addplot [mark=*, black, smooth, very thick] plot coordinates {(1,-1)(3,1)};
	\addplot [black, very thick, mark=*, mark options={scale=1, fill=white}] plot coordinates {(-1,-2)(1,-2)};
\end{axis}
\end{tikzpicture}
%\end{center}
}
\end{minipage}}

\begin{document}
\pagenumbering{gobble}

\ifsolutions
\noindent \textbf{Wacky Limits} \textcolor{blue}{ - Solutions}
\else
\noindent \textbf{Wacky Limits}
\fi

\noindent Problem: These limits are wacky. Help me understand the key. All I have is the answers and not the reasons why the answers are what they are. Do this by providing the correct mathematical reasons/work explaining how one gets the correct answer.

\begin{center}
\begin{tabular}{l r}
\fgraff & \ggraff
\end{tabular}
\end{center}

\vspace{-.3 in}

\ifsolutions
\textcolor{blue}{We will be using limit laws throughout these solutions. One quirk of the limit laws is that they can only be applied if the individual limits exist.
For example, the limit law equation
\[
  \lim_{x \to a} \dfrac{f(x)}{g(x)} = \frac{\lim\limits_{x \to a} f(x)}{\lim\limits_{x \to a} g(x)}
\]
is only true if both $\lim\limits_{x \to a} f(x)$ and $\lim\limits_{x \to a} g(x)$ exist and $\lim\limits_{x \to a} g(x)$ is not equal to $0$
(otherwise the fraction on the right-hand side doesn't make sense). Accordingly, when we use limit laws, we will use them provisionally. If all limits involved turn out to exist, the limit law will work, and we will have our answer. If they do not exist, the limit law equation will be nonsense and we will have to try something else.}
\fi

\begin{enumerate}
\item $\ds \lim_{x\rightarrow 0} (f(x)+g(x))\tsol{=0}$

\sol{Applying the limit law for sums (provisionally),
\[
  \lim_{x \to 0} (f(x) + g(x)) = \lim_{x \to 0} f(x) + \lim_{x \to 0} g(x).
\]
The value of $f(x)$ is 2 whenever $x$ is near $0$, so as $x$ approaches $0$, $f(x)$ approaches 2. In other words,
$\ds \lim_{x \to 0} f(x) = 2$.
The value of $g(x)$ is $-2$ whenever $x$ is near $0$, so $\ds \lim_{x \to 0} g(x) = -2$.
Substituting into the equation from above,
\[
   \lim_{x \to 0} (f(x) + g(x)) = 2 + (-2) = 0.
\]
}

\item $\ds \lim_{x\rightarrow 2^-} \frac{g(x)}{f(x)}=\lim_{x\rightarrow 2^+} \frac{g(x)}{f(x)}=\lim_{x\rightarrow 2} \frac{g(x)}{f(x)}\tsol{=0}$

\sol{Let's consider $\ds \lim_{x \to 2^-} \frac{g(x)}{f(x)}$ first. This is a left-hand limit, so we want to know the value that
$\dfrac{g(x)}{f(x)}$ approaches when $x$ approaches $2$ from the left. Applying the limit law for fractions,
\[
   \lim_{x \to 2^-} \frac{g(x)}{f(x)} = \frac{\lim\limits_{x \to 2^-} g(x)}{\lim\limits_{x \to 2^-} f(x)}.
\]
To find the value of the limit $\ds \lim_{x \to 2^-} g(x)$, we pick an $x$-value a little less than 2 and see what happens
to the value of $g(x)$ as $x$ increases to $2$. In this case, $g(x)$ approaches 0 as $x$ approaches $2$ from the left,
so $\ds \lim_{x \to 2^-} g(x) = 0$. While it is true that $0$ divided by any number (besides 0) is zero, we are still not done
yet, since we need to confirm that the limit for $f(x)$ exists and is not zero (since in that case the limit law equation would not apply).
To find the value of the limit $\ds \lim_{x \to 2^-} f(x)$, we pick an $x$-value a little less than 2 and see what happens
to the value of $f(x)$ as $x$ increases to $2$. In this case, $f(x)$ approaches 1 as $x$ approaches 2 from the left, so
$\ds \lim_{x \to 2^-} f(x) = 1$. Substituting into the equation from above,
\[
  \lim_{x \to 2^-} \frac{g(x)}{f(x)} = \frac{0}{1} = 0.
\]
Next, we consider the right-hand limit $\ds \lim_{x \to 2^+} \frac{g(x)}{f(x)}$. This time,
\begin{align*}
  \lim_{x \to 2^+} \frac{g(x)}{f(x)} = \frac{\lim\limits_{x \to 2^+} g(x)}{\lim\limits_{x \to 2^+} f(x)}.
\end{align*}
From the graph, $\ds \lim_{x \to 2^+} g(x) = 0$ and $\ds \lim_{x \to 2^+} f(x) = -1$, so
\begin{align*}
  \lim_{x \to 2^+} \frac{g(x)}{f(x)} = \frac{0}{-1} = 0.
\end{align*}
Since both the left- and right-hand limits exist and agree, we have that the two-sided limit
$\ds \lim_{x \to 2} \frac{g(x)}{f(x)}$ is also $0$, which is what we wanted to show.
}

\item  $\ds \lim_{x\rightarrow -1} (f(x)+g(x))\tsol{=0}$

\sol{
Since $\ds \lim_{x \to -1^-} f(x) = 1$ while $\ds \lim_{x \to -1^+} f(x) = 2$, we know that $\ds \lim_{x \to -1} f(x)$ does not exist. This means we cannot apply the limit laws for $\ds \lim_{x \to -1} f(x)$.
We fall back to our standard trick: taking left- and right-hand limits individually. Applying the limit law for
sums to the left-hand limit,
\begin{align*}
   \lim_{x \to -1^-} (f(x) + g(x)) &= \lim_{x \to -1^-} f(x) + \lim_{x \to -1^-} g(x) \\
                                                   &= 1 + (-1) \\
                                                   &= 0.
\end{align*}
Similarly, for the right-hand limit we get
\begin{align*}
\lim_{x \to -1^+} (f(x) + g(x)) &= \lim_{x \to -1^+} f(x) + \lim_{x \to -1^+} g(x) \\
                                                   &= 2 + (-2) \\
                                                   &= 0.
\end{align*}
Since the two one-sided limits exist and agree, the two-sided limit $\ds \lim_{x \to -1}(f(x) + g(x))$ exists
and is also equal to $0$, which is what we wanted to show.
}

\item $\ds \lim_{x\rightarrow -1} \frac{f(x)}{g(x)}\tsol{=-1}$

\sol{
Since $\ds \lim_{x \to -1^-} f(x) = 1$ while $\ds \lim_{x \to -1^+} f(x) = 2$, we know that 
$\ds \lim_{x \to -1} f(x)$ does not exist. This means that we cannot apply the limit laws for $\ds \lim_{x \to -1} f(x)$.
This forces us to fall back to our standard trick: taking left- and right-hand limits individually. Applying the limit
law for quotients to the left-hand limit,
\begin{align*}
  \lim_{x \to -1^-} \frac{f(x)}{g(x)} &= \frac{\lim\limits_{x \to -1^-} f(x)}{\lim\limits_{x \to -1^-} g(x)} \\
     &= \frac{1}{-1} \\
     &= -1.
\end{align*}
Similarly, for the right-hand limit we get
\begin{align*}
  \lim_{x \to -1^+} \frac{f(x)}{g(x)} &= \frac{\lim\limits_{x \to -1^+} f(x)}{\lim\limits_{x \to -1^+} g(x)} \\
     &= \frac{2}{-2} \\
     &= -1.
\end{align*}
Since the two one-sided limits exist and agree, the two-sided limit $\ds \lim_{x \to -1}\frac{f(x)}{g(x)}$ exists
and is equal to $-1$, which is what we wanted to show.
}

\item $\ds \lim_{x\rightarrow 2} (f(x)g(x))\tsol{=0}$

\sol{Applying the limit law for products (provisionally),
\[
  \lim_{x \to 2} (f(x)g(x)) = \left(\lim_{x \to 2} f(x)\right) \cdot \left(\lim_{x \to 2} g(x) \right)
\]
Now from the graph, $\ds \lim_{x \to 2} g(x) = 0$. It is true that any number times 0 is 0, but we are not
done yet, since we still need to check that $\ds \lim_{x \to 2} f(x)$ exists. Unfortunately, it does not, since
$\ds \lim_{x \to 2^-} f(x) = 1$ while $\ds \lim_{x \to 2^+} f(x) = -1$. This means we cannot use the limit
law equation above, so we will have to try something else.
We fall back to our standard trick of considering two one-sided limits. Applying the limit law for products to the left-hand limit,
\begin{align*}
\lim_{x \to 2^-} (f(x)g(x)) &= \left(\lim_{x \to 2^-} f(x)\right) \cdot \left(\lim_{x \to 2^-} g(x) \right) \\
                                         &= 1 \cdot 0 \\
                                         &= 0.
\end{align*}
Similarly,
\begin{align*}
\lim_{x \to 2^+} (f(x)g(x)) &= \left(\lim_{x \to 2^+} f(x)\right) \cdot \left(\lim_{x \to 2^+} g(x) \right) \\
                                         &= -1 \cdot 0 \\
                                         &= 0.
\end{align*}
Since the two one-sided limits exist and agree, the two-sided limit $\ds \lim_{x \to 2} (f(x)g(x))$ exists and is equal
to 0, which is what we wanted to show.
}

\item $\ds \lim_{x\rightarrow 3^-} f(g(x))\tsol{=2}$

\sol{We need to visualize what happens to $f(g(x))$ when $x$ approaches $3$ from the left. Since we find $f(g(x))$ by plugging $g(x)$ into $f$, a good way to start is to think of how $g(x)$ behaves when $x$
approaches $3$ from the left.\\
From the graph, we notice that $\ds \lim_{x \to 3^-} g(x) = 1$. Moreover, $g(x)$ \emph{increases} to $1$ as $x$ increases to $3$. This means
\[
   \lim_{x \to 3^-} f(g(x)) = \lim_{x \to 1^-} f(x).
\]
We are taking the limit as $x$ approaches 1 from the left since $g(x)$ approaches 1 from the left.
Now, from the graph, $\ds \lim_{x \to 1^-} f(x) = 2$, so $\ds \lim_{x \to 3^-} f(g(x)) = 2$.
}

\ifsolutions\else
\newpage

\begin{center}
\begin{tabular}{l r}
\fgraff & \ggraff
\end{tabular}
\end{center}
\fi

\item $\ds \lim_{x\rightarrow 1^+} f(g(x))\tsol{=2}$

\sol{From the graph, $\ds \lim_{x \to 1^+} g(x) = -1$. Moreover, $g(x)$ \emph{decreases} to $-1$ as $x$ decreases to 1.
This means
\[
   \lim_{x \to 1^+} f(g(x)) = \lim_{x \to -1^+} f(x).
\]
We are approaching $-1$ from the right since $g(x)$ approaches $-1$ from the right. From the graph, $\ds \lim_{x \to -1^+} f(x) = 2$, so $\ds \lim_{x \to 1^+} f(g(x)) = 2$.
}

\item $\ds \lim_{x \rightarrow -2^-} g(f(x))\tsol{=-1\text{ (and NOT -2)}}$

\sol{From the graph, $\ds \lim_{x \to -2^-} f(x) = 1$. Moreover, $f(x)$ is \emph{decreasing} to 1 as $x$ increases to $-1$.
This means
\[
   \lim_{x \to -2^-} g(f(x)) = \lim_{x \to 1^+} g(x).
\]
We are approaching 1 from the right since $f(x)$ approaches 1 from the right.
From the graph, $\ds \lim_{x \to 1^+} g(x) = -1$, so $\ds \lim_{x \to -2^-} g(f(x)) = -1$.
}

\item $\ds \lim_{x\rightarrow 1^-} f(g(x))\tsol{=2\text{ (and NOT 1)}}$

\sol{To get a feel for the problem, we start with a table of values:
\begin{center}
\begin{tabular}{c|c|c}
  $x$ & $g(x)$ & $f(g(x))$ \\ \hline
   0    &  $-2$ & $2$  \\
   0.9 &  $-2$ & $2$ \\
   0.99 & $-2$ & $2$ \\
   0.999 & $-2$ & $2$
\end{tabular}
\end{center}
Notice $g(x)$ is exactly $-2$ whenever $x$ is sufficiently close to $1$ and less than $1$.
This means that $f(g(x))$ is $f(-2)$ for these same $x$-values. Therefore
\[
   \lim_{x \to 1^-} f(g(x)) = f(-2) = 2.
\]
}

\item $\ds \lim_{x\rightarrow 2^-} \frac{f(x)}{g(x)}\tsol{=-\infty}$

\sol{We start by applying the limit law for quotients (provisionally):
\[
  \lim_{x \to 2^-} \frac{f(x)}{g(x)} = \frac{\lim\limits_{x \to 2^-} f(x)}{\lim\limits_{x \to 2^-} g(x)}
\]
However, $\ds \lim_{x \to 2^-} g(x) = 0$, so the limit law does not apply. We are forced to reason about the
limit another way.\\
So we consider the values of the fraction $\dfrac{f(x)}{g(x)}$ when $x$ is near $2$ and less than $2$. Since
$\ds \lim_{x \to 2^-} f(x) = 1$, $f(x)$ should be close to 1 for these $x$ values. On the other hand, $g(x)$ is negative and
approaching 0 for these $x$ values. That is, for $x$ near 2 and less than 2,
\[
   \frac{f(x)}{g(x)} \approx \frac{1}{\text{small negative number}}
\]
which is a large negative number.
Since the small negative number can be made as small as we like by choosing $x$ close enough to 2, the fraction $\dfrac{f(x)}{g(x)}$ can be made as large a negative number as we like. That is, $\ds \lim_{x \to 2^-} \frac{f(x)}{g(x)} = -\infty$.
}

\item $\ds \lim_{x\rightarrow 2^+} \frac{f(x)}{g(x)}\tsol{=-\infty}$.\\
\sol{We consider the values of the fraction $\dfrac{f(x)}{g(x)}$ when $x$ is near 2 and greater than 2. Since
$\ds\lim_{x \to 2^+} f(x) = -1$, $f(x)$ is near $-1$ for these $x$ values. On the other hand, $g(x)$ is positive and approaching
0 for these $x$ values. That is, for $x$ near 2 and greater than 2,
\[
  \frac{f(x)}{g(x)} \approx \frac{-1}{\text{small postive number}}
\]
which is a large negative number. Since the small positive number can be made as small as we like by choosing $x$
close enough to 2, the fraction $\dfrac{f(x)}{g(x)}$ can be made as large a negative number as we like. That is,
$\ds \lim_{x \to 2^+} \frac{f(x)}{g(x)} = -\infty$.
}

\item $\ds \lim_{x\rightarrow 2} \frac{f(x)}{g(x)}\tsol{=-\infty}$

\sol{From the previous two problems, the two one-sided limits both tend to $-\infty$, so the two-sided limit $\ds \lim_{x \to 2} \frac{f(x)}{g(x)}$ also approaches $-\infty$.
}

\end{enumerate}

\end{document}

