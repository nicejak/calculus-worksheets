\documentclass[12pt]{article}
\usepackage{amsmath,amsfonts,amsthm,fullpage,graphicx,fancyhdr,enumerate,color, multicol, latexsym,  hyperref}

\def\Red{\color{red}}
\def\Black{\color{black}}
\setlength{\headheight}{15pt} \setlength{\headsep}{7pt} \setlength{\parindent}{0pt}
\def\Red{\color{red}}
\def\Black{\color{black}}
\pagestyle{fancy} \lhead{Calculus I \chead{\Red SOLUTIONS\Black} \rhead{ Flood Data Project}} \cfoot{}

\begin{document}

The graph below depicts Boulder Creek flow during the September 2013 Colorado flood, as measured by the U.S. Geological Survey (USGS). The graph uses a base-10 logarithmic scale; that is, the horizontal lines above 100 represent 200, 300, 400, 500, 600, 700, 800, 900, 1000, then 2000, 3000, etc.

\begin{center}
\includegraphics[width=6.5in, clip=true, trim=.1in .1in .1in .6in]{BoulderFloodGraph2.pdf}
\end{center}

(The red asterisks represent certain redundant measurements that were used for calibration.  You needn't worry about these.)

\begin{enumerate}[(a)]

\item What was the approximate flow rate at noon on Thursday, September 12, 2013?

\Red About 4,000 cubic feet per second (or just a bit less).\Black
\vfill\eject

\item Using {\it only} your answer to part (a), approximate the quantity of water (in cubic feet) that flowed through Boulder Creek at this station during the day (24-hour period) of Thursday, September 12, 2013.

\Red
Using the above number -- 4,000 cubic feet per second -- as an estimate of the average flow rate over the course of that day, we estimate the total quantity of water flowing through the creek, at this station, on this day, to be
\begin{align*}& \hbox{4,000}\, {\textstyle \frac{\hbox{ft$^3$}}{\hbox{sec}}}\times  24\,  \hbox{hr} \\=& \hbox{4,000}\,  {\textstyle\frac{\hbox{ft$^3$}}{\hbox{sec}}}\times 60\,  {\textstyle\frac{\hbox{sec }}{\hbox{min}}}\times 60\,  {\textstyle\frac{\hbox{min}}{\hbox{hr}}}\times  24\,  \hbox{hr}\\=& \hbox{4,000}\times 60\,   \times 60\, \times  24 \, \hbox{ft$^3$} 
\\=& \hbox{345,600,000}\, \hbox{ft$^3$.} \end{align*}(All units cancel except for \hbox{ft$^3$}.)\Black
\vfill

\item  What is the 27-year median flow rate (the ``median daily statistic'') for September 12? Approximate the median quantity of water that would have flowed through Boulder Creek at this station on the 12th of September, during this 27-year period.

\Red From the graph, we see that the 27-year median flow rate for September 12 is about 70  ft$^3$/sec.  So the median quantity of water through the creek, at this station, on this date, is about
\begin{align*}& 70\,  {\textstyle\frac{\hbox{ft$^3$}}{\hbox{sec}}}\times  24\,  \hbox{hr} \\=& 70\,  {\textstyle\frac{\hbox{ft$^3$}}{\hbox{sec}}}\times 60\,  {\textstyle\frac{\hbox{sec }}{\hbox{min}}}\times 60\,  {\textstyle\frac{\hbox{min}}{\hbox{hr}}}\times  24\,  \hbox{hr}
\\=& 70\times 60\,   \times 60\, \times  24 \, \hbox{ft$^3$} 
\\=& \hbox{6,048,000}\, \hbox{ft$^3$.} \end{align*} \Black

\vfill
\end{enumerate}
\newpage

\begin{center}
\includegraphics[width=6.5in, clip=true, trim=.1in .2in .1in .6in]{BoulderFloodRiemannGraph2.pdf}
\end{center}

\begin{enumerate}[(a)]\addtocounter{enumi}{3}
\item Take $\Delta t = 12$ hours, and use a left-endpoint Riemann sum to approximate the total quantity of water that flowed through Boulder Creek at this station from the beginning of Wednesday, September 11 through the end of Tuesday, September 17. {\it Draw your boxes directly on top of the graph above}   (and write your calculations below).

\Red Because the vertical scale is logarithmic, precise interpolation is a bit tricky.  For example:  a rectangle  whose top is halfway  (in terms of actual distance)  between grid lines does {\it not} correspond to a flow rate that's halfway between the rates represented by those grid lines. See the discussion a bit further below.


For the present purposes, we'll interpolate somewhat crudely:  if the top of a rectangle is ``near'' a grid line, we'll take the height of the rectangle to be that of the grid line; if the top of a rectangle is ``far from'' a grid line, we'll take the height of the rectangle to be the simple {\it average} of those of the two nearest grid lines.  Using this rubric, we estimate (quite roughly)  that the total flow  over the week in question is
\begin{align*}&\hbox{the sum of areas of the rectangles, in cubic feet}\\=&
\hbox{(baselength, in sec)} \times \Bigl(\hbox{sum of heights, in}\  {\textstyle \frac{\hbox{ft$^3$}}{\hbox{sec}}}\Bigr)\\=& 60\,  {\textstyle\frac{\hbox{sec }}{\hbox{min}}}\times 60\,  {\textstyle\frac{\hbox{min}}{\hbox{hr}}}\times12 \, \hbox{hr}\times  \biggl(90 + 250 + 1500 + 3500\\+&  6000 + 4500 + 2000 + 1500 + 1500 + 1500 + \
2000 + 1500 + 1500 + 1500\biggr) \, {\textstyle\frac{\hbox{ft$^3$}}{\hbox{sec}}}
\\=& \hbox{43,200} \, \hbox{sec} \times \hbox{28,340}\, {\textstyle\frac{\hbox{ft$^3$}}{\hbox{sec}}}  =\hbox{1,245,888,000}\, \hbox{ft$^3$}.\end{align*} 


To obtain a more accurate interpolation, we'd have to observe the following:  Moving upwards one $n$th of the way (in terms of actual distance) from a flow rate $y$ to a flow rate $z$ means an  increase in  flow rate by a factor equal to the $n$th root of  $z/y$.  This follows from properties of exponents  -- specifically, it follows from the fact that, if $z=10^b$ and $y=10^a$, then $ \root{n}\of{z/y}=\root{n}\of{10^b/10^a}=10^{(b-a)/n}$ -- or, equivalently, from properties of logarithms.

 
For example:  the ninth rectangle above achieves a height roughly halfway between the 1000 ft$^3$/sec gridline and the 2000 ft$^3$/sec gridline.  Consequently,  that rectangle represents a flow rate of about  $\root{2}\of {2000/1000}=\root2\of2\approx 1.41$ times greater than that of the lower grid line;  that is, the rectangle represents a flow rate of about  $1000\times 1.41=1,410$ ft$^3$/sec.  


Using such ideas would give us an estimated flow rate closer to $\hbox{1,224,288,000}\, \hbox{ft$^3$}$.  But note that  this number differs from our previous, ``cruder'' estimate by just over 1\%, which is negligible compared to other inaccuracies inherent in our approximation procedure.  \Black
\end{enumerate}\vfill\eject
 


\begin{center}
\includegraphics[width=6.5in, clip=true, trim=.1in .1in .1in 1in]{BoulderFloodRiemannGraph3.pdf}
\end{center}
\begin{enumerate}[(a)]\addtocounter{enumi}{4}
\item Use the median daily statistic data, and a Riemann sum with $\Delta t = 24$ hours, to approximate the median quantity of water that would have flowed through the creek over the course of these same dates, during the 27-year period considered. {\it Draw your boxes directly on top of the graph above} (and write your calculations below).

\Red Arguing as in part (d) above, we find the median flow to be about

\begin{align*} & 60\,  {\textstyle\frac{\hbox{sec }}{\hbox{min}}}\times 60\, {\textstyle\frac{\hbox{min}}{\hbox{hr}}}\times24\, \hbox{hr}\\\times &(70+70 +63+70+63+54+52) \,{\textstyle\frac{\hbox{ft$^3$}}{\hbox{sec}}}
\\=& \hbox{86,400}\, \hbox{sec} \times 442\,{\textstyle\frac{\hbox{ft$^3$}}{\hbox{sec}}}  =\hbox{38,188,800}\, \hbox{ft$^3$}.\end{align*} 

\vfill\Black

For more flow data, see \url{http://waterdata.usgs.gov/nwis/uv?06730200} and \\ \url{http://www.dwr.state.co.us/SurfaceWater/data/district.aspx?div=1&dist=6}

\end{enumerate}

\end{document}
