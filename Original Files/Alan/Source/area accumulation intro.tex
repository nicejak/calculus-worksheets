\documentclass[letterpaper,11pt]{article}
\usepackage[margin=.5in]{geometry}
\usepackage{calc}

\usepackage{amsmath, mathtools, comment, graphicx, fancyhdr, color, setspace, multicol, enumerate, hyperref}
\usepackage{tikz,tikz-3dplot,pgfplots}

\usepackage{amssymb,latexsym,epsfig,amscd,arydshln}
			
\parskip = 0.2in

\def\ds{\displaystyle}

\setlength{\parindent}{0pt}

\newcounter{saveenum}

\newif\ifsolutions
\solutionsfalse
\ifsolutions
    \newcommand{\opsol}{1}
    \newcommand{\usol}[1]{\underline{\textcolor{red}{#1} \hspace{.2 in}}}
    \newcommand{\sol}[1]{\textcolor{red}{#1}}
\else
    \newcommand{\opsol}{0}
    \newcommand{\usol}[1]{\underline{\textcolor{white}{#1} \hspace{.2 in}}}
    \newcommand{\sol}[1]{\textcolor{white}{#1}}
\fi



\begin{document}
\pagenumbering{gobble}


{\large \textbf{Area accumulation functions -- an introduction}} \\
Given a function $f(x)$, we create a new function $F(x)$ by evaluating how much area is accumulated under $f(x)$.

\begin{enumerate}
\item Example:

\vspace{-.2 in}
\begin{center}
\begin{tikzpicture}
\begin{axis}[
   	xmin=-2.2, xmax=7.2,
	ymin=-1.2, ymax=2.2,
	major tick length={0},
	line width=1pt,
 	axis lines=center, height=3 in, width=6 in, grid=major, ylabel={$f$},
 	restrict y to domain=-1.2:3,
 	axis equal
	]
	\addplot[<-] [very thick, samples=99, domain=-2:1] {1};
	\addplot[->] [very thick, samples=99, domain=6:7] {-1};
	\addplot [black, very thick] plot coordinates {(1,1)(2,2)};
	\addplot [black, very thick] plot coordinates {(2,2)(3,2)};
	\addplot [black, very thick] plot coordinates {(3,2)(6,-1)};
\end{axis}
\end{tikzpicture}
\end{center}

\begin{enumerate}

\vspace{-.1 in}
\item Define $\ds F(x)=\int_0^x f(t) \,dt$. Evaluate the following:

\begin{multicols}{2}
$F(0)= \sol{0}$  \\
$F(1)= \sol{1} $ \\

$F(2)= \sol{2.5} $  \\
$F(-1)= \sol{-1}$  \\
\end{multicols}

\vspace{-.2 in}
\item Shade in and find the area represented by $F(3)-F(1)$.

\sol{It is the area under the curve between $x=1$ and $x=3$, an area of 3.5.}
\vfill

\item Find a formula for $F(x)$ between $x=0$ and $x=1$

\sol{$x$}
\vfill

\item Give two values at which $F(x)=0$. (Hint: assume the graph continues to the right.)

\sol{$x=0$ and $x=12$}
\vfill

\item Which is larger: $F(3)$ or $F(4)$? Explain.

\sol{$F(4)$ is larger because $F(x)$ accumulates area and all the area between $x=3$ and $x=4$ counts as positive since $f(x)$ is positive there.}
\vfill

\item Which is larger: $F(5)$ or $F(6)$? Explain.

\sol{$F(5)$ is larger because the area accumulated between $x=5$ and $x=6$ counts negatively, since $f(x)$ is negative there.}
\vfill

\item Give open intervals on which $F(x)$ is increasing. Explain.

\sol{$F(x)$ is increasing from $x=0$ to $x=5$. On this interval the area is all positive, so as it accumulates, the value of $F(x)$ must increase.  To the left of $x=0$, $F(x)$ is also increasing because the value of the integrals becomes less negative as the value of $x$ moves to the right.}
\vfill

\item $F(x)$ has a local extremum at $x=5$. Is it a maximum or a minimum? Explain.

\sol{It is a local maximum because $F(x)$ is increasing up to $x=5$ and decreasing after.}
\vfill

\item $F(x)$ is increasing at both $x=1$ and $x=2$. At which value is $F(x)$ increasing faster? Explain.

\sol{$F(x)$ is increasing faster at $F(2)$ because the value of $f(x)$ is larger, so as $x$ grows, the area is accumulating at a faster rate.}
\vfill

\end{enumerate}

\newpage

\item $g$ is a piecewise function composed of line segments and a semi-circle.

\begin{center}
\begin{tikzpicture}
\begin{axis}[
   	xmin=-2.5, xmax=12.5,
	ymin=-4.2, ymax=4.2,
	major tick length={0},
	line width=1pt,
 	axis lines=center, height=3 in, width=4 in, grid=major, ylabel={$g$},
 	restrict y to domain=-4.2:4.2,
 	axis equal
	]
	\addplot[<-] [very thick, samples=99, domain=-2:0] {-x};
	\addplot [smooth, very thick, samples=999, domain=0:4] {-(4-(x-2)^2)^.5};
	\addplot [black, smooth, very thick] plot coordinates {(4,0) (8,4)};
	\addplot[->] [black, very thick, domain=8:12] {-2*x+20};
\end{axis}
\end{tikzpicture}
\end{center}

\begin{enumerate}
\item $\ds G(x)=\int_4^x g(t) \,dt$

\begin{multicols}{2}
$G(4)= \sol{0}$ \\
$G(10)=\sol{12}$ \\

$G(12)=\sol{8}$ \\
$G(0)=\sol{2\pi}$ \\
\end{multicols}

\vspace{-.2 in}
\item On what open intervals is $G(x)$ increasing? decreasing?

\sol{$G(x)$ is decreasing on $(0,4) \cup (10,\infty)$. $G(x)$ is increasing on $(-\infty, 0) \cup (4,10)$.}
\vfill

\item Find all local extreme values of $G(x)$ by determining where $G(x)$ switches from increasing to decreasing and from decreasing to increasing.

\sol{$G(x)$ has local maximum values at $x=0$ and $x = 10$ (the area accumulation function switches from increasing to decreasing there), and the local max values are $G(0)=2 \pi$ and $G(10)=12$.
The area accumulation function switches from decreasing to increasing at $x=4$, so $G$ has a local minimum there.
The local minimum is $G(4)=0$.
}
\vfill

\item What are the critical numbers of $G(x)$?

\sol{$G(x)$ has critical numbers at $x=0, 4, 10$.}
\vfill

\item On the interval from $[0,12]$ where is $G(x)$ increasing fastest?

\sol{At $x=8$.}
\vfill

\end{enumerate}
\newpage

\item The peak of Boulder's epic rainstorm of 2013 occured between 4pm, Sept 12, and 4am, Sept 13. During those 12 hours the rate of rainfall can be modelled by $\ds r(t)=\frac{240e^{2t/3}}{\left( 60+e^{2t/3} \right)^2}$ in inches per hour, where $t=0$ represents 4pm on Sept 12.

\begin{center}
\begin{tikzpicture}
\begin{axis}[
   	xmin=-.2, xmax=12.2,
	ymin=-.2, ymax=1.2,
	major tick length={0},
	line width=1pt,
 	axis lines=center, height=3 in, width=4 in, grid=major, ytick={0,.25,...,1}, xtick={0,1,...,12},
	]
	\addplot [smooth, very thick, samples=99, domain=0:12] {(240*e^(2*x/3))*(60+e^(2*x/3))^(-2)};
\end{axis}
\end{tikzpicture}
\end{center}

Let $\ds R(x)=\int_0^x \frac{240e^{2t/3}}{\left( 60+e^{2t/3} \right)^2} \,dt$.

\begin{enumerate}
\item Use the graph to estimate $R(4)$. What does it represent? (include units)

\sol{At $R(4) \approx 1$. This represents the total number of inches of rain that have fallen between 4pm, Sept 12, and 4 hours later, at 8pm, Sept 12.}
\vfill

\item Use technology to calculate $R(12)$. What does it represent? (include units)

\sol{At $R(12) \approx 5.78326$. This represents the total number of inches of rain that have fallen between between 4pm, Sept 12, and 12 hours later, at 4am, Sept 13.}
\vfill

\item What does $R(x)$ represent?

\sol{The amount of rain that has fallen between 4pm, Sept 12 and $x$ hours later.}
\vfill

\item Where is $R(x)$ changing the fastest?

\sol{At $x=6$.}
\vfill

\end{enumerate}
\end{enumerate}
\end{document}
