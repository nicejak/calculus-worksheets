
\documentclass[letterpaper,11pt]{article}
\usepackage{amsmath}
\usepackage[letterpaper,margin=1in,includehead=true]{geometry}
\usepackage{comment}
\usepackage{graphicx}
\usepackage{fancyhdr}
\pagestyle{fancy}
\usepackage{color}
\usepackage{setspace}
\usepackage{tikz,tikz-3dplot,pgfplots}
\usepackage{comment}
\usepackage{multicol}
\usepackage{array}

\setlength{\headheight}{15pt}%needed to remove fancyhdr error--not crucial%

\newcounter{mycounter}  
\newenvironment{noindlist}
 {\begin{list}{\arabic{mycounter}.~~}{\usecounter{mycounter} \labelsep=0em \labelwidth=0em \leftmargin=0em \itemindent=0em}}
 {\end{list}}

\newcommand{\unenumerate}[1]{\setcounter{saveenum}{\value{enumi}}\end{enumerate}
	\noindent #1 
	\begin{enumerate} \setcounter{enumi}{\value{saveenum}}}

\newcounter{saveenum}

\def\ds{\displaystyle}

%To print solutions, use \solutionstrue; To hide solutions, use \solutionsfalse.
%\sol takes two arguments. #1 is the vertical length. #2 is the text.

\newif\ifsolutions
\solutionsfalse
\ifsolutions
    \newcommand{\sol}[2]{\begin{minipage}[c][#1]{\linewidth}{\textcolor{blue}{\textbf{Solution:}}\quad \textcolor{blue}{#2}}\end{minipage}}
    \newcommand{\opsol}[1]{#1}
    \newcommand{\tblsol}[1]{\textcolor{blue}{#1}}
\else
    \newcommand{\sol}[2]{\begin{minipage}[c][#1]{\linewidth}{\vfill}\end{minipage}}
    \newcommand{\opsol}[1]{0}
    \newcommand{\tblsol}[1]{\textcolor{white}{#1}}
\fi

\newcommand{\ddx}{\dfrac{d}{dx}}
\newcommand{\dydx}{\dfrac{dy}{dx}}

\begin{document}
\lhead{Math 1300: Calculus I}
\rhead{\bf Graphing functions using analysis} 



\noindent Goal:  To collect information about the first and second derivatives of a function, then use this information to
graph the function without using technology.
 
\begin{enumerate} 
\item Consider the function 
$ f(x) = 3x^4-8x^3+6x^2$.
\begin{enumerate}
\item Determine the open intervals on which the function is increasing/decreasing.

\sol{1.6 in}{$f'(x)=12x^3-24x^2+12x = 12x(x-1)^2$. Solving $0=12x^3-24x^2+12x$ gives us $x=0$ and $x=1$. The derivative can only change signs at these two points.  To determine the sign within each interval, we substitute a value within each interval. Substituting $x=-1$ into the derivative gives $f'(-1) = -48 < 0$, so $f$ is decreasing on $(-\infty, 0)$.  Now $f'(\frac{1}{2}) = \frac{3}{2} > 0$, so $f$ is increasing on $(0,1)$.  Finally, $f'(2) = 24 > 0$, so $f$ is also increasing on $(1, \infty)$. (In fact, by the definition of ``increasing on an interval'', the last two intervals can be joined and $f(x)$ is actually increasing on the entire interval $(0,\infty)$.)}

\item Find the local maxima and local minima of $f(x)$, if any.  Be sure to find the critical points, classify them using either the first or second derivative test, then substitute the $x$-values into $f(x)$ to find the local mininum/maximum values.

\sol{1.9in}{From the previous part, the critical points are at $x=0$ and $x=1$. $f''(x)=36x^2-48x+12$, so $f''(0)=12$ and $f''(1)=0$. By the second derivative test, we can say there is a local minimum at $x=0$. Since $f''(1)=0$ we must use the first derivative test. From the previous part, the function is increasing on both sides of $x=1$, and therefore this is neither a local min nor a local max there. \\\\
Evaluating the function at our critical values gives $f(0)=0$ (the local minimum of $f$) and $f(1)=1$ (a stationary point of $f$).}

\item Find the inflection points of the function, if any.  Be sure to find where the second derivative is zero, use a sign chart to determine whether or not the second derivative changes, then substitute the $x$-values into $f(x)$ to find the $y$-value at each inflection point.

\sol{2.7 in}{$f''(x)=36x^2-48x+12 = 12(3x-1)(x-1)$. Solving $0=12(3x-1)(x-1)$ gives us $x=1/3$ and $x=1$. \\\\
Testing on either side of $x=1/3$. Left: $f''(0)=12$. Right: $f''(1/2)=-3$. The signs change so this is an inflection point.\\\\
Testing on either side of $x=1$. Left: $f''(1/2)=-3$. Right: $f''(2)=60$. The signs change so this is an inflection point.\\\\
Evaluating the function at these $x$ values gives $f(1/3)=11/27$ and $f(1)=1$.  So the points $(0,0)$ and $(1/3, 11/27)$ are inflection points.}

\newpage

\item Plot the local extrema and the inflection points on the graph.  Transfer the information from parts (a) and (b) to the number lines for $f'(x)$ and $f''(x)$. Sketch the graph of the function $f(x)=3x^4-8x^3+6x^2$, using all of the information.\\

\begin{center}
\begin{tikzpicture}
\begin{axis}[
   	xmin=-1.5, xmax=2.5,
	ymin=-0.5, ymax=3,
	major tick length={0},
	line width=1pt,
 	axis lines=center, height=5in, width=5in, grid=major,
 	after end axis/.code={
            \draw[<->, very thick, black,] (axis cs:-1.5,-.9) -- (axis cs:2.5,-.9);
            \node at (axis cs:-2,-.9) {$f'(x)$:};
            \draw[<->, very thick, black] (axis cs:-1.5,-1.25) -- (axis cs:2.5,-1.25);
            \node at (axis cs:-2, -1.25) {$f''(x)$:};
            \draw[<-, red,very thick, opacity=\opsol{1}] (axis cs:-1.5,-.9) -- (axis cs:0,-.9);
            \draw[->,black,very thick, opacity=\opsol{1}] (axis cs:0,-.9) -- (axis cs:2.5,-.9);
            \node[opacity=\opsol{1}] at (axis cs:-.75, -.75) {$-$};
            \node[opacity=\opsol{1}] at (axis cs:.5, -.75) {$+$};
            \node[opacity=\opsol{1}] at (axis cs:1.75, -.75) {$+$};
            \fill[black, opacity=\opsol{1}] (axis cs:0,-.9) circle (3 pt);
            \fill[black, opacity=\opsol{1}] (axis cs:1,-.9) circle (3 pt);
            \draw[<-, black,very thick, opacity=\opsol{1}] (axis cs: -1.5,-1.25) -- (axis cs:.33,-1.25);
            \draw[red,very thick, opacity=\opsol{1}] (axis cs: .33,-1.25) -- (axis cs:1,-1.25);
            \draw[->,black,very thick, opacity=\opsol{1}] (axis cs: 1,-1.25) -- (axis cs:2.5,-1.25);
            \node[opacity=\opsol{1}] at (axis cs:-.59, -1.1) {$+$};
            \node[opacity=\opsol{1}] at (axis cs:.67, -1.1) {$-$};
            \node[opacity=\opsol{1}] at (axis cs:1.75, -1.1) {$+$};
            \fill[black, opacity=\opsol{1}] (axis cs:.33,-1.25) circle (3 pt);
            \fill[black, opacity=\opsol{1}] (axis cs:1,-1.25) circle (3 pt);
             }]
    \addplot [opacity=\opsol{1}, blue, smooth, ultra thick, samples=100, domain=-2:2.5] {3*x^4-8*x^3+6*x^2};
\end{axis}
\end{tikzpicture}
\end{center}

\item Now use your graphing calculator to get the graph of $y=f(x)$ on this domain, and compare it to the graph you just drew. How well did you do?

\sol{1in}{Great!  They look just the same!}
\end{enumerate}
\newpage
\item  Using the same process as in the previous problem, graph $\displaystyle f(x) = x^{\frac{1}{3}}(x+4)$ on the next page.

\sol{7in}{\[f'(x)=x^{\frac{1}{3}}+\frac{1}{3}x^{-\frac{2}{3}}(x+4)=x^{\frac{1}{3}}+\frac{x+4}{3x^{\frac{2}{3}}}\]
Combine these terms by finding common denominators:
\[f'(x)=\frac{3x}{3x^{\frac{2}{3}}} +\frac{x+4}{3x^{\frac{2}{3}}} = \frac{4x+4}{3x^{\frac{2}{3}}} = \frac{4(x+1)}{3x^{\frac{2}{3}}}\]
There is a critical point at $x=-1$ (horizontal tangent line) and a critical point at $x=0$ (vertical tangent line, possibly a cusp).
Substituting points $x=-2$, $x=-\frac{1}{2}$ and $x=1$ into the derivative, we see that $f(x)$ is decreasing on $(-\infty, -1)$ and increasing on $(-1,0)$ as well as $(0, \infty)$. (In fact, by the definition of ``increasing on an interval'', the last two intervals can be joined and $f(x)$ is thus increasing on the interval $(-1, \infty)$.)\\
By the first derivative test, since $f'(x)$ changes signs from negative to positive at $x=-1$, $f$ has a local minimum of $f(-1) = -3$ at $x=-1$.  Since $f'(x)$ does not change sign at $x=0$, $f$ has no local extremum at $x=0$. The graph has a vertical tangent line there and no cusp.\\\\
Now on to the second derivative.  We need the quotient rule.
\[f''(x) = \frac{(3x^{\frac{2}{3}})\cdot 4 - (4x+4)\cdot 2x^{-\frac{1}{3}}}{(3x^{\frac{2}{3}})^2} = \frac{12x^{\frac{2}{3}} - (8x+8)\cdot x^{-\frac{1}{3}}}{9x^{\frac{4}{3}}}\]
Now get rid of negative exponents by multiplying numerator and denominator by $x^{\frac{1}{3}}$:
\[f''(x)=\frac{12x^{\frac{2}{3}} - (8x+8)\cdot x^{-\frac{1}{3}}}{9x^{\frac{4}{3}}} \cdot\frac{x^{\frac{1}{3}}}{x^{\frac{1}{3}}}=\frac{12x-(8x+8)}{9x^{\frac{5}{3}}} = \frac{4x-8}{9x^{\frac{5}{3}}}=\frac{4(x-2)}{9x^{\frac{5}{3}}}\]
The second derivative is 0 at $x=2$ and undefined at $x=0$.  Substituting $x=-1$, $x=1$, and $x=4$ we see that $f(x)$ is concave up on $(-\infty, 0)\cup(2,\infty)$ and concave down on $(0,2)$, and so there are inflection points at $x=0$ and $x=2$. }

\newpage

\begin{center}
Graph of $\displaystyle f(x) = x^{\frac{1}{3}}(x+4)$

\vspace{.1in}

\begin{tikzpicture}
\begin{axis}[
   	xmin=-3, xmax=3,
	ymin=-3.5, ymax=11,
	xtick={-3,..., 3},  
	ytick={-3,-1,...,11},
	major tick length={0},
	line width=1pt,
 	axis lines=center, height=7in, width=5in, grid=major,
 	after end axis/.code={
            \draw[<->, very thick, black,] (axis cs:-3,-4.5) -- (axis cs:3,-4.5);
            \node at (axis cs:-3.5,-4.5) {$f'(x)$:};
            \draw[<->, very thick, black] (axis cs:-3,-6) -- (axis cs:3,-6);
            \node at (axis cs:-3.5, -6) {$f''(x)$:};
            \draw[<-, red,very thick, opacity=\opsol{1}] (axis cs:-3,-4.5) -- (axis cs:-1,-4.5);
            \draw[->,black,very thick, opacity=\opsol{1}] (axis cs:-1,-4.5) -- (axis cs:3,-4.5);
            \node[opacity=\opsol{1}] at (axis cs:-2, -4) {$-$};
            \node[opacity=\opsol{1}] at (axis cs:-.5, -4) {$+$};
            \node[opacity=\opsol{1}] at (axis cs:1.5, -4) {$+$};
            \fill[black, opacity=\opsol{1}] (axis cs:-1,-4.5) circle (3 pt);
            \draw[black, opacity=\opsol{1}] (axis cs:0,-4.5) circle (3 pt);
            \draw[<-, black,very thick, opacity=\opsol{1}] (axis cs: -3,-6) -- (axis cs:0,-6);
            \draw[red,very thick, opacity=\opsol{1}] (axis cs: 0,-6) -- (axis cs:2,-6);
            \draw[->,black,very thick, opacity=\opsol{1}] (axis cs: 2,-6) -- (axis cs:3,-6);
            \node[opacity=\opsol{1}] at (axis cs:-1.5, -5.5) {$+$};
            \node[opacity=\opsol{1}] at (axis cs:1, -5.5) {$-$};
            \node[opacity=\opsol{1}] at (axis cs:2.5, -5.5) {$+$};
            \draw[black, opacity=\opsol{1}] (axis cs:0,-6) circle (3 pt);
            \fill[black, opacity=\opsol{1}] (axis cs:2,-6) circle (3 pt);
             }]
     \addplot [opacity=\opsol{1}, blue, smooth, ultra thick, samples=99, domain=-3:3] {x/abs(x)*abs(x)^(1/3)*(x+4)};
\end{axis}
\end{tikzpicture}
\end{center}

\end{enumerate}
\end{document}
