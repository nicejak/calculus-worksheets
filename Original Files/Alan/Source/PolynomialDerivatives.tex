\documentclass[letterpaper,11pt]{article}
\usepackage{amsmath}
\usepackage[letterpaper,margin=1in,includehead=true]{geometry}
\usepackage{comment}
\usepackage{graphicx}
\usepackage{fancyhdr}
\pagestyle{fancy}
\usepackage{color}
\usepackage{setspace}
\usepackage{tikz,tikz-3dplot,pgfplots}

\newcounter{mycounter}  
\newenvironment{noindlist}
 {\begin{list}{\arabic{mycounter}.~~}{\usecounter{mycounter} \labelsep=0em \labelwidth=0em \leftmargin=0em \itemindent=0em}}
 {\end{list}}

%To print solutions, use \solutionstrue
%To repress solutions, use \solutionsfalse
%\sol takes two arguments. #1 is the vertical length. #2 is the text.

\newif\ifsolutions
\solutionsfalse
\ifsolutions
    \newcommand{\sol}[2]{\begin{minipage}[c][#1]{\linewidth}{\textcolor{blue}{\textbf{Solution:}}\quad \textcolor{blue}{#2}}\end{minipage}}
    \newcommand{\fsol}[2]{\includegraphics[scale=#1]{#2sol}}
    \newcommand{\tsol}[2]{\textcolor{blue}{\textbf{Solution:}}\quad\vspace{-.25 in}#2}
\else
    \newcommand{\sol}[2]{\begin{minipage}[c][#1]{\linewidth}{\vfill}\end{minipage}}
    \newcommand{\fsol}[2]{\includegraphics[scale=#1]{#2}}
    \newcommand{\tsol}[2]{#1}
\fi

\newcommand{\unenumerate}[1]{\setcounter{saveenum}{\value{enumi}}\end{enumerate}
	\noindent #1 
	\begin{enumerate} \setcounter{enumi}{\value{saveenum}}}

\newcounter{saveenum}

\def\ds{\displaystyle}

\begin{document}
\lhead{Math 1300: Calculus I}
\chead{}
\rhead{\bf Polynomial Derivatives} 

\begin{enumerate}
\item Use technology to graph  
$g(x) = 3x^4-16x^3+24x^2+6$.


\tsol{
\begin{center}
\begin{tikzpicture}
\begin{axis}[
    axis lines=center, height=2.3in, grid=major,
    	ymin=0, ymax=45,xmin=-1,xmax=3
	]
    \addplot [opacity=0, blue, smooth, very thick, samples=1000, domain=-1:3] {3*x^4-16*x^3+24*x^2+6};
\end{axis}
\end{tikzpicture}
\end{center}}
{
\begin{center}
\begin{tikzpicture}
\begin{axis}[
    axis lines=center, height=2.3in, grid=major,
      	ymin=0, ymax=45,xmin=-1,xmax=3
	]
    \addplot [blue, smooth, very thick, samples=1000, domain=-1:3] {3*x^4-16*x^3+24*x^2+6};
\end{axis}
\end{tikzpicture}
\end{center}}

\begin{enumerate}
\item  Looking at the graph, where does it appear that $g(x)$ has relative minima and relative maxima (valleys and peaks)?

\sol{.5 in}{There appears to be a low point (relative minimum) at $x=0$.}

\item  Looking at the graph, where does it appear that $g(x)$ has inflection points?

\sol{.5 in}{g(x) appears to have inflection points at around $x=0.5$ and $x=2$.}

\item  Calculate $g'(x)$ and find where it is zero using algebra.

\sol{1.9 in}{
\begin{align*}
g'(x)&=12x^3-48x^2+48x \\
0&=12x^3-48x^2+48x \\
0&=12x(x^2-4x+4) \\
0&=12x(x-2)(x-2) \\
\text{ So } x&=0 \text{ or } x=2 \\
\end{align*}
}

\item  Use technology to graph $g'(x)$, and check that you correctly found its zeroes.~
\tsol{
\begin{center}
\begin{tikzpicture}
\begin{axis}[
    axis lines=center, height=2.5in, grid=major,ymin=-10,ymax=20,xmin=-1,xmax=3]
	]
    \addplot [opacity=0, blue, smooth, very thick, samples=1000, domain=-1:3] {12*x^3-48*x^2+48*x};
\end{axis}
\end{tikzpicture}
\end{center}}
{
\begin{center}
\begin{tikzpicture}
\begin{axis}[
    axis lines=center, height=2.5in, grid=major,ymin=-10,ymax=20,xmin=-1,xmax=3
	]
    \addplot [blue, smooth, very thick, samples=1000, domain=-1:3] {12*x^3-48*x^2+48*x};
    \addplot [red, only marks, mark=square, very thick] coordinates{(0,0) (2,0)};
\end{axis}
\end{tikzpicture}
\end{center}
}
\newpage
\item  Now interpret the graph of $g'(x)$, explaining how it can be used to determine where $g(x)$ (the original function) has its relative minima and maxima.

\sol{1.5 in}{
Left of 0, $g'(x)<0$, and right of 0, $g'(x)>0$, therefore $g(x)$ has a relative minimum at $x=0$. \\\\
Left of 2, $g'(x)>0$, right of 2, $g'(x)>0$, therefore $g(x)$ does not have a relative low or high at $x=2$.}
\item  Calculate $g''(x)$ and find its zeroes.

\sol{1.5 in}{\vspace{-.2 in}
\begin{align*} g''(x)&=36x^2-96x+48 \\
0&=36x^2-96x+48=12(3x^2-8x+4) \\
0&=12(x-2)(3x-2), \text{ so } x=2, \frac{2}{3}
\end{align*}
}
\item  Use technology to graph $g''(x)$, and check that you correctly found its zeroes.

\tsol{
\begin{center}
\begin{tikzpicture}
\begin{axis}[
    axis lines=center, height=2in, grid=major,ymin=-30,ymax=60
	]
    \addplot [opacity=0, blue, smooth, very thick, samples=1000, domain=-1:3] {36*x^2-96*x+48};
\end{axis}
\end{tikzpicture}
\end{center}}
{
\begin{center}
\begin{tikzpicture}
\begin{axis}[
    axis lines=center, height=2in, grid=major,ymin=-30,ymax=60
	]
    \addplot [blue, smooth, very thick, samples=1000, domain=-1:3] {36*x^2-96*x+48};
    \addplot [red, only marks, mark=square, very thick] coordinates{(.667,0) (2,0)};
\end{axis}
\end{tikzpicture}
\end{center}}

\item  Explain how you can use the graph of $g''(x)$ to determine the exact location of the inflection points of $g(x)$.

\sol{.8 in}{The sign of $g''(x)$ changes at the marked points.  The zeroes of $g''(x)$ are close to where we guessed the inflection of points of $g(x)$ would be.} 

\item  Explain how you can use the graph of $g'(x)$ to determine the exact location of the inflection points of $g(x)$.

\sol{1.3 in}{At $x= \frac{2}{3}$ $g'(x)$ has a local maximum.  This means that $g'(x)$ switches from increasing to decreasing there.  So the derivative of $g'(x)$, namely $g''(x)$, must switch from positive to negative there.  This means that $g(x)$ switches from concave up to concave down at $x=\frac{2}{3}$, and thus has an inflection point.  A similar argument can be made at $x=2$, where $g'(x)$ has a local minimum.  Bottom line: where the derivative has a local extreme point, the original function must have an inflection point}
\end{enumerate}
\newpage
\item  Now consider the function $f(x) = 3x^4-16x^3+23x^2+6$.  Notice how similar its formula is to the function is in the previous problem.
\begin{enumerate}
\item  Compare its graph (using technology) to the graph of the previous function.

\tsol{
\begin{center}
\begin{tikzpicture}
\begin{axis}[
    axis lines=center, height=2.0in, grid=major,
    	ymin=0, ymax=45,xmin=-1,xmax=3,
    	]
    \addplot [opacity=0, blue, smooth, very thick, samples=1000, domain=-1:3] {3*x^4-16*x^3+23*x^2+6};
\end{axis}
\end{tikzpicture}
\end{center}}
{
\begin{center}
\begin{tikzpicture}
\begin{axis}[
     axis lines=center, height=2.0in, grid=major,xmin=-1,xmax=3,
     	ymin=0, ymax=45
	]
    \addplot [blue, smooth, very thick, samples=1000, domain=-1:3] {3*x^4-16*x^3+23*x^2+6};
\end{axis}
\end{tikzpicture}
\end{center}}

\item Looking at the graph, where does it appear that $f(x)$ has relative minima and relative maxima?

\sol{.5 in}{Relative max at $x=1.5$ and relative min at $x=0$ and $x=2.5$.}

\item Calculate $f'(x)$ and use technology to graph it.

\tsol{
\begin{center}
\hspace{2 in}
\begin{tikzpicture}
\begin{axis}[
    axis lines=center, height=2in, grid=major, ymin=-30, ymax=30,xmin=-1,xmax=3
	]
    \addplot [opacity=0, blue, smooth, very thick, samples=1000, domain=-1:3] {12*x^3-48*x^2+46*x};
\end{axis}
\end{tikzpicture}
\end{center}}
{\textcolor{blue}{$f'(x)=12x^3-48x^2+46x$}
\begin{center}
\hspace{2 in}
\begin{tikzpicture}
\begin{axis}[
    axis lines=center, height=2in, grid=major, ymin=-30, ymax=30,xmin=-1,xmax=3
	]
    \addplot [blue, smooth, very thick, samples=1000, domain=-1:3] {12*x^3-48*x^2+46*x};
    \addplot [red, only marks, mark=square, very thick] coordinates{(0,0) (1.6,0) (2.41,0)};
\end{axis}
\end{tikzpicture}
\end{center}}

\item Use the graph of $f'(x)$ to answer the question of where $f(x)$ has its relative maxima and minima.  Explain.

\sol{1.1 in}{
Left of 0, $f'(x)<0$, and right of 0, $f'(x)>0$, therefore $f(x)$ has a relative minimum at $x=0$. \\
Left of 1.5, $f'(x)>0$, and right of 1.5, $f'(x)<0$, therefore $f(x)$ has relative maximum at $x=1.5$. \\
Left of 2.5, $f'(x)<0$, and right of 2.5, $f'(x)>0$, therefore $f(x)$ has a relative minimum at $x=2.5$.}

\item What are the key differences between the graphs of $g(x)$ (from problem 1) and $f(x)$ (from this problem)?  How are these differences reflected in the graphs of $f'(x)$ and $g'(x)$?

\sol{1.2in}{$g(x)$ has only one local extreme point, and one stationary point.  But in the graph of $f(x)$, that stationary point looks like it's been turned a little bit, so there's a now a local maximum and local minimum.  There's a dip in the function, where water could pool if this was a roof.  You can see this in the derivatives: $g'(x)$ has a zero ($x$-intercept) where the graph just touches the axis but does not change sign. The graph of $f'(x)$, however, dips slightly below the $x$-axis.  $f(x)$ has a local extreme point at each of the places where the derivative crosses.}
\end{enumerate}




\end{enumerate}
\end{document}
