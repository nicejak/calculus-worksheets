\documentclass[letterpaper,11pt]{article}
\usepackage{amsmath, mathtools, comment, graphicx, fancyhdr, color, setspace, comment, multicol, hyperref, tabu}
\usepackage[letterpaper,margin=1in,includehead=true]{geometry}
\pagestyle{fancy}
\usepackage{tikz,tikz-3dplot,pgfplots}

\DeclarePairedDelimiter\abs{\lvert}{\rvert}
\newcommand*\Eval[1]{\left.#1\right\rvert}

\setlength{\headheight}{15pt}%needed to remove fancyhdr error--not crucial%
%\pgfplotsset{compat=1.9}%Removes pgfplots backward compatibility error--not crucial%

\newcounter{mycounter}  
\newenvironment{noindlist}
 {\begin{list}{\arabic{mycounter}.~~}{\usecounter{mycounter} \labelsep=0em \labelwidth=0em \leftmargin=0em \itemindent=0em}}
 {\end{list}}

\newcommand{\unenumerate}[1]{\setcounter{saveenum}{\value{enumi}}\end{enumerate}
	\noindent #1 
	\begin{enumerate} \setcounter{enumi}{\value{saveenum}}}

\newcommand{\hop}[3]{\lim_{x\rightarrow #1}\frac{#2}{#3}\stackrel{\text{H}}{=}\lim_{x\rightarrow 0}\frac{\frac{d}{dx} #2}{\frac{d}{dx} #3}}

\newcounter{saveenum}

\def\ds{\displaystyle}

%To print solutions, use \solutionstrue; to hide solutions, use \solutionsfalse.
%\sol takes two arguments. #1 is the vertical length. #2 is the text.

\newif\ifsolutions
\solutionsfalse
\ifsolutions
    \newcommand{\sol}[2]{\begin{minipage}[c][#1]{\linewidth}{\textcolor{blue}{\textbf{Solution:}}\quad \textcolor{blue}{#2}}\end{minipage}}
    \newcommand{\opsol}[1]{#1}
    \newcommand{\tblsol}[1]{\textcolor{blue}{#1}}
\else
    \newcommand{\sol}[2]{\begin{minipage}[c][#1]{\linewidth}{\vfill}\end{minipage}}
    \newcommand{\opsol}[1]{0}
    \newcommand{\tblsol}[1]{\textcolor{white}{#1}}
\fi

\begin{document}
\lhead{Math 1300: Calculus I}
\rhead{\bf Definite Integrals and the FTC} 

\begin{enumerate}
\item  Review and Warm-up: The graph of $f$ is shown below.  Calculate exactly each of the definite integrals that follow.

\begin{center}
\begin{tikzpicture}
\begin{axis}[
   	xmin=0, xmax=8,
	ymin=-6, ymax=6,
	xtick={0,...,8},  
	ytick={-6,-3,0,3,6},
	major tick length={0},
	line width=1pt,
 	axis lines=center, height=2 in, grid=major,
	]
    \addplot [smooth, ultra thick, samples=9, domain=0:2] {6};
    \addplot [smooth, ultra thick, samples=9, domain=2:4] {18-6*x};
    \addplot [smooth, ultra thick, samples=9, domain=4:5] {-6};
    \addplot [smooth, ultra thick, samples=9, domain=5:8] {2*x-16};
\end{axis}
\end{tikzpicture}
\end{center}

\begin{multicols}{2}
\begin{enumerate}
\item $\int_0^2 f(x)\,dx=$ \sol{0in}{12}
\item $\int_0^3 f(x)\,dx=$ \sol{0in}{15}
\item $\int_4^5 f(x)\,dx=$ \sol{0in}{-6}
\item $\int_5^8 f(x)\,dx=$ \sol{0in}{-9}
\item $\int_2^4 f(x)\,dx=$ \sol{0in}{0}
\item $\int_0^8 f(x)\,dx=$ \sol{0in}{-3}
\end{enumerate}
\end{multicols}

\item  Let $s(t)$ be the position, in feet, of a car along a straight east/west highway at time $t$ seconds.  Positive values of $s$ indicate eastward displacement of the car from home, and negative values indicate westward displacement. At $t=0$ the car is at home.  Let $v(t)$ represent the velocity of this same car, in feet per second, at time $t$ seconds (see graph below).
\begin{multicols}{2}
\begin{tikzpicture}
\begin{axis}[
   	axis lines=center,
   	x label style={at={(axis description cs:1.2,0.6)},anchor=north},
    y label style={at={(axis description cs:-.17,.5)},rotate=90,anchor=south},
    xlabel={\small time (sec)},
 	ylabel={\small velocity (ft/sec)},
   	xmin=0, xmax=40,
	ymin=-60, ymax=60,
	xtick={0,10,...,40},  
	ytick={-60,-30,0,30,60},
	major tick length={0},
	line width=1pt,
 	height=2 in, grid=major,
	]
    \addplot [smooth, ultra thick, samples=1000, domain=0:10] {60};
    \addplot [smooth, ultra thick, samples=1000, domain=10:30] {120-6*x};
    \addplot [smooth, ultra thick, samples=1000, domain=30:40] {-60};
\end{axis}
\end{tikzpicture}

\begin{tikzpicture}
\begin{axis}[
   	axis lines=center,
   	x label style={at={(axis description cs:1.22,0.12)},anchor=north},
    y label style={at={(axis description cs:-.17,.5)},rotate=90,anchor=south},
    xlabel={\small time (sec)},
 	ylabel={\small position (ft)},
   	xmin=0, xmax=40,
	ymin=0, ymax=1000,
	major tick length={0},
	line width=1pt,
 	height=2 in, grid=major,
	]
    \addplot [opacity=\opsol{1}, blue, smooth, ultra thick, samples=9, domain=0:10] {60*x};
    \addplot [opacity=\opsol{1}, blue, ultra thick, samples=9, domain=10:30] {-3*x^2+120*x-300};
    \addplot [opacity=\opsol{1}, blue, ultra thick, samples=9, domain=30:40] {-60*x+2400};
\end{axis}
\end{tikzpicture}
\end{multicols}

\begin{enumerate}
\item Write a definite integral representing each of the following:\\\\
$s(10) =$ \sol{.3 in}{$\int_0^{10} v(t)\,dt$} \\
$s(30) =$ \sol{.3 in}{$\int_0^{30} v(t)\,dt$} \\
$s(t) = $ \sol{.3 in}{$\int_0^t v(u)\,du$} \\
Now use these integrals and the velocity graph to help you fill in the chart below:\\\\
\begin{tabular}{|c||c|c|c|c|c|}
\hline
$t$&0&10&20&30&40\\
\hline
$s(t)$&0&\tblsol{600}&\tblsol{900}&\tblsol{600}&\tblsol{0}\\
\hline
\end{tabular} \\

\item Use these values to help you plot the position function.

\newpage
\item  Fill in the chart below:\\
{\tabulinesep=1.2mm
\begin{tabu}{| l | l |}
\hline
Definite integral of velocity \text{\hspace{1in}}&Change in position\text{\hspace{1in}}\\ \hline
$\displaystyle\int_0^{10} v(t)dt=\tblsol{600}$\text{\hspace{.7in}} & $s(10)-s(0)=\tblsol{600-0}$\text{\hspace{.7in}}\\ \hline
$\displaystyle\int_{10}^{20} v(t)dt=\tblsol{300}$\text{\hspace{.7in}} & $s(20)-s(10)=\tblsol{900-600}$\text{\hspace{.7in}}\\ \hline
$\displaystyle\int_0^{40} v(t)dt=\tblsol{0}$\text{\hspace{.7in}} & $s(40)-s(0)=\tblsol{0-0}$\text{\hspace{.7in}}\\ \hline
\end{tabu}}

\item Why do these two columns give the same answers?

\sol{.25 in}{The definite integral of a velocity function with respect to time represents a change in position over the same interval of time.}

\end{enumerate}

\item  The following data is from the U.S. Bureau of Economic Analysis.  It shows the rate of change $r(t)$ (in dollars per month) of per capita personal income, where $t$ is the number of months after January 1, 2012.\\\\
\begin{tabular}{|c||c|c|c|c|c|c|c|}
\hline
$t$ (months) & 0&2&4&6&8&10&12\\ \hline
$r(t)$ (dollars per month) & 154 & 17& 10 & 79 & 278 & -432 & 290 \\ \hline
\end{tabular}
\\\\
Use left-hand Riemann sums to estimate the total change in personal income during 2012.

\sol{1.8 in}{
\[\ds \sum_{i=1}^n r(a+i\Delta t)\cdot \Delta t \quad \text{where, in this case, $a=0$ and $\Delta t=2$}\]
\vspace{-.2in}
\begin{align*}
&=r(0)\cdot 2 + r(2)\cdot 2 + r(4)\cdot 2 + r(6)\cdot 2 + r(8)\cdot 2 + r(10)\cdot 2\\
&=2\left(r(0)+r(2)+r(4)+r(6)+r(8)+r(10)\right)\\
&=2\left(154+17+10+79+278-432\right)\\
&=212
\end{align*}
}
\vspace{-.3 in}
\item A can of soda is put into a refrigerator to cool.  The temperature of the soda is given by $F(t)$. The {\bf rate} at which the temperature of the soda is changing is given by
\[F'(t) = f(t) = -25e^{-2t} \text{ (in degrees Fahrenheit per hour)}\]
\begin{enumerate}
\item  Find the rate at which the soda is cooling after 0, 1, and 2 hours.  Then use this information to estimate the temperature of the soda after 3 hours if the soda was $60^\circ F$ when it was placed in the refrigerator.

\sol{1.75 in}{We can use another left-hand Riemann sum with $\Delta t=1$ hr to find the \emph{change} in temperature:
\begin{align*}
\Delta F&=f(0) \cdot 1 + f(1) \cdot 1+f(2) \cdot 1\\
\Delta F&=-25e^{-2\cdot 0}+-25e^{-2\cdot 1}+-25e^{-2\cdot 2}\\
\Delta F&\approx -28.84
\end{align*} 
If the temperature changed by $\approx -28.84^\circ F$ over three hours, and started at $60^\circ F$, then the final temperature is $\approx 31.16^\circ F$. 
}
\newpage
\item  Now we will find the exact temperature of the soda after three hours have passed. 
\begin{enumerate}
\item Find an antiderivative of $f(t)$, that is, find a function $G(t)$ such that $F'(t)=f(t)=-25e^{-2t}$. \\(Hint: take a couple of derivatives of $f(t)$ and try to find a pattern.)

\sol{2 in}{
\[f'(t)=50e^{-2t} \text{ and } f''(t)=-100e^{-2t}\]
It looks like we gain a factor of $-2$ for every derivative we take, so our anti-derivative should divide by $-2$. Then any function of the form $\ds G(t)=\frac{-25}{-2}e^{-2t}+C=12.5e^{-2t}+C$ will be an antiderivative of $f(x)$. Since the question only asked for one such function, we'll use $G(t)=12.5e^{-2t}$.}

\item The Fundamental Theorem of Calculus (the Evaluation Theorem) tells us that $\ds \int_a^b f(t)\, dt=F(b)-F(a)$. Use this theorem and the function you found in the last step to find the temperature of the soda after 3 hours have passed. 

\sol{3 in}{
Evaluating the integral gives us:
\[\int_0^3 f(t)\,dt=12.5e^{-6}-12.5=F(3)-F(0)\]
Substituting the evaluated integral and the given initial temperature $F(0)=60$, we have:
\[12.5e^{-6}-12.5=F(3)-60\]
Solving for F(3)
\[F(3)\approx 47.531^\circ F\]
}
\end{enumerate}

\item Why do you think your estimate in part (a) is so far off?

\sol{.5in}{$f(t)$ changes very quickly between 0 and 3. We should have used a smaller $\Delta t$  }

\end{enumerate}


\newpage
\item 
\begin{enumerate}
\item Write an integral which represents the area between $f(x) = x^4$ and the $x$-axis, between $x=0$ and $x=2$.

\sol{.7 in}{\[\int_0^2 x^4\,dx\]}

\item Evaluate this integral using the Fundamental Theorem of Calculus (the Evaluation Theorem).

\sol{1.3 in}{Any function of the form $\ds \frac{x^5}{5}+C$ is an antiderivative of $f(x)=x^4$, thus $\ds F(x)=\frac{x^5}{5}$ is an example of an antiderivative of $f(x)=x^4$, and so
\[\int_0^2 x^4\,dx=F(2)-F(0)=\frac{2^5}{5}-\frac{0^5}{5}=6.4\]
}

\end{enumerate}
\item  
\begin{enumerate}
\item Using the Fundamental Theorem of Calculus (as in the last problem), evaluate $\ds \int_0^\pi \cos(x)\,dx$.

\sol{1.2 in}{$F(x)=\sin(x)$ is an antiderivative of $f(x)=\cos(x)$, and so
\[\int_0^\pi \cos(x)\,dx=F(\pi)-F(0)=\sin(\pi)-\sin(0)=0\]
}

\item Show the area represented by the integral in part (a) on the graph.

\begin{center}
\begin{tikzpicture}
\begin{axis}[
   	xmin=0, xmax=3.2,
	ymin=-1.05, ymax=1.05,
	xtick={1.5708,3.14159}, 
	xticklabels={$\frac{\pi}{2}$,$\pi$}, 
	ytick={-1,-0.5,0.5,1},
	major tick length={0},
	line width=1pt,
 	axis lines=center, height=2 in, grid=major
	]
	\addplot [opacity=\opsol{0.5}, fill=gray, draw=none, domain=0:1.5708] {cos(deg(x))} \closedcycle;
	\addplot [opacity=\opsol{0.5}, fill=red, draw=none, domain=1.5708:3.14159] {cos(deg(x))} \closedcycle;
    \addplot [opacity=\opsol{1}, blue, smooth, very thick, samples=1000, domain=0:3.2] {cos(deg(x))};
\end{axis}
\end{tikzpicture}
\end{center}

\end{enumerate}

\item  Evaluate $\ds \int_0^1\frac{1}{1+x^2}\,dx$ without using technology.

\sol{1.55 in}{$F(x)=\arctan(x)$ is an antiderivative of $\ds f(x)=\frac{1}{1+x^2}$, and so
\[\int_0^\pi \frac{1}{1+x^2}\,dx=F(1)-F(0)=\arctan(1)-\arctan(0)=\frac{\pi}{4}\]
}

\end{enumerate}
\end{document}
