\documentclass[letterpaper,11pt]{article}
\usepackage[margin=.5in]{geometry}
\usepackage{calc}

\usepackage{amsmath, mathtools, comment, graphicx, fancyhdr, color, setspace, multicol, enumerate, hyperref, tabu}
\usepackage{tikz,tikz-3dplot,pgfplots}

\usepackage{amssymb,latexsym,epsfig,amscd,arydshln}
			
\parskip = 0.2in

\def\ds{\displaystyle}

\setlength{\parindent}{0pt}

\newcounter{saveenum}

\newif\ifsolutions
%\solutionstrue
\solutionsfalse
\ifsolutions
    \newcommand{\opsol}{1}
    \newcommand{\usol}[1]{\underline{\hspace{.1 in} \textcolor{blue}{#1} \hspace{.1 in}}}
    \newcommand{\sol}[1]{\textcolor{blue}{#1}}
\else
    \newcommand{\opsol}{0}
    \newcommand{\usol}[1]{\underline{\hspace{.1 in} \textcolor{white}{#1} \hspace{.1 in}}}
    \newcommand{\sol}[1]{\textcolor{white}{#1}}
\fi



\begin{document}
\pagenumbering{gobble}


{\large \textbf{Area accumulation functions and the FTC, graphical perspective}} \\

\begin{enumerate}
\item Example:

\vspace{-.2 in}
\begin{center}
\begin{tikzpicture}
\begin{axis}[
   	xmin=-2, xmax=6,
	ymin=-1.2, ymax=2.2,
	major tick length={0},
	line width=1pt,
 	axis lines=center, height=2 in, width=4 in, grid=major, ylabel={$f$},
 	restrict y to domain=-1.2:3
	]
	\addplot [black, smooth, very thick] plot coordinates {(-2,1)(1,1)};
	\addplot [black, smooth, very thick] plot coordinates {(1,1)(2,2)};
	\addplot [black, smooth, very thick] plot coordinates {(2,2)(3,2)};
	\addplot [black, smooth, very thick] plot coordinates {(3,2)(6,-1)};
\end{axis}
\end{tikzpicture}
\end{center}

\begin{enumerate}

\vspace{-.1 in}
\item $\ds F(x)=\int_0^x f(t) \,dt$. $F'(x)=\usol{f(x)}$

\item Evaluate the following:
\begin{multicols}{3}
$F(0)= \sol{0}$  \\
$F'(0)= \sol{1} $ \\
$F''(0)= \sol{0} $ \\

$F(2)= \sol{2.5} $  \\
$F'(2)= \sol{2} $  \\
$F''(2)= \sol{DNE} $  \\

$F(5)= \sol{6.5} $  \\
$F'(5)= \sol{0} $  \\
$F''(5)= \sol{-1} $  \\
\end{multicols}

\vspace{-.2 in}
\item Find a formula for $F'(x)$ between $x=1$ and $x=2$.

\sol{$F'(x)=x$ for $1 \leq x \leq 2$.}
\vfill

\item Find a formula for $F(x)$ between $x=1$ and $x=2$

\sol{$\ds F(x)=\frac{x^2}{2}+\frac{1}{2}$ for $1 \leq x \leq 2$.}
\vfill

\item Sketch $F(x)$.

\vspace{-.2 in}
\begin{center}
\begin{tikzpicture}
\begin{axis}[
   	xmin=-2, xmax=6,
	ymin=-2, ymax=7.2,
	major tick length={0},
	line width=1pt,
 	axis lines=center, height=2 in, width=4 in, grid=major,
 	restrict y to domain=-2:7.2
	]
	\addplot [opacity=\opsol, blue, very thick, domain=-2:1] {x};
	\addplot [opacity=\opsol, blue, smooth, very thick, domain=1:2] {x^2/2+.5};
	\addplot [opacity=\opsol, blue, smooth, very thick, domain=2:3] {2*x-1.5};
	\addplot [opacity=\opsol, blue, smooth, very thick, domain=3:6] {-x^2/2+5*x-6};
\end{axis}
\end{tikzpicture}
\end{center}

\item Sketch $F'(x)$.

\vspace{-.2 in}
\begin{center}
\begin{tikzpicture}
\begin{axis}[
   	xmin=-2, xmax=6,
	ymin=-1.2, ymax=2.2,
	major tick length={0},
	line width=1pt,
 	axis lines=center, height=2 in, width=4 in, grid=major,
 	restrict y to domain=-1.2:3
	]
	\addplot [opacity=\opsol, blue, smooth, very thick] plot coordinates {(-2,1)(1,1)};
	\addplot [opacity=\opsol, blue, smooth, very thick] plot coordinates {(1,1)(2,2)};
	\addplot [opacity=\opsol, blue, smooth, very thick] plot coordinates {(2,2)(3,2)};
	\addplot [opacity=\opsol, blue, smooth, very thick] plot coordinates {(3,2)(6,-1)};
\end{axis}
\end{tikzpicture}
\end{center}
\end{enumerate}

\newpage

\item Example:

\vspace{-.2 in}
\begin{center}
\begin{tikzpicture}
\begin{axis}[
   	xmin=-2, xmax=8,
	ymin=-1.2, ymax=1.2,
	major tick length={0},
	line width=1pt,
 	axis lines=center, height=2 in, width=5 in, grid=major, ylabel={$h$},
	]
	\addplot [black, smooth, very thick, domain=-2:3] {-.25*(x-1)^2+1};
	\addplot [black, smooth, very thick, domain=3:8] {.25*(x-5)^2-1};
\end{axis}
\end{tikzpicture}
\end{center}

\vspace{-.1in}
\begin{enumerate}
\item $\ds H_1(x)=\int_0^x h(t) \,dt$. $H_1'(x)=\usol{h(x)}$ \\

At what $x$ values does $H_1(x)$ have:

{\tabulinesep=1.4mm
\begin{tabu}{ r  l }
 critical points & \usol{$x=-1, 3, 7$} \\
 local minima & \usol{$x=-1, 7$} \\
 local maxima & \usol{$x=3$} \\
 inflection pts & \usol{$x=1, 5$} \\
\end{tabu}}

\vspace{.3 in}

\item Now $\ds H_2(x)=\int_{-1}^x h(t) \,dt$. \\

At what $x$ values does $H_2(x)$ have:

{\tabulinesep=1.4mm
\begin{tabu}{ r  l }
 critical points & \usol{$x=-1, 3, 7$} \\
 local minima & \usol{$x=-1, 7$} \\
 local maxima & \usol{$x=3$} \\
 inflection pts & \usol{$x=1, 5$} \\
\end{tabu}}

\vspace{.3 in}

\item What is the difference between $H_1(x)$ and $H_2(x)$?

\sol{$H_1(x)$ and $H_2(x)$ have the same shape, but $H_2(x)$ is shifted up from $H_1(x)$ by the area accumulated under the curve between $x=-1$ and $x=0$.}

\end{enumerate}
\end{enumerate}
\end{document}
