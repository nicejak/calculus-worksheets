\documentclass[letterpaper,11pt]{article}
\usepackage{amsmath}
\usepackage[letterpaper,margin=.8in,includehead=true]{geometry}
\usepackage{comment}
\usepackage{graphicx}
\usepackage{fancyhdr}
\pagestyle{fancy}
\usepackage{color}
\usepackage{setspace}
\usepackage{hyperref}
\usepackage{tabularx}

\hypersetup{colorlinks, linkcolor=red}

\newcounter{mycounter}  
\newenvironment{noindlist}
 {\begin{list}{\arabic{mycounter}.~~}{\usecounter{mycounter} \labelsep=0em \labelwidth=0em \leftmargin=0em \itemindent=0em}}
 {\end{list}}

%To print solutions, use \solutionstrue
%To repress solutions, use \solutionsfalse
%\sol takes two arguments. #1 is the vertical length. #2 is the text.

\def\imagetop#1{\vtop{\null\hbox{#1}}} % this is to make figures line up properly with text

\newif\ifsolutions
%\solutionstrue
\solutionsfalse
\ifsolutions
    \newcommand{\sol}[2]{\begin{minipage}[c][#1]{\linewidth}{\textcolor{red}{\textbf{Solution:}}\quad \textcolor{red}{#2}}\end{minipage}}
    \newcommand{\fsol}[2]{\includegraphics[scale=#1]{#2sol}}
\else
    \newcommand{\sol}[2]{\begin{minipage}[c][#1]{\linewidth}{\vfill}\end{minipage}}
    \newcommand{\fsol}[2]{\includegraphics[scale=#1]{#2}}
\fi

\newcommand{\unenumerate}[1]{\setcounter{saveenum}{\value{enumi}}\end{enumerate}
	\noindent #1 
	\begin{enumerate} \setcounter{enumi}{\value{saveenum}}}

\newcounter{saveenum}

\def\ds{\displaystyle}

\begin{document}
\lhead{Math 1300: Calculus I}
\chead{}
\rhead{\bf Project: Graphing using signs} 

\begin{enumerate}
\item  In the graph of $f(x)$ below, number lines are used to mark where $f(x)$ is zero/positive/negative/undefined, where $f'(x)$ is zero/positive/negative/undefined,
and where $f''(x)$ is zero/positive/negative/undefined.  Closed circles on the number lines indicate a zero-value, and open-circles indicate an undefined value.
\begin{center}\hskip-.4in
\includegraphics[scale=.35]{fig1.png}
\end{center}
\begin{enumerate}
\item  What do the closed circles on the number line for $f(x)$ correspond to on the graph of $f(x)$?

\sol{.4 in}{This is where $f(x)$ has x-intercepts (zeroes).}

\item  How  does each $+$ or $-$ sign on the number line for $f(x)$ relate to the graph?

\sol{.4 in}{The $+$ and $-$ signs indicate where $f(x)$ is positive or negative.}

\item  What does the closed circle at $x=-1$ on the number line for $f'(x)$ correspond to on the graph of $f(x)$?

\sol{.4 in}{This is where $f(x)$ has a horizontal tangent line.}

\item  What does the open circle at $x=2$ on the number line for $f'(x)$ correspond to on the graph of $f(x)$?

\sol{.4 in}{This is where $f(x)$ has a vertical tangent line, so its slope, and $f'(x)$, is undefined here.}

\item   How does each $+$ or $-$ sign on the number line for $f'(x)$ relate to the graph?

\sol{.4 in}{They indicate where $f'(x)$ is positive or negative. $f'(x)$ is negative where $f(x)$ is decreasing and positive where $f(x)$ is increasing.}

\item What does the open circle at $x=2$ on the number line for $f''(x)$ correspond to on the graph of $f(x)$?

\sol{.4 in}{Since $f'(x)$ is undefined here, so is $f''(x)$.}

\item How does each $+$ or $-$ sign on the number line for $f''(x)$ relate to the graph?

\sol{.4 in}{They indicate where $f''(x)$ is positive or negative. $f''(x)$ is negative where $f(x)$ is concave down and positive where $f(x)$ is concave up.}

\end{enumerate}


\newpage
\item  For the graph of $f(x)$ shown below, fill in the number lines for $f(x)$, $f'(x)$ and $f''(x)$, marking closed circles where there is a zero, marking open circles for undefined points, and marking $+$ and $-$ signs on each interval to show positive/negativeness.
\begin{center}\hskip-.4in
\fsol{.35}{fig2a}
\end{center}

\item  Draw a graph of $f(x)$ that fits the information shown in the number lines.
\begin{center}\hskip-.4in
\fsol{.35}{fig2b}
\end{center}
\vspace{1in}

\newpage

\item  The middle graph drawn below shows $f'(x)$.  Using the principles you learned in the previous problem, draw a possible graph of $f(x)$ above it, and a graph of $f''(x)$ below it. (If you are stuck try drawing the graph of $f''(x)$ first.)

\vspace{.4 in}

\begin{tabular}{ll}
$f(x):$
&
\imagetop{\fsol{1}{fig5a}}\\

\vspace{.4 in}\\

$f'(x):$&
\imagetop{\includegraphics[scale=1]{fig5b}}\\

\vspace{.4 in}\\

$f''(x):$&
\imagetop{\fsol{1}{fig5c}}\\

\end{tabular}

\newpage

\item  This problem investigates the derivative of the absolute value function.  Recall that we define the absolute value as:
\begin{displaymath}
   |x| = \left\{
     \begin{array}{ll}
       x & \mbox{ if } x\geq 0,\\
	-x & \mbox{ if } x<0.
     \end{array}
   \right.
\end{displaymath}
	\begin{enumerate}
	\item In the space provided, draw a graph of the function $f(x)=|x|$.
\begin{center}
\fsol{.6}{fig4a}
\end{center} 

\item Using your graph from part (a), and your understanding of the derivative as the rate of change/slope of the tangent line, find the derivative function $f'(x)$ of the above function $f(x)=|x|$, for $x$ not equal to $0$  (fill in the blanks):
\begin{displaymath}
  f'(x) = \left\{
     \begin{array}{ll}
      \underline{\hskip1in} & \mbox{ if } x> 0, \sol{0 in}{1 if $x>0$}\\\\
	 \underline{\hskip1in}& \mbox{ if } x<0. \sol{0 in}{-1 if $x<0$}
     \end{array}
   \right.
\end{displaymath}.

\item But what about $f'(0)$?  
	It is not so clear from the picture even how to draw a tangent line to the function at the origin.
	So let's try to compute $f'(0)$ by first looking at the corresponding lefthand and righthand limits of the difference quotient. 
\begin{enumerate}
\item Compute $\lim_{h\rightarrow 0^+}\frac{f(0+h)-f(0)}{h}$.  Hint:  use the piecewise definition of $f(x)$ given above.

\sol{1 in}{
$$\lim_{h\rightarrow 0^+}\frac{f(0+h)-f(0)}{h}=\lim_{h\rightarrow 0^+}\frac{|h|-|0|}{h}=\lim_{h\rightarrow 0^+}\frac{h-0}{h}=\lim_{h\rightarrow 0^+}1=1$$(since $h\to 0^+$ means $h$ is positive, so $|h|=h$).
}

\item  Compute $\lim_{h\rightarrow 0^-}\frac{f(0+h)-f(0)}{h}$. 

\sol{1in}{
$$\lim_{h\rightarrow 0^-}\frac{f(0+h)-f(0)}{h}=\lim_{h\rightarrow 0^-}\frac{|h|-|0|}{h}=\lim_{h\rightarrow 0^-}\frac{-h-0}{h}=\lim_{h\rightarrow 0^-}-1=-1$$(since $h\to 0^-$ means $h$ is negative, so $|h|=-h$).
}

\item What do your answers to parts (i) and (ii) tell you about $f'(0)$?  Please explain.

\sol{1 in}{
Since the righthand limit $\lim_{h\rightarrow 0^+}\frac{f(0+h)-f(0)}{h}$ does NOT equal the lefthand limit $\lim_{h\rightarrow 0^-}\frac{f(0+h)-f(0)}{h}$, the (two-sided) limit $\lim_{h\rightarrow 0}\frac{f(0+h)-f(0)}{h}$ does not exist.  But this  (two-sided) limit is $f'(0)$, so $f'(0)$ does not exist.
}
\end{enumerate}
\end{enumerate}

\newpage
\item  Using what you've learned above, sketch the graph of a {\it continuous} function  $g(x)$  such that $g(x)$ is not differentiable at $x=-1$, $x=2$, nor $x=3$. 
\begin{center}
\fsol{.6}{fig4d}
\end{center}
\item Create a piecewise function where one piece is a quadratic function and the other piece is a linear function which is continuous everywhere but not differentiable at $x=0$.
\begin{displaymath}
  f(x) = \left\{
     \begin{array}{ll}
      \underline{\hskip1in} & \mbox{ if } x\ge 0, \\ \\
	 \underline{\hskip1in}& \mbox{ if } x<0.
     \end{array}
   \right.
\end{displaymath}.

\sol{.8 in}{
\begin{displaymath}
  f(x) = \left\{
     \begin{array}{ll}
      x^2 & \mbox{ if } x\ge0 \\ \\
      x & \mbox{ if } x<0.
     \end{array}
   \right.
\end{displaymath}.
}

\item Create a piecewise function where one piece is a quadratic function and the other piece is a linear function which is continuous and differentiable everywhere.
\begin{displaymath}
  f(x) = \left\{
     \begin{array}{ll}
      \underline{\hskip1in} & \mbox{ if } x> 0, \\ \\
	 \underline{\hskip1in}& \mbox{ if } x \leq 0.
     \end{array}
   \right.
\end{displaymath}.

\sol{1.5 in}{
\begin{displaymath}
  f(x) = \left\{
     \begin{array}{ll}
      (x+1)^2-1 & \mbox{ if } x\ge0 \\ \\
      2x & \mbox{ if } x<0.
     \end{array}
   \right.
\end{displaymath}.
\\
Note, because $\dfrac{d}{dx} \bigg|_{x=0} (x+1)^2-1=2(0+1)=2$ and $\dfrac{d}{dx} \bigg|_{x=0} 2x=2$, so 
the right and left difference quotient limits agree at $x=0$ and therefore the function is differentiable at $x=0$.
}

\end{enumerate}



\end{document}
