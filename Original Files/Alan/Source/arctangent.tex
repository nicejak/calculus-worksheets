\documentclass[letterpaper,10pt]{article}
\usepackage[letterpaper,margin=.25in,includehead=true]{geometry}
\usepackage{amsmath, amssymb, fullpage, graphicx, calc}
\usepackage{tikz,tikz-3dplot,pgfplots}
\usepackage{color}
\def\ds{\displaystyle}

\openup1\jot
\def\scrr{$\underline{\hskip.4in}$}
\pagestyle{empty}
\openup1\jot
\addtolength{\textheight}{.25in}
\addtolength{\topmargin}{-.25in}

\newif\ifsolutions
\solutionsfalse
\ifsolutions
    \newcommand{\opsol}{1}
    \newcommand{\sol}[1]{\textcolor{red}{#1}}
\else
    \newcommand{\opsol}{0}
    \newcommand{\sol}[1]{\textcolor{white}{#1}}
\fi
 
\begin{document}
 
\centerline{\large {\bf Project: the  arctangent function }}
 
\hrule
\bigskip

\begin{enumerate}

\item On the axes below is a graph of the function $y=f(x)=\tan x$, for $-\frac{\pi}{2}< x< \frac{\pi}{2}$.

\begin{center}
\begin{tikzpicture}
\begin{axis}[
   	xmin=-1.6, xmax=1.6,
	xtick={-1.5708, -0.7854, 0.7854, 1.5708}, 
	ymin=-8, ymax=8, ytick={-6,-4,...,6},
	xticklabels={$-\frac{\pi}{2}$,$-\frac{\pi}{4}$,$\frac{\pi}{4}$, $\frac{\pi}{2}$}, 
	major tick length={0},
	line width=1pt,
 	axis lines=center, height=4 in, width=1.6in,grid=major,
 	restrict y to domain=-8:8,
 	axis equal
	]
    \addplot [black, smooth, very thick, samples=299, domain=-1.6:1.6] {tan(deg(x))};
%    \addplot [dotted, smooth, thin, domain = -2.1:2.1] {x};
\end{axis}
\end{tikzpicture}
\end{center}

(As $x$ approaches $-\pi/2$ from the right, the graph of $f(x)$ shoots off towards $-\infty$; as $x$ approaches $\pi/2$ from the left, the graph of $f(x)$ shoots off towards $\infty$.)
\begin{enumerate} 
\item Explain why this function $f(x)$, when restricted to the given domain, has an inverse function $f^{-1}(x)=\arctan x$.  (Pronounced ``arctangent of $x$.'')

\sol{The above function $y=\tan(x)$ is steadily increasing on the given domain.  (This is clear from the picture. Alternatively, note that $\frac{d}{dx}\tan(x)=\sec^2(x)>0$, since the square of a nonzero number is always positive.  But a function with a positive derivative on an interval is increasing there.)   So $y=\tan(x)$ satisfies the horizontal line test on this interval, so there is only {\it one}  input $x$ for each output  $y$,  so that, if we interchange the roles of $x$ and $y$, we get only one output $y$ per input  $x$.  This is what it means to say $f(x)=\tan(x)$ has an inverse function, which we denote by $g(x)=\arctan(x)$.}

\vfill

\item Sketch the graph of $y=f^{-1}(x)=\arctan x$, on the axes below.

\begin{center}
\begin{tikzpicture}
\begin{axis}[
   	ymin=-1.6, ymax=1.6,
	ytick={-1.5708, -0.7854, 0.7854, 1.5708}, 
	xmin=-8, xmax=8, xtick={-6,-4,...,6},
	yticklabels={$-\frac{\pi}{2}$,$-\frac{\pi}{4}$,$\frac{\pi}{4}$, $\frac{\pi}{2}$}, 
	major tick length={0},
	line width=1pt,
 	axis lines=center, height=1.6 in, width=4 in, grid=major,
 	restrict x to domain=-8:8,
 	axis equal
	]
	   \addplot [opacity=\opsol, red, smooth, very thick, samples=299, domain=-8:8]{rad(atan(x))};
\end{axis}
\end{tikzpicture}
\end{center}
\end{enumerate}

\newpage

\item  Note that, because we have defined $\arctan x$ as the inverse of $\tan x$, we have: \[\tan(\arctan x)=x.\] Differentiate both sides of this equation to find a formula for the derivative of $\arctan x$. Express your answer in terms of $\sec^2(\arctan x)$. (You'll need to recall that $\displaystyle\frac{d}{dx}[\tan x]=\sec^2 x$.)

\sol{\[  \frac{d}{dx}[\tan(\arctan x)]=\frac{d}{dx}[x].\] So, by the chain rule and the facts that $\displaystyle\frac{d}{dx}[\tan x]=\sec^2 x$ and $\displaystyle\frac{d}{dx}[x]=1,$\[\sec^2(\arctan x)
\frac{d}{dx}[ \arctan x]=1.\]  Divide by $\sec^2(\arctan x)
$ to get
\[
\frac{d}{dx}[ \arctan x]=\frac{1}{\sec^2(\arctan x)}.\]}
 
\item Referring to the triangle below, explain why $\sec^2(\arctan x)=1+x^2$. 
\begin{center}
\begin{tikzpicture}[scale=.5]
    \draw (0,0) -- (5,0) -- (5,3) -- cycle;
    \draw (4.5,0) -- (4.5,.5) -- (5,.5);
    \draw[opacity=\opsol, red] (1.3,2.2) node {$\sqrt{1+x^2}$};
    \draw[opacity=\opsol, red] (1.4,0.4) node {$\theta$};
    \draw (5.5,1.5) node {$x$};
    \draw (2.7,-0.5) node {$1$};
\end{tikzpicture}
\end{center}
\sol{Consider the angle $\theta$ depicted above.  As the diagram shows, $\tan\theta=x/1=x$, so $\theta=\arctan x$.  Also, the hypotenuse of the above triangle is $\sqrt{1+x^2}$, by the Pythagorean Theorem.  So $$\sec^2(\arctan x)=\sec^2\theta=\left(\sqrt{1+x^2}/1\right)^2=1+x^2.$$}
\vfill
\item Use the results of problems 2 and 3 above to show that $\ds \frac{d}{dx}  \arctan x =\frac{1}{1+x^2}.$
\sol{By problems 2 and 3 successively,
\[\frac{d}{dx}[ \arctan x]=\frac{1}{\sec^2(\arctan x)}=\frac{1}{1+x^2}.\]}
\vfill
\end{enumerate}
  
 
  
\end{document}
 